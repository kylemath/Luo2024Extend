% LaTeX Equations extracted from PDF
% Compile with pdflatex or similar

\documentclass{article}
\usepackage{amsmath}
\usepackage{amssymb}
\usepackage{amsthm}

\begin{document}

\section*{Equations}

% Equation 1
\begin{equation}
Marginal Reputation
\end{equation}

% Equation 2
\begin{equation}
ature. For example, Schelling (1966) writes,
\end{equation}

% Equation 3
\begin{equation}
(deterministic communication mechanisms) and all linear partitions with randomization at
\end{equation}

% Equation 4
\begin{equation}
We start with a deterrence game.4 A long-run player with discount factor \delta faces a
\end{equation}

% Equation 5
\begin{equation}
d, generated with conditional probability Pr(c|C) = Pr(d|D) = p \in(0, 1), before choosing
\end{equation}

% Equation 6
\begin{equation}
Assume that g, l > 0 and x, y \in(0, 1), so that D is dominant for the short-run player and
\end{equation}

% Equation 7
\begin{equation}
What can be said about the equilibria of this game when the discount factor \delta is close
\end{equation}

% Equation 8
\begin{equation}
0-confirmed best response to (A, F).
\end{equation}

% Equation 9
\begin{equation}
1) state that, as \delta \to1, the long-run player’s payoff in any Nash equilibrium is at least her
\end{equation}

% Equation 10
\begin{equation}
Proposition 1. Let U 1(\delta) and ¯U1(\delta) be the infimum and supremum of the long-run player’s
\end{equation}

% Equation 11
\begin{equation}
(1) If x + y < 1 then lim inf\delta\to1 U 1(\delta) \geqp for all µ0 > 0. (2) If x + y > 1 then limµ0\to0 U 1(\delta) = 1 −p + py for all \delta < 1.
\end{equation}

% Equation 12
\begin{equation}
6These two conditions holds iff p > max{ 1+g+l
\end{equation}

% Equation 13
\begin{equation}
A \succF, c \succd) and is strictly submodular in the second. In the strictly supermodular case, F
\end{equation}

% Equation 14
\begin{equation}
game”) adapted from Pei (2020). There is a state \theta \in{G(ood), B(ad)}, drawn i.i.d. across
\end{equation}

% Equation 15
\begin{equation}
\theta before taking an action a1 \in{H(igh) Effort, L(ow) Effort}. Simultaneously (and having
\end{equation}

% Equation 16
\begin{equation}
a2 \in{T(rust), N(ot Trust)}. Payoffs in each state are given by the following matrices, with
\end{equation}

% Equation 17
\begin{equation}
Payoffs in State \theta = G
\end{equation}

% Equation 18
\begin{equation}
Payoffs in State \theta = B
\end{equation}

% Equation 19
\begin{equation}
the short-run player only if he believes that, with high enough probability, both \theta = G and a1 = H. For example, suppose that player 1 is the chef of a seafood restaurant, \theta is the
\end{equation}

% Equation 20
\begin{equation}
(1) If min{w, z} > 0 then lim inf\delta\to1 U 1(\delta) \geq3/2 for all µ0 > 0. (2) If max{w, z} < 0 then limµ0\to0 U 1(\delta) = 0 for all \delta < 1.
\end{equation}

% Equation 21
\begin{equation}
long-run player’s payoff is strictly supermodular in (\theta, a1) for any a2, which we show lets
\end{equation}

% Equation 22
\begin{equation}
long-run player’s payoff is strictly submodular in (\theta, a1) for any a2, which we show limits her
\end{equation}

% Equation 23
\begin{equation}
Finally, suppose the stage game is a signaling game. The state \theta \in\Theta \subsetR is drawn i.i.d. across periods. In each period, the long-run player observes \theta before taking an action a1 \inA1 \subsetR. The short-run player observes the current action a1 (but not \theta) and the history of past actions (but not past states) and then takes an action r \inR in response. Assume \Theta, A1, and R are finite, the long-run player’s payoff is given by (1 −\lambda)v(r) −\lambdaw(a1, \theta) for some functions v and w and some \lambda \in(0, 1), and w is strictly submodular. Thus, the long-run player’s preferences over the short-run player’s action r are independent of the state \theta, and the parameter \lambda measures the weight on the long-run player’s signaling cost w(a1, \theta) relative
\end{equation}

% Equation 24
\begin{equation}
standard assumption in signaling theory: for example, in Spence (1973), w(a1, \theta) = a1/\theta.10 9If one of {w, z} is positive and the other negative, the long-run player’s payoff is supermodular in (\theta, a1) for one a2 \in{T, N} and submodular for the other. Our results do not cover this case. We also note that Proposition 2 is roughly consistent with Pei’s (2020) results for the case where \theta is perfectly persistent: Pei
\end{equation}

% Equation 25
\begin{equation}
10Our analysis of signaling games remains valid for more general preferences of the form (1 −\lambda)v(a1, r) − \lambdaw(a1, \theta). That is, v can depend on a1 as well as r.
\end{equation}

% Equation 26
\begin{equation}
A strategy for the long-run player is now a mapping s1 : \Theta \to\Delta(A1). Assume that the
\end{equation}

% Equation 27
\begin{equation}
selection from supp(s1(\theta)) is monotone in \theta.
\end{equation}

% Equation 28
\begin{equation}
independently of \theta is a 0-confirmed best response to ˆs. However, we have the following. Proposition 3. If ˆs1 is monotone then lim inf\delta\to1 U 1(\delta) \geqˆv1.
\end{equation}

% Equation 29
\begin{equation}
run player’s Stackelberg strategy for the limiting payoffs where \lambda = 0 (i.e., the Bayesian
\end{equation}

% Equation 30
\begin{equation}
run player secures a payoff only slightly below her Stackelberg payoff when \lambda is small. This
\end{equation}

% Equation 31
\begin{equation}
librium payoff is bounded away from her commitment payoff for all \lambda > 0. The logic is that
\end{equation}

% Equation 32
\begin{equation}
There are three players, i \in{0, 1, 2}. Player 1 is the long-run player; players 0 and 2
\end{equation}

% Equation 33
\begin{equation}
\rho(·|ai) \in\Delta(Yi). The signals satisfy the following assumption.
\end{equation}

% Equation 34
\begin{equation}
(1) The signal y0 has a full support distribution: \rho(y0|a0) > 0 for all y0 \inY0, a0 \inA0. (2) The signal y1 statistically identifies the long-run player’s action: the |A1| vectors \rho(·|a1)a1\inA1
\end{equation}

% Equation 35
\begin{equation}
y2, drawn independently from \rho(·|a1) and \rho(·|a2), respectively, and publicly observed
\end{equation}

% Equation 36
\begin{equation}
Thus, stage game strategies for players 0 and 2 are simply mixed actions \alpha0 \in\Delta(A0) and \alpha2 \in\Delta(A2), respectively, while a stage game strategy for player 1 is a function s1 : Y0 \to\Delta(A1). Note that a strategy profile (\alpha0, s1, \alpha2) induces a joint distribution \gamma(\alpha0, s1) \in \Delta(Y0 \times A1) (independent of \alpha2) over player 1’s private signal y0 and action a1 according to \gamma(\alpha0, s1)[y0, a1] =
\end{equation}

% Equation 37
\begin{equation}
a0\inA0 \alpha0(a0)\rho(y0|a0)s1(y0)[a1],
\end{equation}

% Equation 38
\begin{equation}
and it induces a joint distribution p(\alpha0, s1, \alpha2) \in\Delta(Y1 \times Y2) over the public signals y1 and
\end{equation}

% Equation 39
\begin{equation}
p(\alpha0, s1, \alpha2)[y1, y2] =
\end{equation}

% Equation 40
\begin{equation}
a0\inA0
\end{equation}

% Equation 41
\begin{equation}
y0\inY0
\end{equation}

% Equation 42
\begin{equation}
a1\inA1
\end{equation}

% Equation 43
\begin{equation}
a2\inA2 \alpha0(a0)\rho(y0|a0)s1(y0)[a1]\rho(y1|a1)\alpha2(a2)\rho(y2|a2).
\end{equation}

% Equation 44
\begin{equation}
Assumption 1, they do not identify her strategy s1 (whenever |Y0|\geq2), because y0 is player 1’s private information. Throughout the paper, for any joint distribution \chi \in\Delta(X1 \times X2) over a product set X1 \times X2, \piXi(\chi) denotes its marginal on Xi. With slight abuse of notation, we also denote the marginal of \gamma(\alpha0, s1) over Y0 (which depends only on \alpha0) by \rho(\alpha0) = \piY0(\gamma(\alpha0, s1)), and we denote its marginal over A1 by ϕ(\alpha0, s1) = \piA1(\gamma(\alpha0, s1)).
\end{equation}

% Equation 45
\begin{equation}
A2 \toR for players i \in{1, 2}. Thus, player 0’s payoff depends on his own action and player
\end{equation}

% Equation 46
\begin{equation}
and is satisfied in our applications.11 Finally, we write ui(\alpha0, s1, \alpha2) for player i’s expected payoff at stage-game strategy profile (\alpha0, s1, \alpha2), and we let u1 = mina0,s1,a2 u1(a0, s1, a2) and ¯u1 = maxa0,s1,a2 u1(a0, s1, a2).
\end{equation}

% Equation 47
\begin{equation}
Nature). Communication games fit by making A0 a singleton; viewing \rho(y0) as the prior distribution of a payoff-relevant state y0; letting \rho(y1|a1) = 1({y1 = a1}) (so a1 is perfectly
\end{equation}

% Equation 48
\begin{equation}
that u1 and u2 depend on a2 only through the induced response a2(a1) \inR.
\end{equation}

% Equation 49
\begin{equation}
strategies (\alpha0, \alpha2) satisfying supp(\alpha0) \subsetargmax a0\inA0 u0(a0, s1)
\end{equation}

% Equation 50
\begin{equation}
supp(\alpha2) \subsetargmax a2\inA2 u2(\alpha0, s1, a2), so that player 0 best responds to s1, and player 2 best responds to \alpha0 and s1. With this
\end{equation}

% Equation 51
\begin{equation}
1 = sup s1\in\Delta(A1)Y0 inf (\alpha0,\alpha2)\inB(s1) u1(\alpha0, s1, \alpha2).
\end{equation}

% Equation 52
\begin{equation}
V (s1) = inf (\alpha0,\alpha2)\inB(s1) u1(\alpha0, s1, \alpha2).12
\end{equation}

% Equation 53
\begin{equation}
Definition 1. For any long-run player strategy s1 and any \eta \geq0, a short-run player strategy (\alpha0, \alpha2) \in\Delta(A0) \times \Delta(A2) is an \eta-confirmed best response to s1 if there exists s′
\end{equation}

% Equation 54
\begin{equation}
(1) (\alpha0, \alpha2) \inB(s′
\end{equation}

% Equation 55
\begin{equation}
(2) ||p(\alpha0, s1, \alpha2) −p(\alpha0, s′ 1, \alpha2)||\leq\eta.
\end{equation}

% Equation 56
\begin{equation}
1 that satisfies these conditions, we say that it \eta-confirms (\alpha0, \alpha2) against s1.
\end{equation}

% Equation 57
\begin{equation}
B\eta(s1) \supsetB(s1) for all \eta′ \geq\eta \geq0, where B0(s1) = B(s1) if s1 is identified (which, again, is not the case in our model whenever |Y0|\geq2), and B1(s1) is the set of all short-run player
\end{equation}

% Equation 58
\begin{equation}
as \eta \to0, by upper hemi-continuity of the short-run players’ best-response correspondences.
\end{equation}

% Equation 59
\begin{equation}
V0(s1) = inf (\alpha0,\alpha2)\inB0(s1) u1(\alpha0, s1, \alpha2). Note that V (s1) \geqV0(s1) for each s1, since B(s1) \subsetB0(s1).
\end{equation}

% Equation 60
\begin{equation}
with discount factor \delta \in(0, 1), while players 0 and 2 are short-lived and take myopic best
\end{equation}

% Equation 61
\begin{equation}
ht = (y1,t′, y2,t′)t−1 t′=0 \in(Y1 \times Y2)t. Let Ht be the set of period-t (public) histories, H = S
\end{equation}

% Equation 62
\begin{equation}
the set of all finite histories, and H\infty= (Y1 \times Y2)\inftythe set of infinite histories. A repeated game strategy \sigmai for player i maps public histories to stage game strategies: formally, \sigmai is a function from H to \Delta(Ai) for i \in{0, 2}, and is a function from H to \Delta(A1)Y0 for i = 1.14 The long-run player’s type, denoted \omega \inΩ, is either rational (\omega = \omegaR) or is one of a
\end{equation}

% Equation 63
\begin{equation}
s1 \in(\Delta(A1))Y0, where type \omegas1 plays s1 in every period.15 The type \omega is drawn according to a full-support prior µ0 \in\Delta(Ω) at the start of the game and is perfectly persistent. We study U 1(\delta) and ¯U1(\delta), the infimum and supremum of the long-run player’s payoff in any Nash equilibrium (\sigma∗ 0, \sigma∗ 1, \sigma∗
\end{equation}

% Equation 64
\begin{equation}
the paper, \sigma∗
\end{equation}

% Equation 65
\begin{equation}
strategy denoted \sigma∗ 1(\omegaR).)
\end{equation}

% Equation 66
\begin{equation}
Theorem 0 (Fudenberg and Levine, 1992). If \omegas∗ 1 \inΩ, then lim inf \delta\to1 U 1(\delta) \geqV0(s∗
\end{equation}

% Equation 67
\begin{equation}
Definition 2. A strategy s1 is not behaviorally confounded if, for any \omegas′ 1 \inΩsuch that
\end{equation}

% Equation 68
\begin{equation}
1 \neq s1 and any (\alpha0, \alpha2) \inB1(s1), we have p(\alpha0, s1, \alpha2) \neq p(\alpha0, s′ 1, \alpha2).
\end{equation}

% Equation 69
\begin{equation}
indistinguishable from a mixture of two commitment types \omegas′ 1, \omegas′ 2 \inΩ.16 Note also that if there is only one commitment type \omegas1 then s1 is not behaviorally confounded.
\end{equation}

% Equation 70
\begin{equation}
this is fairly innocuous, as \alpha0 is exogenous, so the identification condition p(\alpha0, s1, \alpha2) \neq p(\alpha0, s′ 1, \alpha2) for all \alpha2 \inB0(s1) holds for generic s1 \neq s′
\end{equation}

% Equation 71
\begin{equation}
much more restrictive, because \alpha0 is endogenous, so the identification condition need not
\end{equation}

% Equation 72
\begin{equation}
strategy (A, F) is not behaviorally confounded iff each other type \omegas′ 1 \inΩsatisfies either
\end{equation}

% Equation 73
\begin{equation}
any behavioral confound that induces an \eta-confirmed best response that is not also a best
\end{equation}

% Equation 74
\begin{equation}
1 \inΩ
\end{equation}

% Equation 75
\begin{equation}
distributions \rho \in\Delta(Y0) and ϕ \in\Delta(A1), and any strategy for player 2 \alpha2 \in\Delta(A2), define
\end{equation}

% Equation 76
\begin{equation}
not behaviorally confounded if, for any (\alpha0, \alpha2) \inB0(s1), p(\alpha0, s1, \alpha2) lies outside the convex hull of the set
\end{equation}

% Equation 77
\begin{equation}
1\neqs1:\omegas′ 1\inΩp(\alpha0, s′ 1, \alpha2).
\end{equation}

% Equation 78
\begin{equation}
OT(\rho, ϕ; \alpha2) : max \gamma\in\Delta(Y0\timesA1)
\end{equation}

% Equation 79
\begin{equation}
u1(y0, a1, \alpha2)d\gamma
\end{equation}

% Equation 80
\begin{equation}
\piY0(\gamma) = \rho and \piA1(\gamma) = ϕ.
\end{equation}

% Equation 81
\begin{equation}
(1) For all \varepsilon > 0, there exists \eta > 0 such that for any (\alpha0, \alpha2) \inB\eta(s∗
\end{equation}

% Equation 82
\begin{equation}
1||> \varepsilon but ||p(\alpha0, s′ 1, \alpha2) −p(\alpha0, s∗ 1, \alpha2)||< \eta, there exists ˜s1 satisfying p(\alpha0, ˜s1, \alpha2) = p(\alpha0, s′ 1, \alpha2) and u1(\alpha0, ˜s1, \alpha2) > u1(\alpha0, s′ 1, \alpha2). (2) For any (\alpha0, \alpha2) \inB0(s∗ 1), \gamma(\alpha0, s∗ 1) is the unique solution to OT(\rho(\alpha0), ϕ(\alpha0, s∗ 1); \alpha2).
\end{equation}

% Equation 83
\begin{equation}
over signals against some \eta-confirmed best response—is undetectably dominated—in that
\end{equation}

% Equation 84
\begin{equation}
1 itself undetectably dominates any strategy s′
\end{equation}

% Equation 85
\begin{equation}
Proof. If the second definition fails, there exist (\alpha0, \alpha2) \inB0(s∗
\end{equation}

% Equation 86
\begin{equation}
1||> \varepsilon such that \gamma(\alpha0, s′ 1) solves OT(\rho(\alpha0), ϕ(\alpha0, s∗ 1); \alpha2). Since B0(s∗
\end{equation}

% Equation 87
\begin{equation}
1) for all \eta > 0, this implies that the first definition fails. Conversely, if the first definition fails, there exists \varepsilon > 0 such that for all \eta > 0,
\end{equation}

% Equation 88
\begin{equation}
1 and (\alpha\eta 0, \alpha\eta 2) \inB\eta(s∗
\end{equation}

% Equation 89
\begin{equation}
1||> \varepsilon, ||p(\alpha\eta
\end{equation}

% Equation 90
\begin{equation}
1, \alpha\eta 2) −p(\alpha\eta
\end{equation}

% Equation 91
\begin{equation}
1, \alpha\eta
\end{equation}

% Equation 92
\begin{equation}
1 is not undetectably dominated: u1(\alpha\eta
\end{equation}

% Equation 93
\begin{equation}
1, \alpha\eta 2) \gequ1(\alpha0, ˜s1, \alpha\eta
\end{equation}

% Equation 94
\begin{equation}
that p(\alpha\eta
\end{equation}

% Equation 95
\begin{equation}
1, \alpha\eta 2) = p(\alpha\eta 0, ˜s, \alpha\eta 2). Since a1 is identified, p(\alpha\eta 0, s\eta, \alpha\eta 2) = p(\alpha\eta 0, ˜s, \alpha\eta
\end{equation}

% Equation 96
\begin{equation}
ϕ(\alpha\eta
\end{equation}

% Equation 97
\begin{equation}
1) = ϕ(\alpha\eta
\end{equation}

% Equation 98
\begin{equation}
1 solves OT(\rho(\alpha\eta 0), ϕ(\alpha\eta
\end{equation}

% Equation 99
\begin{equation}
1); \alpha\eta 2). Now, since B\eta(s1) ↓ B0(s1), \Delta(A0)\times(\Delta(A1))Y0 \times\Delta(A2) is compact, and OT(\rho(\alpha0), ϕ(\alpha0, s1); \alpha2) is jointly upper hemi-continuous in (\alpha0, s1, \alpha2), passing to the limit yields s0 1 and (\alpha0, \alpha2) \inB0(s∗
\end{equation}

% Equation 100
\begin{equation}
1||\geq\varepsilon and s0 1 solves OT(\rho(\alpha0), ϕ(\alpha0, s∗ 1); \alpha2). Thus, the second definition fails.
\end{equation}

% Equation 101
\begin{equation}
one-dimensional supermodular games.
\end{equation}

% Equation 102
\begin{equation}
itself. Here and throughout the paper, given a repeated game strategy profile (\sigma0, \sigma1, \sigma2) and a period-t history ht, we abbreviate p(\sigma0(ht), \sigma1(ht), \sigma2(ht)) to p(\sigma0, \sigma1, \sigma2|ht). Lemma 1. Fix a Nash equilibrium (\sigma∗ 0, \sigma∗ 1, \sigma∗
\end{equation}

% Equation 103
\begin{equation}
(1) ||p(\sigma∗ 0, \sigma∗ 1, \sigma∗ 2|ht) −p(\sigma∗
\end{equation}

% Equation 104
\begin{equation}
1, \sigma∗
\end{equation}

% Equation 105
\begin{equation}
(2) ||p(\sigma∗ 0, \sigma∗ 1(\omegaR), \sigma∗ 2|ht) −p(\sigma∗
\end{equation}

% Equation 106
\begin{equation}
1, \sigma∗
\end{equation}

% Equation 107
\begin{equation}
we have ||\sigma∗ 1(ht, \omegaR) −s∗ 1||\leq\varepsilon.
\end{equation}

% Equation 108
\begin{equation}
||\sigma∗ 1(ht, \omegaR) −s∗ 1||> \varepsilon. Condition (1) and the fact (\sigma∗ 0, \sigma∗ 1, \sigma∗
\end{equation}

% Equation 109
\begin{equation}
(\sigma∗ 0(ht), \sigma∗
\end{equation}

% Equation 110
\begin{equation}
1, as \sigma∗ 1(ht) \eta-confirms it against s∗
\end{equation}

% Equation 111
\begin{equation}
condition (2) along with ||\sigma∗ 1(ht, \omegaR) −s∗ 1||> \varepsilon and confound-defeatingness imply that there exists some strategy ˜s1 such that p(\sigma∗ 0, ˜s1, \sigma∗ 2|ht) = p(\sigma∗ 0, \sigma∗ 1(\omegaR), \sigma∗ 2|ht) and u1(\sigma∗ 0, ˜s1, \sigma∗
\end{equation}

% Equation 112
\begin{equation}
u1(\sigma∗ 0, \sigma∗ 1(\omegaR), \sigma∗ 2|ht). But this implies that if the long-run player deviates from \sigma∗ 1(ht, \omegaR) to
\end{equation}

% Equation 113
\begin{equation}
deviation is profitable, contradicting the assumption that (\sigma∗ 0, \sigma∗ 1, \sigma∗
\end{equation}

% Equation 114
\begin{equation}
Theorem 1. If \omegas∗ 1 \inΩand s∗
\end{equation}

% Equation 115
\begin{equation}
lim inf \delta\to1 U 1(\delta) \geqV (s∗
\end{equation}

% Equation 116
\begin{equation}
p(\sigma∗ 0(ht), s∗ 1, \sigma∗ 2(ht)) at public history ht. Since s∗
\end{equation}

% Equation 117
\begin{equation}
also eventually learn that player 1 is not some commitment type other than \omegas∗
\end{equation}

% Equation 118
\begin{equation}
the short-run players come to believe that player 1 is either the commitment type \omegas∗
\end{equation}

% Equation 119
\begin{equation}
Proof. Fix any \varepsilon > 0. We show that there exists ¯\delta < 1 such that for all \delta > ¯\delta, we have U 1(\delta) \geqV (s∗ 1) −\varepsilon. To do so, we fix any Nash equilibrium (\sigma∗ 0, \sigma∗ 1, \sigma∗
\end{equation}

% Equation 120
\begin{equation}
probability measure over infinite histories H\infty, and let Q denote the corresponding measure
\end{equation}

% Equation 121
\begin{equation}
For any \eta > 0, define the set of period-t histories where the equilibrium signal distribution is within \eta of that under the deviation by
\end{equation}

% Equation 122
\begin{equation}
\eta =
\end{equation}

% Equation 123
\begin{equation}
ht : ||p(\sigma∗
\end{equation}

% Equation 124
\begin{equation}
1, \sigma∗ 2|ht) −p(\sigma∗ 0, \sigma∗ 1, \sigma∗ 2|ht)||\leq\eta
\end{equation}

% Equation 125
\begin{equation}
of periods t where ht /\inHt \eta. We include a proof in Appendix B.1.
\end{equation}

% Equation 126
\begin{equation}
t : ht /\inHt
\end{equation}

% Equation 127
\begin{equation}
< ¯T(\eta, µ0) := −2 log µ0(\omegas∗
\end{equation}

% Equation 128
\begin{equation}
M\zeta =
\end{equation}

% Equation 129
\begin{equation}
µ \in\Delta(Ω) : µ
\end{equation}

% Equation 130
\begin{equation}
{\omegaR, \omegas∗
\end{equation}

% Equation 131
\begin{equation}
\geq1 −\zeta
\end{equation}

% Equation 132
\begin{equation}
formly in \delta. The proof, which relies on the martingale convergence theorem and the as-
\end{equation}

% Equation 133
\begin{equation}
follows, given a history h, µt(·|h) \in\Delta(Ω) denotes the posterior belief over Ωconditional on
\end{equation}

% Equation 134
\begin{equation}
Lemma 3. For all \zeta > 0, there exists a set of infinite histories G(\zeta) \subsetH\inftysatisfying Q(G(\zeta)) > 1 −\zeta and a period ˆT(\zeta) (independent of \delta and the choice of equilibrium) such that, for any h \inG(\zeta) and any t \geqˆT(\zeta), we have µt(·|h) \inM\zeta. Now, for any ξ > 0, we say that a short-run player strategy (\alpha0, \alpha2) is a ξ-close best
\end{equation}

% Equation 135
\begin{equation}
1 (denoted (\alpha0, \alpha2) \inˆBξ(s∗ 1)) if (\alpha0, \alpha2) \inB(s1) for some s1 such that ||s1−s∗
\end{equation}

% Equation 136
\begin{equation}
1) = B(s∗
\end{equation}

% Equation 137
\begin{equation}
lim inf
\end{equation}

% Equation 138
\begin{equation}
inf (\alpha0,\alpha2)\inˆBξ(s∗ 1) u1(\alpha0, s∗ 1, \alpha2) = inf (\alpha0,\alpha2)\inB(s∗ 1) u1(\alpha0, s∗ 1, \alpha2) = V (s∗
\end{equation}

% Equation 139
\begin{equation}
Lemma 4. There exist strictly positive functions \zeta(\eta) and ξ(\eta), satisfying lim\eta\to0 \zeta(\eta) = lim\eta\to0 ξ(\eta) = 0, such that if ht \inHt \eta and µt(·|ht) \inM\zeta(\eta) then (\sigma∗ 0(ht), \sigma∗ 2(ht)) \inˆBξ(\eta)(s∗
\end{equation}

% Equation 140
\begin{equation}
(at least) probability 1 −\zeta(\eta) event that h \inG(\zeta(\eta), the expected number of periods where either ht /\inHt \eta or µt(·|ht) /\inM\zeta(\eta) is at most ¯T(\eta, µ0) + ˆT(\zeta(\eta)). By Lemma 4, in any period where ht \inHt \eta and µt(·|ht) \inM\zeta(\eta), we have (\sigma∗ 0(ht), \sigma∗ 2(ht)) \inˆBξ(\eta)(s∗
\end{equation}

% Equation 141
\begin{equation}
u1(\sigma∗ 0(ht), s∗ 1, \sigma∗ 2(ht)) \geqlim inf
\end{equation}

% Equation 142
\begin{equation}
inf (\alpha0,\alpha2)\inˆBξ(\eta)(s∗ 1) u1(\alpha0, s∗ 1, \alpha2) −\varepsilon 3 = V (s∗
\end{equation}

% Equation 143
\begin{equation}
Front-loading the expected periods where ht /\inHt \eta or µt(·|ht) /\inM\zeta(\eta) (and positing the
\end{equation}

% Equation 144
\begin{equation}
1 −\delta
\end{equation}

% Equation 145
\begin{equation}
u1 + \delta
\end{equation}

% Equation 146
\begin{equation}
As \delta \to1, this lower bound converges to
\end{equation}

% Equation 147
\begin{equation}
By continuity, at a cost of \varepsilon/3, this bound stays valid for all large enough \delta < 1. Finally, taking \eta to be small enough so \zeta(\eta)|u1−¯V (s∗
\end{equation}

% Equation 148
\begin{equation}
There, we show that if \omegas∗ 1 \inΩand s∗
\end{equation}

% Equation 149
\begin{equation}
lim inf \delta\to1 U 1(\delta) \geq\beta(s∗
\end{equation}

% Equation 150
\begin{equation}
1) + (1 −\beta(s∗
\end{equation}

% Equation 151
\begin{equation}
for some function \beta(s∗ 1; µ0) (constructed explicitly in Appendix A), where \beta(s∗ 1; µ0) = 1 if s∗
\end{equation}

% Equation 152
\begin{equation}
is not behaviorally confounded, and \beta(s∗
\end{equation}

% Equation 153
\begin{equation}
joint distributions over Y0 \times A1 they induce, and uses this characterization to develop a
\end{equation}

% Equation 154
\begin{equation}
Definition 4. Fix finite sets X, Y and a function u : X \times Y \toR. A set S \subsetX \times Y is
\end{equation}

% Equation 155
\begin{equation}
(xi, yi)N i=1
\end{equation}

% Equation 156
\begin{equation}
i=1
\end{equation}

% Equation 157
\begin{equation}
(with convention yN+1 = y1),
\end{equation}

% Equation 158
\begin{equation}
i=1 u(xi, yi) >
\end{equation}

% Equation 159
\begin{equation}
i=1
\end{equation}

% Equation 160
\begin{equation}
Proposition 5. A joint distribution \gamma \in\Delta(X \times Y ) satisfying \piX(\gamma) = \rho and \piY (\gamma) = ϕ is
\end{equation}

% Equation 161
\begin{equation}
max \gamma′\in\Delta(X\timesY )
\end{equation}

% Equation 162
\begin{equation}
u(x, y)d\gamma′
\end{equation}

% Equation 163
\begin{equation}
\piX(\gamma′) = \rho and \piY (\gamma′) = ϕ
\end{equation}

% Equation 164
\begin{equation}
if and only if its support supp(\gamma) \subsetX \times Y is strictly u-cyclically monotone.
\end{equation}

% Equation 165
\begin{equation}
\gamma(\alpha0, s∗
\end{equation}

% Equation 166
\begin{equation}
supp(s∗ 1) := {(y0, a1) \inY0 \times A1 : a1 \insupp(s∗ 1(y0))} . In addition, letting u1(·, \alpha2) denote player 1’s utility u1(y0, a1, \alpha2) as a function of (y0, a1) for a fixed player 2 strategy \alpha2, we say that u1 is strictly cyclically separable if whenever a set S \subsetY0 \times A1 is strictly u1(·, \alpha2)-cyclically monotone for some \alpha2, it is strictly u1(·, \alpha2)- cyclically monotone for all \alpha2. In this case, the strict u1-cyclical monotonicty of a set S \subset Y0 \times A1 is well-defined independent of \alpha2.20 Finally, we say that a strategy s∗
\end{equation}

% Equation 167
\begin{equation}
u1(·, \alpha2) (resp., u1)-cyclically monotone if supp(s∗ 1) \subsetY0 \times A1 is strictly u1(·, \alpha2) (resp.,
\end{equation}

% Equation 168
\begin{equation}
1 is confound-defeating if and only if it is strictly u1(·, \alpha2)-cyclically monotone for all (\alpha0, \alpha2) \inB0(s∗
\end{equation}

% Equation 169
\begin{equation}
:= sup
\end{equation}

% Equation 170
\begin{equation}
\omegas1 \inΩand s1 is strictly u1-cyclically
\end{equation}

% Equation 171
\begin{equation}
min (\alpha0,\alpha2)\inB(s1) u1(\alpha0, s1, \alpha2). Corollary 2. If u1 is strictly cyclically separable, \omegas1 \inΩ, and s1 is strictly u1-cyclically
\end{equation}

% Equation 172
\begin{equation}
lim inf \delta\to1 U 1(\delta) \geqV (s∗
\end{equation}

% Equation 173
\begin{equation}
lim inf \delta\to1 U 1(\delta) \geqvCM
\end{equation}

% Equation 174
\begin{equation}
:= sup
\end{equation}

% Equation 175
\begin{equation}
max (\alpha0,\alpha2)\inB(s1) u1(\alpha0, s1, \alpha2),
\end{equation}

% Equation 176
\begin{equation}
In the following statement, u1 is cyclically separable if whenever a set S \subsetY0 \times A1 is u1(·, \alpha2)-cyclically monotone for some \alpha2, it is u1(·, \alpha2)-cyclically monotone for all \alpha2.
\end{equation}

% Equation 177
\begin{equation}
such that, for any prior µ0 satisfying µ0(\omegaR) > 1 −κ and any \delta < 1, ¯U1(\delta) < ¯vCM
\end{equation}

% Equation 178
\begin{equation}
multiple long-run players and incomplete information. Lemma 5. For any Nash equilibrium (\sigma∗ 0, \sigma∗ 1, \sigma∗ 2) and any history ht, \sigma∗ 1(ht, \omegaR) is u1-
\end{equation}

% Equation 179
\begin{equation}
Proof. Note that \sigma∗ 1(ht, \omegaR) must solve OT(\sigma∗ 0(ht), ϕ(\sigma∗ 0(ht), \sigma∗ 1(ht, \omegaR)), \sigma∗ 2(ht)).
\end{equation}

% Equation 180
\begin{equation}
payoff at history ht than \sigma∗ 1(ht, \omegaR) does, but gives the same signal distribution, and hence
\end{equation}

% Equation 181
\begin{equation}
Santambrogio, 2015), this implies that supp(\gamma(\sigma∗ 0(ht), \sigma∗ 1(ht, \omegaR))) is u1(·, \sigma∗ 2(ht))-cyclically monotone. Hence, by cyclical separability, \sigma∗ 1(ht, \omegaR) is u1-cyclically monotone.
\end{equation}

% Equation 182
\begin{equation}
The relevant supermodularity notion is strict supermodularity in (y0, a1) for all \alpha2.
\end{equation}

% Equation 183
\begin{equation}
u1(y0, a1, a2) −u1(y0, a′
\end{equation}

% Equation 184
\begin{equation}
Definition 6. A long-run player strategy s1 is monotone if, for any y0 \succy′ 0, a1 \insupp(s1(y0)),
\end{equation}

% Equation 185
\begin{equation}
1 \insupp(s1(y′
\end{equation}

% Equation 186
\begin{equation}
(1) s∗
\end{equation}

% Equation 187
\begin{equation}
(2) s∗
\end{equation}

% Equation 188
\begin{equation}
(3) s∗
\end{equation}

% Equation 189
\begin{equation}
Lemma 6. Suppose u1 is strictly supermodular. Then, for any (\alpha0, \alpha2), s∗
\end{equation}

% Equation 190
\begin{equation}
\gamma(\alpha0, s∗ 1) is the unique solution to OT(\rho(\alpha0), ϕ(\alpha0, s∗ 1); \alpha2). Proof. By Lemma 2.8 in Santambrogio (2015), if s∗ 1 is monotone then \gamma(\alpha0, s∗
\end{equation}

% Equation 191
\begin{equation}
co-monotone transport plan between \rho(\alpha0) and ϕ(\alpha0, s∗
\end{equation}

% Equation 192
\begin{equation}
tion \gamma \in\Delta(Y0 \times A1) with marginals \rho(\alpha0) and ϕ(\alpha0, s∗ 1) such that, according to \gamma, y0 and a1 are co-monotone random variables. Conversely, since \rho(\alpha0) has full support, if s∗
\end{equation}

% Equation 193
\begin{equation}
monotone then \gamma(\alpha0, s∗
\end{equation}

% Equation 194
\begin{equation}
the unique solution to OT(\rho(\alpha0), ϕ(\alpha0, s∗ 1); \alpha2).
\end{equation}

% Equation 195
\begin{equation}
1 is confound-defeating then \gamma(\alpha0, s∗
\end{equation}

% Equation 196
\begin{equation}
OT(\rho(\alpha0), ϕ(\alpha0, s∗ 1); \alpha2) for any (\alpha0, \alpha2) \inB0(s∗
\end{equation}

% Equation 197
\begin{equation}
1 is monotone then it is the unique solution to OT(\rho(\alpha0), ϕ(\alpha0, s∗ 1); \alpha2) for any (\alpha0, \alpha2),
\end{equation}

% Equation 198
\begin{equation}
1 is monotone if and only if \gamma(\alpha0, s∗
\end{equation}

% Equation 199
\begin{equation}
monotone, as shown in the proof of Lemma 6, and \gamma(\alpha0, s∗
\end{equation}

% Equation 200
\begin{equation}
= sup
\end{equation}

% Equation 201
\begin{equation}
\omegas1 \inΩand s1 is monotone
\end{equation}

% Equation 202
\begin{equation}
min (\alpha0,\alpha2)\inB(s1) u1(\alpha0, s1, \alpha2),
\end{equation}

% Equation 203
\begin{equation}
= sup
\end{equation}

% Equation 204
\begin{equation}
max (\alpha0,\alpha2)\inB(s1) u1(\alpha0, s1, \alpha2).
\end{equation}

% Equation 205
\begin{equation}
lim inf \delta\to1 U 1(\delta) \geqvmon
\end{equation}

% Equation 206
\begin{equation}
Conversely, for all \varepsilon > 0, there exists κ > 0 such that for any prior µ0 satisfying µ0(\omegaR) > 1 −κ and any \delta < 1, ¯U1(\delta) < ¯vmon
\end{equation}

% Equation 207
\begin{equation}
that µ0(\omegas1) > 0. This implies the supermodular cases of Proposition 1 (as in the deterrence
\end{equation}

% Equation 208
\begin{equation}
and Proposition 2 (as in the trust game u1 is strictly supermodular with the order H \succL and T \succN when min{w, z} > 0), as well as Proposition 3 (as in a signaling game u1 = (1 −\lambda)v(a2(a1)) −\lambdaw(a1, \theta) is strictly supermodular in (a1, \theta) for any function a2 : a1 \toR
\end{equation}

% Equation 209
\begin{equation}
payoff from any monotone, non-behaviorally confounded strategy s1 such that µ0(\omegas1) > 0,
\end{equation}

% Equation 210
\begin{equation}
case”) then u1 is strictly supermodular with the order F \succA and C \succD. Since any
\end{equation}

% Equation 211
\begin{equation}
= ¯vmon
\end{equation}

% Equation 212
\begin{equation}
µ0(\omegaR) and \delta both approach 1. For example, this holds in the deterrence and trust games.
\end{equation}

% Equation 213
\begin{equation}
that the model covers communication games by dropping player 0; viewing \rho(y0) as the prior distribution of a payoff-relevant state y0; letting \rho(y1|a1) = 1({y1 = a1}) (so a1 is perfectly
\end{equation}

% Equation 214
\begin{equation}
that u1 and u2 depend on a2 only through the induced response a2(a1) \inR. In this section, we refer to player 1 as the sender and player 2 as the receiver, and we relabel y0 as \theta.
\end{equation}

% Equation 215
\begin{equation}
that u1(\theta, a1, r) = (1−\lambda)v(a1, r)−\lambdaw(a1, \theta) for some functions v and w and some \lambda \in(0, 1), 21Moreover, this result extends to preferences of the form u1(a1, r, \theta) = (1 −\lambda)v(a1, r) −\lambdaw(a1, \theta).
\end{equation}

% Equation 216
\begin{equation}
and if w is strictly submodular, then lim sup\delta\to1 ¯U1(\delta) \geqvmon. Thus, a patient sender with
\end{equation}

% Equation 217
\begin{equation}
utility v(r) for the sender and a utility u2(\theta, r) for the receiver, so the sender’s action a1 is payoff irrelevant. In this game, the sender’s utility u1(\theta, a1, r) is independent of a1 and hence cannot be strictly supermodular in (˜r, \theta). However, suppose that the sender has a “grain of
\end{equation}

% Equation 218
\begin{equation}
game by committing to pay a small communication cost w(a1, \theta) whenever she takes action a1 in state \theta. We ask what commitment payoffs V (s1) can be secured by leveraging such a
\end{equation}

% Equation 219
\begin{equation}
strategy ˆs1 : \Theta \to\Delta(A1), there exists a direct communication mechanism s1 : \Theta \to\Delta(R)
\end{equation}

% Equation 220
\begin{equation}
a communication mechanism s1 : \Theta \to\Delta(R) to be approximately securable. Proposition 8. If a communication mechanism s1 : \Theta \to\Delta(R) is monotone with respect to some order (\succeq\Theta, \succeqR) and is such that \omegas1 \inΩand s1 is not behaviorally confounded, then
\end{equation}

% Equation 221
\begin{equation}
lim inf \delta\to1 U w(\delta) \geq(1 −\varepsilon)V (s1) −\varepsilon, where U w(\delta) is the infimum of the long-run player’s payoff in any Nash equilibrium in the
\end{equation}

% Equation 222
\begin{equation}
u1(\theta, ˜r, r) = (1 −\varepsilon)v(r) −\varepsilonw(˜r, \theta).
\end{equation}

% Equation 223
\begin{equation}
An interpretation of the communication cost w(˜r, \theta) is that this represents a “lying cost”
\end{equation}

% Equation 224
\begin{equation}
˜r in state \theta. In particular, if R = \Theta and the receiver’s optimal action in state \theta is r = \theta, we can interpret the sender’s message ˜r \in\Theta as a report of the state, and we can interpret w(˜r, \theta) as the lying cost associated with misreporting state \theta as ˜r. This example matches the
\end{equation}

% Equation 225
\begin{equation}
main example in Kartik (2009), where it is likewise assumed that the lying cost w(˜r, \theta) is
\end{equation}

% Equation 226
\begin{equation}
To operationalize Proposition 8, it remains to characterize what mechanisms s1 : \Theta \to
\end{equation}

% Equation 227
\begin{equation}
a state \theta and an action r are linked if r \insupp(s1(\theta)). We will see that if s1 is monotone
\end{equation}

% Equation 228
\begin{equation}
Definition 7. A forbidden triple for a mechanism s1 : \Theta \to\Delta(R) is either (1) A set of three distinct actions {r1, r2, r3} and four distinct state {\theta1, \theta2, \theta3, \theta4} where rk \insupp(s1(\thetak)) for k \in{1, 2, 3} and {r1, r2, r3} \subsetsupp(s1(\theta4)); or (2) A set of three distinct states {\theta1, \theta2, \theta3} and four distinct actions {r1, r2, r3, r4} where {rk, r4} \insupp(s1(\thetak)) for k \in{1, 2, 3}. \theta1 \theta2 \theta3 \theta4
\end{equation}

% Equation 229
\begin{equation}
Notes. Monotonicity is violated for any placement of \theta4 in the order \succeq\Theta.
\end{equation}

% Equation 230
\begin{equation}
\theta1 \prec\theta2 \prec\theta3, then s1 cannot be monotone with respect to any order, as if \theta4 \prec\theta2 then s1
\end{equation}

% Equation 231
\begin{equation}
is non-monotone because r3 \insupp(s1(\theta4)) but r2 \insupp(s1(\theta2)); and if \theta4 \succ\theta2 then s1 is non-monotone because r1 \insupp(s1(\theta4)) but r2 \insupp(s1(\theta2)). See Figure 1.
\end{equation}

% Equation 232
\begin{equation}
Proposition 9. A communication mechanism s1 : \Theta \to\Delta(R) is monotone with respect to some order (\succeq\Theta, \succeqR) if and only if G(s1) is acyclic and does not contain a forbidden triple.
\end{equation}

% Equation 233
\begin{equation}
some order includes, for example, all partitions (i.e., deterministic mechanisms s1 : \Theta \toR)
\end{equation}

% Equation 234
\begin{equation}
optimal mechanism. For example, if the receiver’s optimal action is r = E[\theta] and \theta \in{0, 1} with equal probability, and the sender’s utility is 1({r \in{1/3, 2/3}}), then the unique optimal mechanism induces r \in{1/3, 2/3} with equal probability, but this mechanism is
\end{equation}

% Equation 235
\begin{equation}
a joint distribution over X \times Y that is co-monotone with respect to some order (\succeqX, \succeqY ).
\end{equation}

% Equation 236
\begin{equation}
noted in the text, this extension is particularly important for games where \alpha0 is endogenous,
\end{equation}

% Equation 237
\begin{equation}
Then we calculate the minimum probability \beta that the long-run weight on s∗
\end{equation}

% Equation 238
\begin{equation}
level under the deviation measure Q. We call \beta the “salience” of s∗
\end{equation}

% Equation 239
\begin{equation}
bound for a patient long-run player’s payoff as a function of \beta. If s∗
\end{equation}

% Equation 240
\begin{equation}
belief µ \in\Delta(Ω) and a subset Ω′ \subsetΩ, we denote the conditional distribution of µ over Ω′ by µ(·|Ω′). In addition, given a belief µ \in\Delta(Ω\ {\omegaR}), we slightly abuse notation by also
\end{equation}

% Equation 241
\begin{equation}
\omegas1\inΩ\{\omegaR} µ(\omegas1)s1(y0)[a1]. Definition 8. For any \eta, ς > 0, a number c \in[0, 1] is an (\eta, ς)-confounding weight if there exists a belief µ \in\Delta(Ω) satisfying the following conditions. (1) µ(\omegaR) < 1. (2) There exists s1 \in\Delta(A1)Y0 such that ||µ(·|Ω\ {\omegaR}) −s1||< ς and B(s1) \ B(s∗
\end{equation}

% Equation 242
\begin{equation}
(3) µ(Ω\eta(s∗
\end{equation}

% Equation 243
\begin{equation}
1) = {\omegas1 \inΩ\{\omegaR} : ||p(\alpha0, s1, \alpha2)−p(\alpha0, s∗ 1, \alpha2)||< \eta for some (\alpha0, \alpha2) \inB1(s∗ 1)}\cup{\omegaR}. (4) µ(s∗ 1|Ω\ {\omegaR}) = c. Let c\eta,ς = sup{c : c is (\eta, ς)-confounding}, with the convention that if no (\eta, ς)-confounding weight exists, then c\eta,ς = −\infty. Finally, let c0 = limς\to0 lim\eta\to0 cς,\eta. Condition (1) implies that µ(·|Ω\ {\omegaR}) is well-defined. Condition (2) says that µ(·|Ω\ {\omegaR}) is within ς of a belief to which the short-run players have a best response outside
\end{equation}

% Equation 244
\begin{equation}
1). Condition (3) says that µ puts at most \eta weight on commitment types that induce signals that are not \eta close to those induced by s∗ 1. Condition (4) says that µ(·|Ω\ {\omegaR})
\end{equation}

% Equation 245
\begin{equation}
1) :=
\end{equation}

% Equation 246
\begin{equation}
\eta>0
\end{equation}

% Equation 247
\begin{equation}
is well-defined and contains {\omegas∗ 1, \omegaR}.
\end{equation}

% Equation 248
\begin{equation}
1 \inΩis \beta = max
\end{equation}

% Equation 249
\begin{equation}
1) \ {\omegaR}) −c0
\end{equation}

% Equation 250
\begin{equation}
with the convention that if c0 = −\inftythen \beta = 1.
\end{equation}

% Equation 251
\begin{equation}
Bayes’ rule) \beta is the minimum probability the long-run weight on s∗
\end{equation}

% Equation 252
\begin{equation}
Theorem 2. If \omegas∗ 1 \inΩand s∗ 1 is confound-defeating and has salience \beta, then lim inf \delta\to1 U 1(\delta) \geq\betaV (s∗ 1) + (1 −\beta)V0(s∗
\end{equation}

% Equation 253
\begin{equation}
1) = {\omegas∗ 1, \omegaR} for sufficiently small \eta. By upper hemi-continuity of B(·), this implies that c\eta,ς = −\inftyfor sufficiently small \eta and ς, so \beta = 1. Theorem 2 therefore generalizes Theorem 1. Moreover, since c0 < 1, we have \beta \to1 whenever µ0|Ω0(s∗ 1)\{\omegaR}(s∗ 1) \to1, in which case
\end{equation}

% Equation 254
\begin{equation}
1)\{\omegaR}(s∗ 1) =
\end{equation}

% Equation 255
\begin{equation}
1)\{\omegaR}(s∗
\end{equation}

% Equation 256
\begin{equation}
\beta = max
\end{equation}

% Equation 257
\begin{equation}
1)\{\omegaR}((A,F))−c0
\end{equation}

% Equation 258
\begin{equation}
, and Theorem 2 implies that as \delta \to1 the long-run player is assured a payoff of at least \betap + (1 −\beta)(1 −p). In particular, whenever C is the unique
\end{equation}

% Equation 259
\begin{equation}
increases, \beta converges to 1 and the long-run player is assured her pure Stackelberg payoff p.
\end{equation}

% Equation 260
\begin{equation}
The proof of Theorem 2 follows from the proof of Theorem 1 and the fact that (\alpha0, \alpha2) \in
\end{equation}

% Equation 261
\begin{equation}
1) (and hence u1(\sigma∗
\end{equation}

% Equation 262
\begin{equation}
1, \sigma∗ 2) \geqinf(\alpha0,\alpha2)\inB\eta(s∗ 1) u1(\alpha0, s∗ 1, \alpha2)) for all ht \inHt
\end{equation}

% Equation 263
\begin{equation}
Lemma 7. For any \eta sufficiently small, any t > ¯T(\eta) (chosen as in Theorem 1), and any Nash equilibrium (\sigma∗ 0, \sigma∗ 1, \sigma∗
\end{equation}

% Equation 264
\begin{equation}
lim inf
\end{equation}

% Equation 265
\begin{equation}
Q(h \inH\infty: (\sigma∗ 0(ht), \sigma∗ 2(ht)) \inˆBξ(s∗ 1)) \geq\beta.
\end{equation}

% Equation 266
\begin{equation}
from a point to a set, and conv(·) denotes convex hull), then B(˜µ) \subsetB(s∗
\end{equation}

% Equation 267
\begin{equation}
Lemma 8. For any s1 \in\Delta(A1)Y0 and ς > 0, let Cς(s1) = {s′
\end{equation}

% Equation 268
\begin{equation}
1) \subsetB(s1) for all s′′
\end{equation}

% Equation 269
\begin{equation}
1∥\leqς}. Then, there exists ϑ(ς, s1) > 0, vanishing as ς \to0, such that for all ˜s1, s′
\end{equation}

% Equation 270
\begin{equation}
1 \in Cς(s1) and d(˜s1, conv({s1, s′ 1})) \leqϑ(ς, s1), we have B(˜s1) \subsetB(s1). Proof. We first show that B(˜s1) \subsetB(s1) if ϑ(ς, s1) = 0, so that ˜s1 \inconv({s1, s′
\end{equation}

% Equation 271
\begin{equation}
1) \subsetB(s1), the set of player 0 best responses at any ˜s1 \in
\end{equation}

% Equation 272
\begin{equation}
1) \subsetB(s1).
\end{equation}

% Equation 273
\begin{equation}
1 \inCς(s1), such that if d(˜s1, conv({s1, s′ 1})) \leqϑ(ς, s1) then B(˜s1) \subsetB(s1). For any ς, note
\end{equation}

% Equation 274
\begin{equation}
1 \inCς(s1) (the closure of Cς(s1)), then B(s′′ 1) \subsetB(s1). From here, suppose no such
\end{equation}

% Equation 275
\begin{equation}
1 \inCς(s1). Then there exists a sequence (sn
\end{equation}

% Equation 276
\begin{equation}
1 \inCς(s1), d(˜sn
\end{equation}

% Equation 277
\begin{equation}
1) \ B(s1) \neq \emptysetfor large enough n. Taking
\end{equation}

% Equation 278
\begin{equation}
1 \inCς(s1) (since this
\end{equation}

% Equation 279
\begin{equation}
established above, that B(˜s1) \subsetB(s1) when ˜s1 \inconv({s1, s′
\end{equation}

% Equation 280
\begin{equation}
Proof of Lemma 7. We show there exists strictly positive functions \zeta(\eta) and ξ(\eta), vanishing
\end{equation}

% Equation 281
\begin{equation}
Q(h \inH\infty: (\sigma∗ 0(ht), \sigma∗ 2(ht)) \inˆBξ(\eta)(s∗ 1))) \geq(1 −\zeta(\eta))\beta\zeta,\eta, where \betaς,\eta = (1 −\eta)µ0|Ω0(s∗ 1)\{\omegaR}(s∗
\end{equation}

% Equation 282
\begin{equation}
1) in place of {\omegas∗
\end{equation}

% Equation 283
\begin{equation}
that, on a set of histories G(\zeta(\eta)) satisfying Q(G(\zeta(\eta))) > 1 −\zeta(\eta), both ht \inHt
\end{equation}

% Equation 284
\begin{equation}
1) \ {\omegaR}|ht) > 1 −\eta for all t > ¯T(\eta), independent of the choice of the equilibrium
\end{equation}

% Equation 285
\begin{equation}
consider three possible cases, and show for sufficiently small \eta, that in the first two cases (\sigma∗ 0(ht), \sigma∗ 2(ht)) \inˆBξ(\eta)(s∗ 1) and the third arises with probability at most 1 −\betaς,\eta. This then implies, in total, Q(h \inH\infty: (\sigma∗ 0(ht), \sigma∗ 2(ht)) \inˆBξ(\eta)(s∗ 1))) is no less than (1 −\zeta(\eta))\beta\zeta,\eta,
\end{equation}

% Equation 286
\begin{equation}
First, suppose that µt({\omegaR, \omegas∗ 1}|ht) > 1 −\zeta(\eta). Then, for \zeta(\eta) and ξ(\eta) chosen as in Lemma 4, we have that (\sigma∗ 0(ht), \sigma∗ 2(ht)) \inˆBξ(\eta)(s∗
\end{equation}

% Equation 287
\begin{equation}
Second, suppose that µt({\omegaR, \omegas∗ 1}|ht) \leq1−\zeta(\eta) but µt(·|ht, Ω\{\omegaR}) \inCς(s∗
\end{equation}

% Equation 288
\begin{equation}
ς > 0 fixed independent of \eta. Since ht \inHt
\end{equation}

% Equation 289
\begin{equation}
1) \ {\omegaR}) > 1 −\eta, an argument identical to the proof of Lemma 4 implies that the minimum of µt(\omegaR|ht) and ||\sigma∗ 1(ht) −s∗
\end{equation}

% Equation 290
\begin{equation}
d(\sigma∗ 1(ht), conv({µt(·|ht, Ω\eta(s∗ 1) \ {\omegaR}), s∗ 1})) \leqϑ(ς, s∗
\end{equation}

% Equation 291
\begin{equation}
Since µt(·|ht, Ω\{\omegaR}) \inCς(s∗ 1), Lemma 8 then implies (\sigma∗ 0(ht), \sigma∗ 2(ht)) \inB(s∗
\end{equation}

% Equation 292
\begin{equation}
Third, suppose that µt({\omegaR, \omegas∗ 1}|ht) \leq1 −\zeta(\eta) and µt(·|ht, Ω\ {\omegaR}) \inCς(s∗
\end{equation}

% Equation 293
\begin{equation}
condition implies that it also satisfies Condition (2). Moreover, since µt(Ω\eta(s∗ 1)|ht) \geq1 −\eta (as t > ¯T(\eta)), it also satisfies Condition (3). Thus, by the definition of (\eta, ς)-confounding weights, µt(\omegas∗ 1|ht, Ω\ {\omegaR}) \leqcς,\eta. Now, since \omegas∗ 1 \inΩ\eta(s∗ 1) \ {\omegaR} \subsetΩ\ {\omegaR}, we have µt(\omegas∗ 1|ht, Ω\ {\omegaR}) = µt(\omegas∗
\end{equation}

% Equation 294
\begin{equation}
1) \ {\omegaR})µt(Ω\eta(s∗ 1) \ {\omegaR}|ht). Since µt(\omegas∗ 1|ht, Ω\ {\omegaR}) \leqcς,\eta and µt(Ω\eta(s∗ 1) \ {\omegaR}|ht) \geq1 −\eta, we have µt(\omegas∗
\end{equation}

% Equation 295
\begin{equation}
1) \ {\omegaR}) \leq
\end{equation}

% Equation 296
\begin{equation}
q\eta,ς := Q
\end{equation}

% Equation 297
\begin{equation}
h \inH\infty: µt(\omegas∗
\end{equation}

% Equation 298
\begin{equation}
1) \ {\omegaR}) \leq
\end{equation}

% Equation 299
\begin{equation}
1) \ {\omegaR}) is a Q-submartingale, we have
\end{equation}

% Equation 300
\begin{equation}
+ (1 −q\eta,ς)(1) \geqµ0(s∗
\end{equation}

% Equation 301
\begin{equation}
1) \ {\omegaR})
\end{equation}

% Equation 302
\begin{equation}
q\eta,ς \leqmin
\end{equation}

% Equation 303
\begin{equation}
1) \ {\omegaR})
\end{equation}

% Equation 304
\begin{equation}
= 1 −\beta\eta,ς,
\end{equation}

% Equation 305
\begin{equation}
d(p(\sigma∗
\end{equation}

% Equation 306
\begin{equation}
1, \sigma∗ 2|ht)||p(\sigma∗ 0, \sigma∗ 1, \sigma∗
\end{equation}

% Equation 307
\begin{equation}
\leq−log µ0(\omegas∗
\end{equation}

% Equation 308
\begin{equation}
t : d(p(\sigma∗
\end{equation}

% Equation 309
\begin{equation}
1, \sigma∗ 2|ht)||p(\sigma∗ 0, \sigma∗ 1, \sigma∗
\end{equation}

% Equation 310
\begin{equation}
< −2 log µ0(\omegas∗
\end{equation}

% Equation 311
\begin{equation}
p(\sigma∗
\end{equation}

% Equation 312
\begin{equation}
1, \sigma∗
\end{equation}

% Equation 313
\begin{equation}
p(\sigma∗ 0, \sigma∗ 1, \sigma∗
\end{equation}

% Equation 314
\begin{equation}
\leq\eta2
\end{equation}

% Equation 315
\begin{equation}
=\Rightarrowht \inHt
\end{equation}

% Equation 316
\begin{equation}
We first show that the desired conclusion holds for each \delta and each equilibrium, and then show that ˆT can be fixed independent of the choice of \delta and the equilibrium. Lemma 9. For any \delta < 1, any strategy profile (\sigma∗ 0, \sigma∗ 1, \sigma∗ 2) where (\sigma∗ 0, \sigma∗ 2) \inB1(s∗
\end{equation}

% Equation 317
\begin{equation}
hence, any Nash equilibrium), and any \zeta > 0, there exists a set of infinite histories G(\zeta) \subset H\inftysatisfying Q(G(\zeta)) > 1 −\zeta and a period ˆT such that, for any h \inG(\zeta) and any t \geqˆT, we have µt(·|h) \inM\zeta.
\end{equation}

% Equation 318
\begin{equation}
P, µt(·|h) converges Q-almost surely to some limit distribution µ\infty(·|h) (e.g., Mailath and Samuelson (2006), Lemma 15.4.2). We show that, for Q-almost all histories h \inH\infty, µ\infty({\omegaR, \omegas∗
\end{equation}

% Equation 319
\begin{equation}
µ\infty(Ω\ {\omegaR}|h) > 0, let \omegas1 \inΩ\ {\omegaR} satisfy µ\infty(\omegas1|h) > 0, and let c > 0 and T satisfy µt(\omegas1|h) > c for all t > T. Suppose also that the set of signals y1 that realize infinitely often
\end{equation}

% Equation 320
\begin{equation}
1 = supp(\rho(·|s∗
\end{equation}

% Equation 321
\begin{equation}
histories occur Q-almost surely. Next, for any Ω′ \subseteqΩ\ {\omegaR}, let pY1(\sigma∗
\end{equation}

% Equation 322
\begin{equation}
equilibrium distribution of y1 conditional on reaching history ht and the event \omega \inΩ′; when Ω′ is a singleton, Ω′ = {ˆs1}, we write this as pY1(\sigma∗ 0, ˆs1|ht). Then, for any y1 \inY ∗
\end{equation}

% Equation 323
\begin{equation}
µt+1(\omegas1|Ω\ {\omegaR}, ht, y1) −µt(\omegas1|Ω\ {\omegaR}, ht)
\end{equation}

% Equation 324
\begin{equation}
=
\end{equation}

% Equation 325
\begin{equation}
pY1(\sigma∗ 0, s1|ht)[y1]µt(\omegas1|Ω\ {\omegaR}, ht) pY1(\sigma∗ 0, ˜s1|ht, Ω\ {\omegaR})[y1] −µt(\omegas1|Ω\ {\omegaR}, ht)
\end{equation}

% Equation 326
\begin{equation}
= µt(\omegas1|Ω\ {\omegaR}, ht) pY1(\sigma∗ 0, ˜s1|ht, Ω\ {\omegaR})[y1] pY1(\sigma∗ 0, s1|ht)[y1] −pY1(\sigma∗ 0, ˜s1|ht, Ω\ {\omegaR})[y1]
\end{equation}

% Equation 327
\begin{equation}
pY1(\sigma∗ 0, s1|ht)[y1] −pY1(\sigma∗ 0, ˜s1|ht, Ω\ {\omegaR})[y1]
\end{equation}

% Equation 328
\begin{equation}
lim t\to\infty pY1(\sigma∗ 0, s1|ht)[y1] −pY1(\sigma∗ 0, ˜s1|ht, Ω\ {\omegaR})[y1] = 0. At the same time, applying the argument in Lemma 2 conditional on the event \omega \neq \omegaR implies that, for Q-almost all histories h \inH\infty, lim t\to\infty
\end{equation}

% Equation 329
\begin{equation}
pY1(\sigma∗ 0, ˜s1|ht, Ω\ {\omegaR}) −pY1(\sigma∗
\end{equation}

% Equation 330
\begin{equation}
= 0.
\end{equation}

% Equation 331
\begin{equation}
In particular, since pY1(\sigma∗
\end{equation}

% Equation 332
\begin{equation}
1|ht)[y1] = 0 for all y1 /\inY ∗ 1 , this implies that, for all y1 /\inY ∗
\end{equation}

% Equation 333
\begin{equation}
lim t\to\inftypY1(\sigma∗ 0, ˜s1|ht, Ω\ {\omegaR})[y1] = 0. Since we have already shown that pY1(\sigma∗ 0, ˜s1|ht, Ω\ {\omegaR})[y1] and pY1(\sigma∗
\end{equation}

% Equation 334
\begin{equation}
same limit for all y1 \inY ∗
\end{equation}

% Equation 335
\begin{equation}
lim t\to\infty pY1(\sigma∗ 0, s1|ht) −pY1(\sigma∗ 0, ˜s1|ht, Ω\ {\omegaR}) = 0.
\end{equation}

% Equation 336
\begin{equation}
lim t\to\infty
\end{equation}

% Equation 337
\begin{equation}
pY1(\sigma∗ 0, s1|ht) −pY1(\sigma∗
\end{equation}

% Equation 338
\begin{equation}
= 0. Finally, since \omegas1 \inΩand s∗
\end{equation}

% Equation 339
\begin{equation}
hence µ\infty({\omegaR, \omegas∗ 1}|h) = 1.
\end{equation}

% Equation 340
\begin{equation}
fn : H\infty\toR converges Q-almost surely to f, then for all \zeta > 0, there exists G(\zeta) \subsetH\infty satisfying µ(G(\zeta)) \geq1 −\zeta such that fn \tof uniformly on G(\zeta). Thus, Lemma 9 follows from Egorov’s theorem applied to the sequence of conditional beliefs µt({\omegas∗ 1, \omegaR}|h) and the
\end{equation}

% Equation 341
\begin{equation}
We now show that ˆT can be chosen as a function only of \zeta and not of \delta or the equilibrium strategies. To this end, let Q\sigma0,\sigma2 be the probability measure on H\inftyinduced by strategies (\sigma0, s∗ 1, \sigma2), and let µ\sigma0,\sigma2
\end{equation}

% Equation 342
\begin{equation}
(\omegas∗ 1|Ω\ {\omegaR}, ht) be the conditional belief that the long-run player is of type \omegas∗
\end{equation}

% Equation 343
\begin{equation}
event Ω\{\omegaR}, the rational long-run player’s strategy does not affect µ\sigma0,\sigma2
\end{equation}

% Equation 344
\begin{equation}
once (\sigma0, \sigma2) are given. Next, let L\sigma0,\sigma2 \subsetH\inftybe the set of all histories h where µ\sigma0,\sigma2
\end{equation}

% Equation 345
\begin{equation}
(\omegas∗ 1|Ω\ {\omegaR}, ht) \to1 as t \to\infty. By Lemma 9, Q\sigma0,\sigma2 (L\sigma0,\sigma2) = 1 for all (\sigma0, \sigma2) \inB1(s∗
\end{equation}

% Equation 346
\begin{equation}
d((\sigma0, \sigma2), (\sigma′ 0, \sigma′ 2)) = sup h\inH ( \infty
\end{equation}

% Equation 347
\begin{equation}
t=0
\end{equation}

% Equation 348
\begin{equation}
2t||(\sigma0(ht), \sigma2(ht)) −(\sigma′ 0(ht), \sigma′ 2(ht))||
\end{equation}

% Equation 349
\begin{equation}
This is the sup-norm over undominated short-run player strategies (\sigma0, \sigma2) : H \toB1(s∗
\end{equation}

% Equation 350
\begin{equation}
Endowing H\inftywith the product topology, we have that H is a dense subset of H\infty25. We can
\end{equation}

% Equation 351
\begin{equation}
1)H\infty, that is, the space B1(s∗ 1)H\infty is the set of all continuous functions from infinite sequences of signals H\inftyinto sequences of
\end{equation}

% Equation 352
\begin{equation}
1)\inftyunder the sup norm. Thus, since H\inftyhas countable dense subset H, a
\end{equation}

% Equation 353
\begin{equation}
1)H\infty, d) is compact. We are now ready to prove that ˆT(\zeta, \sigma∗ 0, \sigma∗
\end{equation}

% Equation 354
\begin{equation}
(\sigma∗ 0, \sigma∗ 2) (and hence also independent of \delta). Suppose for contradiction that there exists \zeta > 0 such that for each T \inN, there exist (\sigmaT 0 , \sigmaT 2 ) \inB1(s∗ 1)H and a set of histories ET(\zeta) \subsetH\infty such that Q\sigmaT 0 ,\sigmaT 2 (ET(\zeta)) > \zeta but µT(·|h) \inM\zeta for all h \inET(\zeta). Taking a subsequence if nec-
\end{equation}

% Equation 355
\begin{equation}
1)H, we have (\sigmaT 0 , \sigmaT 2 ) \to(\sigma\infty 0 , \sigma\infty 2 ) \inB1(s∗
\end{equation}

% Equation 356
\begin{equation}
\sigmaT 0 ,\sigmaT
\end{equation}

% Equation 357
\begin{equation}
(ET(\zeta)) > \zeta for all T ′ \geqT. Thus, since QT ′ is continuous in strategies as a finite-dimensional measure, passing (\sigmaT 0 , \sigmaT
\end{equation}

% Equation 358
\begin{equation}
to the limit (while fixing the time T ′ and the set of histories ET(\zeta)) gives Q \sigma\infty 0 ,\sigma\infty
\end{equation}

% Equation 359
\begin{equation}
for all T ′ sufficiently large; and then taking T ′ \to\inftygives Q\sigma\infty 0 ,\sigma\infty
\end{equation}

% Equation 360
\begin{equation}
for all T, we have a sequence of events {ET(\zeta)}T\inN such that Q\sigma\infty 0 ,\sigma\infty
\end{equation}

% Equation 361
\begin{equation}
Q\sigma\infty 0 ,\sigma\infty
\end{equation}

% Equation 362
\begin{equation}
lim sup T\to\inftyET(\zeta)
\end{equation}

% Equation 363
\begin{equation}
\geqlim sup n\to\infty
\end{equation}

% Equation 364
\begin{equation}
k=1 Q\sigma\infty 0 ,\sigma\infty
\end{equation}

% Equation 365
\begin{equation}
1\leqj,k\leqT ′ Q\sigma\infty 0 ,\sigma\infty 2 (Ej(\zeta) \capEk(\zeta)) \geqn2\zeta2
\end{equation}

% Equation 366
\begin{equation}
= \zeta2. Thus, for any history h \inlim supT\to\inftyET(\zeta), there is a sequence of times {Tn} such that µTn(·|h) /\inM\zeta for all n. Since \zeta2 < \zeta, this implies that µTn(·|h) \neq M\zeta2. Thus, for any h \in E\infty(\zeta), µt(·|h) /\inM\zeta2 for infinitely many T; but Q\sigma\infty 0 ,\sigma\infty 2 (E\infty(\zeta)) > \zeta2. But this contradicts Lemma 9 for the strategies (\sigma\infty 0 , \sigma\infty 2 ) \inB1(s∗
\end{equation}

% Equation 367
\begin{equation}
Note that if ht \inHt \eta and µt(·|ht) \inM0 then |µt(\omegaR|ht)|
\end{equation}

% Equation 368
\begin{equation}
p(\sigma∗ 0(ht), \sigma∗ 1(ht, \omegaR), \sigma∗ 2(ht)) −p(\sigma∗
\end{equation}

% Equation 369
\begin{equation}
1, \sigma∗
\end{equation}

% Equation 370
\begin{equation}
and (\sigma∗ 0(ht), \sigma∗ 2(ht)) \inB\eta(s∗ 1). Thus, since p(\sigma∗ 0(ht), \sigma∗ 1, \sigma∗
\end{equation}

% Equation 371
\begin{equation}
exists a strictly positive function \zeta(\eta) satisfying lim\eta\to0 \zeta(\eta) = 0 such that if ht \inHt
\end{equation}

% Equation 372
\begin{equation}
25Formally, H = S t Ht is isomorphic to the set of finite cylinders that generate H\infty.
\end{equation}

% Equation 373
\begin{equation}
µt(·|ht) \inM\zeta(\eta) then |µt(\omegaR|ht)|
\end{equation}

% Equation 374
\begin{equation}
p(\sigma∗ 0(ht), \sigma∗ 1(ht, \omegaR), \sigma∗ 2(ht)) −p(\sigma∗ 0(ht), s∗ 1, \sigma∗ 2(ht))
\end{equation}

% Equation 375
\begin{equation}
and (\sigma∗ 0(ht), \sigma∗ 2(ht)) \inB2\eta(s∗
\end{equation}

% Equation 376
\begin{equation}
Now fix any c > 0. If µt(\omegaR|ht) \geqc then
\end{equation}

% Equation 377
\begin{equation}
p(\sigma∗ 0(ht), \sigma∗ 1(ht, \omegaR), \sigma∗ 2(ht)) −p(\sigma∗ 0(ht), s∗ 1, \sigma∗ 2(ht))
\end{equation}

% Equation 378
\begin{equation}
Hence, as \eta \to0, Lemma 1 implies that ||\sigma∗ 1(ht, \omegaR) −s∗ 1||\to0, and hence (\sigma∗ 0(ht), \sigma∗ 2(ht)) \in
\end{equation}

% Equation 379
\begin{equation}
1) for some strictly positive function ξ1(\eta) satisfying ξ1(\eta) \to0. If instead µt(\omegaR|ht) < c then ||\sigma∗ 1(ht) −s∗ 1||\leq1 −\zeta(\eta) −c, and hence (\sigma∗ 0(ht), \sigma∗ 2(ht)) \inˆB\zeta(\eta)+c(s∗ 1). Taking ξ(\eta) =
\end{equation}

% Equation 380
\begin{equation}
Let \gamma be feasible in OT(\rho, ϕ) and supp(\gamma) not strictly u-CM. Let {(xi, yi)N i=1} \subsetsupp(\gamma) be a collection of pairs witnessing a violation and set \varepsilon = mini \gamma((xi, yi)). Define \gamma′ \in
\end{equation}

% Equation 381
\begin{equation}
\gamma′(x, y) =
\end{equation}

% Equation 382
\begin{equation}
\gamma(x, y) −\varepsilon if (x, y) \in
\end{equation}

% Equation 383
\begin{equation}
(xi, yi)N i=1
\end{equation}

% Equation 384
\begin{equation}
i=1
\end{equation}

% Equation 385
\begin{equation}
\gamma(x, y) + \varepsilon if (x, y) \in
\end{equation}

% Equation 386
\begin{equation}
i=1
\end{equation}

% Equation 387
\begin{equation}
(xi, yi)N i=1
\end{equation}

% Equation 388
\begin{equation}
\gamma(x, y)
\end{equation}

% Equation 389
\begin{equation}
Then \gamma′ \neq \gamma is feasible in OT(\rho, ϕ) and R u(x, y)d\gamma \leq R u(x, y)d\gamma′ (since {(xi, yi)N i=1} wit- nesses a violation of strict u-cyclical monotonicity), so \gamma does not uniquely solve OT(\rho, ϕ). Conversely, if \gamma is feasible in OT(\rho, ϕ) and strictly u-cyclically monotone, consider any \gamma′ \neq \gamma that is feasible in OT(\rho, ϕ). Since \gamma and \gamma′ are both feasible in OT(\rho, ϕ) and \gamma \neq \gamma′, there exists {(xi, yi)N i=1} \subsetsupp(\gamma) such that {(xi, yi+1)N i=1} \subsetsupp(\gamma′). (To see this, let (x1, y1) be any pair such that \gamma(x1, y1) > \gamma′(x1, y1). Since \gamma and \gamma′ transport the same mass into y1, there exists x2 such that \gamma(x2, y1) < \gamma′(x2, y1). But now, since \gamma and \gamma′ transport the same mass out of x2, there exists y2 such that \gamma(x2, y2) > \gamma′(x2, y2). Continuing in this
\end{equation}

% Equation 390
\begin{equation}
manner and using finiteness of X \times Y yields a cycle.) Let \varepsilon = mini \gamma′(xi, yi+1), and let \gamma′′(x, y) =
\end{equation}

% Equation 391
\begin{equation}
\gamma′(x, y) −\varepsilon if (x, y) \in
\end{equation}

% Equation 392
\begin{equation}
i=1
\end{equation}

% Equation 393
\begin{equation}
(xi, yi)N i=1
\end{equation}

% Equation 394
\begin{equation}
\gamma′(x, y) + \varepsilon if (x, y) \in
\end{equation}

% Equation 395
\begin{equation}
(xi, yi)N i=1
\end{equation}

% Equation 396
\begin{equation}
i=1
\end{equation}

% Equation 397
\begin{equation}
\gamma′(x, y)
\end{equation}

% Equation 398
\begin{equation}
Then \gamma′′ is feasible in OT(\rho, ϕ) and R u(x, y)d\gamma′′ > R u(x, y)d\gamma′ (since
\end{equation}

% Equation 399
\begin{equation}
(xi, yi)N i=1
\end{equation}

% Equation 400
\begin{equation}
in the strictly u-cyclically monotone set supp(\gamma)), so \gamma′ does not solve OT(\rho, ϕ). Thus, since no \gamma′ \neq \gamma solves OT(\rho, ϕ), and OT(\rho, ϕ) has a solution as a continuous maximization problem over a compact set, \gamma must uniquely solve OT(\rho, ϕ).
\end{equation}

% Equation 401
\begin{equation}
\geq¯u1, so suppose ¯vCM
\end{equation}

% Equation 402
\begin{equation}
< ¯u1, and fix \varepsilon < 2(¯u1 −¯vCM
\end{equation}

% Equation 403
\begin{equation}
µ0 with µ0(\omegaR) > 0, and an equilibrium (\sigma∗ 0, \sigma∗ 1, \sigma∗
\end{equation}

% Equation 404
\begin{equation}
1||< ς, and any (\alpha0, \alpha2) \inB(s1), we have u1(\alpha0, s1, \alpha2) < ¯vCM
\end{equation}

% Equation 405
\begin{equation}
1, and (\alphan 0, \alphan 2) \inB(˜sn
\end{equation}

% Equation 406
\begin{equation}
u1(\alphan
\end{equation}

% Equation 407
\begin{equation}
1, \alphan
\end{equation}

% Equation 408
\begin{equation}
+ ξ \leqlim sup n\to\inftyu1(\alphan
\end{equation}

% Equation 409
\begin{equation}
1, \alphan 2) \leq sup (\alpha0,\alpha2)\inB(s1) u1(\alpha0, s1, \alpha2) \leq¯vCM
\end{equation}

% Equation 410
\begin{equation}
Now, note that at any history ht where µt(\omegaR|ht) > 1−ς, we have ||\sigma∗ 1(ht)−\sigma∗ 1(ht, \omegaR)||< ς. Thus, by Lemmas 5 and 10, there exists ς > 0 such that at any history ht where µt(\omegaR|ht) >
\end{equation}

% Equation 411
\begin{equation}
1 −ς, we have u1(\sigma∗ 0, \sigma∗ 1, \sigma∗ 2|ht, \omegaR) < ¯vCM
\end{equation}

% Equation 412
\begin{equation}
+ \varepsilon/2. Since µt(\omegaR|ht, \omegaR) is a P-submartingale, (1 −P(µt(\omegaR|ht, \omegaR) > 1 −ς))(1 −ς) + P(µt(\omegaR|ht, \omegaR) > 1 −ς)(1) \geqµ0(\omegaR)
\end{equation}

% Equation 413
\begin{equation}
P(µt(\omegaR|ht, \omegaR) > 1 −ς) \geq1 −1 −µ0(\omegaR)
\end{equation}

% Equation 414
\begin{equation}
1 −1 −µ0(\omegaR)
\end{equation}

% Equation 415
\begin{equation}
+ 1 −µ0(\omegaR)
\end{equation}

% Equation 416
\begin{equation}
µ0(\omegaR) > 1 −
\end{equation}

% Equation 417
\begin{equation}
First, G(s1) cannot contain a cycle (\theta1, r1), (\theta2, r1), . . . , (\thetaK, rK), (\theta1, rK). To see this, sup- pose otherwise, and let \theta1 \prec. . . \prec\thetaK, without loss. Since rk \insupp(s1(\thetak))\capsupp(s1(\thetak+1)) for k \in{1, . . . , K −1} and rK \insupp(s1(\thetaK)), monotonicity requires r1 \prec. . . \precrK. But this gives a contradiction, since rK \insupp(s1(\theta1)) and r1 \insupp(s1(\theta2)).
\end{equation}

% Equation 418
\begin{equation}
suppose that G(s1) is connected, and let (\theta1, r1), (\theta2, r1), . . . , (\thetaK, rK) be any maximum path
\end{equation}

% Equation 419
\begin{equation}
states is almost identical.) Define the orders \prec\Theta on {\theta1, . . . , \thetaK} and \precR on {r1, . . . , rK} by \theta1 \prec\Theta . . . \prec\Theta \thetaK and r1 \precR . . . \precR rK. We claim that for any \theta \in\Theta \ {\theta1, . . . , \thetaK}, there exists k \in{1, . . . , K −1} such that
\end{equation}

% Equation 420
\begin{equation}
supp(s1(\theta)) = {rk}. Note such a state \theta is linked to at most one rk \in{r1, . . . , rK−1}, as if \theta is
\end{equation}

% Equation 421
\begin{equation}
k then appending \theta to both ends of the path from rk to r′
\end{equation}

% Equation 422
\begin{equation}
In addition, \theta cannot be linked to rK, as then it could be appended to the maximum path. Finally, \theta cannot be linked to both some rk \in{r1, . . . , rK−1} and some r \inR\{r1, . . . , rK−1}. For, if k = 1 then replacing (\theta1, r1) with (\theta, r), (\theta, r1) at the beginning of the maximum path would lengthen it; and if k \geq2 then the set of states {\thetak, \theta, \thetak+1} together with the set of actions {r, rk−1, rk, rk+1} would be a forbidden triple, as {rk−1, rk} \insupp(s1(\thetak)), {r, rk} \insupp(s1(\theta)), and {rk, rk+1} \insupp(s1(\thetak+1)) (see Figure 2). \theta1 \theta2 \theta \theta3
\end{equation}

% Equation 423
\begin{equation}
\theta1 \theta2 \theta \theta3
\end{equation}

% Equation 424
\begin{equation}
Figure 2: Each State \theta /\in{\theta1, . . . , \thetaK} Has Only One Neighbor Notes. If \theta /\in{\theta1, . . . , \thetaK} is linked to rk \in{r1, . . . , rK} and r /\in{r1, . . . , rK}, then {\thetak, \theta, \thetak+1} together with {r, rk−1, rk, rk+1} is a forbidden triple. Given this claim, we can extend \prec\Theta to \Theta by ordering each \theta \in\Theta\{\theta1, . . . , \thetaK} such that supp(s1(\theta)) = {rk} in between \thetak and \thetak+1 (and ordering multiple such states arbitrarily between \thetak and \thetak+1). Similarly, for any r \inR \ {r1, . . . , rK}, there exists k \in{2, . . . , K} such that rk is linked only to \thetak in G(s1). Extend \precR to R by ordering each such r in between rk−1 and rk. Note that for any k \geq2 and any r \insupp(s1(\thetak)), we have rk−1 ≾R r ≾R rk. This follows because if r /\in{r1, . . . , rK} then rk−1 ≾R r ≾R rk by construction, and if r = r˜k for some ˜k /\in{k −1, k}, then G(s1) contains a cycle starting with (\thetak, r˜k) and then following the maximum path back to \thetak. Finally, we claim that s1 is monotone with respect to (\succeq\Theta, \succeqR). To see this, fix any \theta \succ\Theta \theta′. Let ˜k = max{k : \theta \succeq\Theta \thetak}. If \theta \succ\Theta \theta˜k \succeq\Theta \theta′, then supp(s1(\theta)) = {r˜k} and r˜k \succeqR r
\end{equation}

% Equation 425
\begin{equation}
for all r \insupp(s1(\theta′)). If \theta = \theta˜k \succ\Theta \theta′, then r\theta˜k−1 is the lowest action in supp(s1(\theta)), and no action in supp(s1(\theta′)) is above r\theta˜k−1. Lastly, if \theta \succ\Theta \theta′ \succeq\Theta \theta˜k, then supp(s1(\theta)) = supp(s1(\theta′)) = {r˜k}. Thus, in all cases, monotonicity holds.
\end{equation}

% Equation 426
\begin{equation}
“Optimal Persuasion via Bi-Pooling.” Theoretical Economics 18 (1): 15–36.
\end{equation}

% Equation 427
\begin{equation}
Political Economy 132 (5): 1305–1337.
\end{equation}

% Equation 428
\begin{equation}
American Economic Review 100 (5): 2361–2382.
\end{equation}

% Equation 429
\begin{equation}
Econometrica 76 (1): 117–136.
\end{equation}

% Equation 430
\begin{equation}
suasion.” Journal of Political Economy 127 (5): 1993–2048.
\end{equation}

% Equation 431
\begin{equation}
Games and Economic Behavior 63 (2): 498–526.
\end{equation}

% Equation 432
\begin{equation}
nomics 118 (3): 785–814.
\end{equation}

% Equation 433
\begin{equation}
Information.” Econometrica 81 (5): 1737–1767.
\end{equation}

% Equation 434
\begin{equation}
(5): 1627–1641.
\end{equation}

% Equation 435
\begin{equation}
Economic Studies 85 (4): 2253–2282.
\end{equation}

% Equation 436
\begin{equation}
straints by Linking Decisions.” Econometrica 75 (1): 241–257.
\end{equation}

% Equation 437
\begin{equation}
Economic Review 101 (6): 2590–2615.
\end{equation}

% Equation 438
\begin{equation}
Studies 76 (4): 1359–1395.
\end{equation}

% Equation 439
\begin{equation}
and Majorization: Economic Applications.” Econometrica 89 (5): 1671–1700.
\end{equation}

% Equation 440
\begin{equation}
Theoretical Economics 13 (2): 607–635.
\end{equation}

% Equation 441
\begin{equation}
(7): .
\end{equation}

% Equation 442
\begin{equation}
Economic Studies 78 (4): 1400–1425.
\end{equation}

% Equation 443
\begin{equation}
Margaria, Chiara, and Alex Smolin. 2018. “Dynamic Communication with Biased
\end{equation}

% Equation 444
\begin{equation}
145 (6): 2241–2259.
\end{equation}

% Equation 445
\begin{equation}
agent problems.” Journal of mathematical economics 10 (1): 67–81.
\end{equation}

% Equation 446
\begin{equation}
132 (10): .
\end{equation}

% Equation 447
\begin{equation}
Economy 118 (5): 949–987.
\end{equation}

% Equation 448
\begin{equation}
Games.” Journal of Economic Theory 148 (2): 502–534.
\end{equation}

% Equation 449
\begin{equation}
(3): 355–374.
\end{equation}

% Equation 450
\begin{equation}
play.” Journal of Economic Theory 145 (1): 42–62.
\end{equation}

\end{document}
