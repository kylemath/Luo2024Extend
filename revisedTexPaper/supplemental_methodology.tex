\documentclass[12pt,a4paper]{article}

\usepackage[utf8]{inputenc}
\usepackage[T1]{fontenc}
\usepackage{amsmath,amssymb,amsthm}
\usepackage{enumitem}
\usepackage{geometry}
\usepackage{hyperref}
\usepackage{booktabs}
\usepackage{array}
\usepackage{graphicx}
\usepackage{xcolor}
\usepackage{setspace}
\usepackage{microtype}
\usepackage{fancyhdr}

\geometry{margin=1in, headheight=14.5pt}
\onehalfspacing

\pagestyle{fancy}
\fancyhf{}
\rhead{\small Supplemental: Human--AI Research Process}
\lhead{\small \thepage}
\renewcommand{\headrulewidth}{0.4pt}

\hypersetup{
    colorlinks=true,
    linkcolor=blue!70!black,
    citecolor=green!50!black,
    urlcolor=blue!60!black
}

% Stats macros
%% Auto-generated by extract_stats.py — do not edit manually
%% Generated from analysis scripts output
%% Updated 2026-02-19: corrected for simultaneous-move timing (C04)

%% Paper metadata
\newcommand{\PaperDate}{February 17, 2026}

%% Game parameters
\newcommand{\BaseAlpha}{0.3}
\newcommand{\BaseBeta}{0.5}
\newcommand{\BasePiG}{0.625}
\newcommand{\BasePiB}{0.375}
\newcommand{\SRBeliefAfterG}{0.70}
\newcommand{\SRBeliefAfterB}{0.50}
\newcommand{\BRThreshold}{0.40}
\newcommand{\SRThreshold}{0.60}
\newcommand{\BRPayoff}{0.625}
\newcommand{\PayoffGapAbsolute}{0.094}
\newcommand{\PayoffGapPayoff}{0.206}

%% Computational parameters
\newcommand{\KLMonteCarloN}{500}
\newcommand{\KLMonteCarloPeriods}{5000}
\newcommand{\OTStabilityMargin}{0.3}
\newcommand{\FilterGridSize}{30}

%% Formulas
\newcommand{\BeliefGapFormula}{\frac{2\alpha\beta|1-\alpha-\beta|}{(\alpha+\beta)^2}}

%% Legacy macros (aliased from original paper for section compatibility)
\newcommand{\TVMean}{\statTvMean}
\newcommand{\BeliefGapBaseline}{\statGapBaseline}
\newcommand{\PayoffStationary}{\statPayoffStationary}
\newcommand{\PayoffFiltered}{\statPayoffFiltered}
\newcommand{\PayoffOverestimation}{\statOverestimationPct}
\newcommand{\SRDisagreement}{\statSrDisagreementPct}
\newcommand{\OTStabilityPct}{\statOtStablePct}
\newcommand{\FilterCorrelation}{\statFilterCorrelation}
\newcommand{\SupermodFraction}{\statSupermodValid}
\newcommand{\SupermodTotal}{\statSupermodTotal}

\newcommand{\statDecayConfirmed}{True}
\newcommand{\statFilterCorrelation}{0.63}
\newcommand{\statGapBaseline}{0.094}
\newcommand{\statGapFormula}{2ab|1-a-b|/(a+b)^2}
\newcommand{\statIidMeanCount}{8.1}
\newcommand{\statKlBoundHolds}{True}
\newcommand{\statMarkovMeanCount}{12.7}
\newcommand{\statMinMargin}{0.8}
\newcommand{\statOtStablePct}{100}
\newcommand{\statOverestimationPct}{36.3}
\newcommand{\statPayoffFiltered}{0.569}
\newcommand{\statPayoffStationary}{0.775}
\newcommand{\statSrDisagreementPct}{37.2}
\newcommand{\statSupermodTotal}{24}
\newcommand{\statSupermodValid}{4}
\newcommand{\statTvMean}{0.412}
\newcommand{\statTvMeanHighPersist}{0.460}


\begin{document}

\begin{center}
{\LARGE\bfseries Supplemental Material:\\[6pt]
Human--AI Research Process}

\vspace{1cm}

{\large
Kyle Elliott Mathewson\textsuperscript{1}
}

\vspace{0.5cm}

{\large February 2026}

\vspace{0.3cm}

\noindent\textsuperscript{1}Faculty of Science, University of Alberta
\end{center}

\vspace{1cm}

\begin{abstract}
\noindent
This supplemental document describes the human--AI collaboration process behind the companion paper ``Extending Marginal Reputation to Persistent Markovian States.'' The main paper presents the mathematical results as a self-contained contribution; this supplement documents the research process, including the agent architecture, computational testing framework, timeline, and reflections on AI-assisted mathematical research.
\end{abstract}

\vspace{0.5cm}

\tableofcontents
\newpage

% ================================================================
\section{Overview}

The companion paper extends the main result of Luo \& Wolitzky (2024) from i.i.d.\ states to persistent Markovian states, introducing the concepts of belief-robustness and the Markov commitment payoff. This supplement documents the research process that produced those results: an initial AI-assisted conjecture phase, expert critique, systematic computational verification, and iterative revision.

% ================================================================
\section{Phase 1: Initial Conjecture (Feb 16, 5:00--9:30 PM)}

The initial extension was developed under a five-hour time constraint. Five AI agents---four instances of Claude Opus 4.6 and one instance of Claude Sonnet 4.5---worked under human coordination, with each agent assigned a specialized role.

\begin{center}
\renewcommand{\arraystretch}{1.2}
\begin{tabular}{@{}llp{8cm}@{}}
\toprule
\textbf{Agent} & \textbf{Role} & \textbf{Key Contribution} \\
\midrule
Sonnet 4.5 Reader & Paper parsing & Multi-level summaries, equation extraction \\
Agent 840 (Opus) & First parse & Identified lifted-state approach, 5 interpretations \\
Agent 841 (Opus) & Proof coordinator & Directed 4 parallel subagents \\
Agent 852 (Opus) & Paper author & 26-page \LaTeX\ document \\
Agent 860 (Opus) & Peer reviewer & Identified continuation value subtlety \\
\bottomrule
\end{tabular}
\end{center}

Within the time window, this architecture produced a 26-page paper, an interactive web demonstration, and a social media summary. The paper proposed that Theorem~1 extends to Markov states via the lifted state $\tilde\theta_t = (\theta_t, \theta_{t-1})$, correctly identifying several key mathematical tools---the process-independence of the KL chain rule, the well-defined stationary distribution on the lifted space, and the filter stability argument.

% ================================================================
\section{Phase 2: Expert Critique (Feb 16, 10:00--11:00 PM)}

Within one hour of submission, Daniel Luo---co-author of Luo \& Wolitzky (2024)---posted two threads of detailed technical feedback comprising 15 distinct points.

The most consequential observations identified a single core issue: the i.i.d.\ assumption disciplines short-run player information sets about the state. Under Markov dynamics with a state-revealing strategy, short-run player beliefs are given by the filtering distribution $F(\cdot|\theta_t)$ rather than the stationary distribution $\pi$, causing the Nash correspondence to become state-contingent.

The single most clarifying observation was:

\begin{quote}
\emph{``To make it clear: suppose $s_1$ just takes an action that reveals the state. In the iid case, this won't affect SR beliefs. But in the Markov case, this can cause beliefs to never settle into the stationary distribution.''} --- Daniel Luo
\end{quote}

% ================================================================
\section{Phase 3: Computational Testing Framework}

To determine precisely which elements of the proof extend and which require modification, we designed a systematic computational investigation organized into seven analysis areas (SA1--SA7).

\subsection{Agent Architecture}

The computational framework employed a hierarchical agent architecture:
\begin{itemize}[nosep]
\item A reusable Python class (\texttt{Agent}) supporting task assignment, report generation, and hierarchical delegation.
\item Seven subagents (SA1--SA7), each producing a synthesized report.
\item Twenty-one sub-subagent scripts with detailed task specifications.
\end{itemize}

Each analysis script was designed around a \emph{hypothesis} (what the paper claims), a \emph{counter-hypothesis} (what the critique implies), and a \emph{test} (what the simulation checks).

\subsection{Execution}

Four parallel subagents executed all 21 scripts simultaneously:

\begin{center}
\renewcommand{\arraystretch}{1.2}
\begin{tabular}{@{}llll@{}}
\toprule
\textbf{Batch} & \textbf{Modules} & \textbf{Scripts} & \textbf{Runtime} \\
\midrule
Batch 1 & SA1 (Beliefs) + SA2 (State-revealing) & 6 & 202s \\
Batch 2 & SA3 (KL bound) + SA4 (Filter stability) & 6 & 143s \\
Batch 3 & SA5 (OT sensitivity) + SA6 (Nash dynamics) & 6 & 19s \\
Batch 4 & SA7 (Monotonicity) + orchestrator & 3+2 & 47s \\
\bottomrule
\end{tabular}
\end{center}

\subsection{Results Summary}

The findings divided sharply:
\begin{itemize}[nosep]
\item \textbf{Surviving claims}: KL counting bound (SA3), filter stability (SA4), OT robustness (SA5), monotonicity characterization (SA7).
\item \textbf{Failing claims}: SR belief convergence to $\pi$ (SA1 refuted), static Nash correspondence (SA6 showed \SRDisagreement\% disagreement), original payoff bound (SA6 showed \PayoffOverestimation\% overestimation).
\end{itemize}

% ================================================================
\section{Phase 4: Manuscript Revision}

The computational evidence guided a structured revision:
\begin{itemize}[nosep]
\item The paper was decomposed into 12 modular \LaTeX\ section files assembled by a master document.
\item All quantitative claims were drawn from an auto-generated statistics file (\texttt{stats.tex}).
\item An automated pipeline (\texttt{generate\_paper.sh}) executes the full sequence: analysis scripts $\to$ statistics extraction $\to$ \LaTeX\ compilation.
\end{itemize}

% ================================================================
\section{Timeline}

\begin{center}
\renewcommand{\arraystretch}{1.2}
\begin{tabular}{@{}lll@{}}
\toprule
\textbf{Time} & \textbf{Phase} & \textbf{Key Event} \\
\midrule
Feb 16, 5:00--9:30 PM & Phase 1 & 5 AI agents produce initial draft \\
Feb 16, 10:00--11:00 PM & Phase 2 & Expert posts 15-point technical feedback \\
Feb 17, 12:00--1:00 AM & Phase 3 & Combined review, agent hierarchy designed \\
Feb 17, 1:00--2:00 AM & Phase 4 & 21 scripts, 40 figures, 28 reports \\
Feb 17, 2:00--3:00 AM & Phase 5 & Corrected paper compiled \\
\bottomrule
\end{tabular}
\end{center}

% ================================================================
\section{Reflections on AI-Assisted Mathematical Research}

Several observations emerge from this process:

\begin{enumerate}
\item \textbf{AI + expert critique}: The combination of AI-assisted rapid exploration and human expert critique proved more productive than either alone. AI agents identified the lifted-state approach and the process-independent tools; the human expert identified the semantic gap between these tools and their game-theoretic interpretation.

\item \textbf{Computational triage}: Rather than attempting to determine \emph{a priori} whether the critique invalidated the entire approach, the seven analysis modules produced quantitative evidence that cleanly separated the surviving claims from the failing ones.

\item \textbf{Scalable architecture}: The three-level hierarchical delegation (orchestrator, subagents, sub-subagents) with structured report aggregation scaled effectively.

\item \textbf{Tools vs.\ semantics}: AI systems can identify correct mathematical tools but may fail to interpret their role correctly within a larger argument. The KL bound \emph{is} process-independent, but the proof's \emph{use} of it depends on what it means for a period to be ``non-distinguishing''---a semantic subtlety that required domain expertise.
\end{enumerate}

% ================================================================
\section*{Repository Structure}

\begin{small}
\begin{verbatim}
revisedTexPaper/
+-- main.tex              # Main paper
+-- supplemental_methodology.tex  # This document
+-- stats.tex             # Auto-generated statistics macros
+-- sections/             # Modular .tex files
+-- figures/              # Diagnostic figures
+-- scripts/              # 7 analysis + automation scripts
+-- response_letter_refineai.tex  # Response to reviewer feedback
\end{verbatim}
\end{small}

\end{document}
