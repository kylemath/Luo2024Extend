\documentclass[12pt,a4paper]{article}

% ========== Packages ==========
\usepackage[utf8]{inputenc}
\usepackage[T1]{fontenc}
\usepackage{amsmath,amssymb,amsthm,mathtools}
\usepackage{enumitem}
\usepackage{geometry}
\usepackage{hyperref}
\usepackage{cleveref}
\usepackage{booktabs}
\usepackage{array}
\usepackage{graphicx}
\usepackage{tikz}
\usepackage{thmtools}
\usepackage{xcolor}
\usepackage{natbib}
\usepackage{setspace}
\usepackage{microtype}
\usepackage{fancyhdr}

\geometry{margin=1in, headheight=14.5pt}
\onehalfspacing

% ========== Theorem Environments ==========
\theoremstyle{plain}
\newtheorem{theorem}{Theorem}[section]
\newtheorem{proposition}[theorem]{Proposition}
\newtheorem{lemma}[theorem]{Lemma}
\newtheorem{corollary}[theorem]{Corollary}

\theoremstyle{definition}
\newtheorem{definition}[theorem]{Definition}
\newtheorem{example}[theorem]{Example}
\newtheorem{assumption}[theorem]{Assumption}
\newtheorem{remark}[theorem]{Remark}

% ========== Custom Commands ==========
\newcommand{\R}{\mathbb{R}}
\newcommand{\N}{\mathbb{N}}
\newcommand{\E}{\mathbb{E}}
\newcommand{\Prob}{\mathbb{P}}
\newcommand{\KL}{D_{\mathrm{KL}}}
\newcommand{\TV}[2]{\left\|#1 - #2\right\|}
\newcommand{\supp}{\operatorname{supp}}
\newcommand{\conv}{\operatorname{conv}}
\DeclareMathOperator*{\argmax}{argmax}
\newcommand{\OT}{\mathrm{OT}}
\DeclareMathOperator*{\liminff}{\underline{\lim}}
\DeclareMathOperator*{\limsupf}{\overline{\lim}}

% ========== Auto-generated Statistics ==========
%% Auto-generated by extract_stats.py — do not edit manually
%% Generated from analysis scripts output
%% Updated 2026-02-19: corrected for simultaneous-move timing (C04)

%% Paper metadata
\newcommand{\PaperDate}{February 17, 2026}

%% Game parameters
\newcommand{\BaseAlpha}{0.3}
\newcommand{\BaseBeta}{0.5}
\newcommand{\BasePiG}{0.625}
\newcommand{\BasePiB}{0.375}
\newcommand{\SRBeliefAfterG}{0.70}
\newcommand{\SRBeliefAfterB}{0.50}
\newcommand{\BRThreshold}{0.40}
\newcommand{\SRThreshold}{0.60}
\newcommand{\BRPayoff}{0.625}
\newcommand{\PayoffGapAbsolute}{0.094}
\newcommand{\PayoffGapPayoff}{0.206}

%% Computational parameters
\newcommand{\KLMonteCarloN}{500}
\newcommand{\KLMonteCarloPeriods}{5000}
\newcommand{\OTStabilityMargin}{0.3}
\newcommand{\FilterGridSize}{30}

%% Formulas
\newcommand{\BeliefGapFormula}{\frac{2\alpha\beta|1-\alpha-\beta|}{(\alpha+\beta)^2}}

%% Legacy macros (aliased from original paper for section compatibility)
\newcommand{\TVMean}{\statTvMean}
\newcommand{\BeliefGapBaseline}{\statGapBaseline}
\newcommand{\PayoffStationary}{\statPayoffStationary}
\newcommand{\PayoffFiltered}{\statPayoffFiltered}
\newcommand{\PayoffOverestimation}{\statOverestimationPct}
\newcommand{\SRDisagreement}{\statSrDisagreementPct}
\newcommand{\OTStabilityPct}{\statOtStablePct}
\newcommand{\FilterCorrelation}{\statFilterCorrelation}
\newcommand{\SupermodFraction}{\statSupermodValid}
\newcommand{\SupermodTotal}{\statSupermodTotal}

\newcommand{\statDecayConfirmed}{True}
\newcommand{\statFilterCorrelation}{0.63}
\newcommand{\statGapBaseline}{0.094}
\newcommand{\statGapFormula}{2ab|1-a-b|/(a+b)^2}
\newcommand{\statIidMeanCount}{8.1}
\newcommand{\statKlBoundHolds}{True}
\newcommand{\statMarkovMeanCount}{12.7}
\newcommand{\statMinMargin}{0.8}
\newcommand{\statOtStablePct}{100}
\newcommand{\statOverestimationPct}{36.3}
\newcommand{\statPayoffFiltered}{0.569}
\newcommand{\statPayoffStationary}{0.775}
\newcommand{\statSrDisagreementPct}{37.2}
\newcommand{\statSupermodTotal}{24}
\newcommand{\statSupermodValid}{4}
\newcommand{\statTvMean}{0.412}
\newcommand{\statTvMeanHighPersist}{0.460}


% ========== Header/Footer ==========
\pagestyle{fancy}
\fancyhf{}
\rhead{\small Extending Marginal Reputation to Markov States}
\lhead{\small \thepage}
\renewcommand{\headrulewidth}{0.4pt}

% ========== Colors for hyperref ==========
\hypersetup{
    colorlinks=true,
    linkcolor=blue!70!black,
    citecolor=green!50!black,
    urlcolor=blue!60!black
}

% ================================================================
\begin{document}

% ========== Title Page ==========
\begin{titlepage}
\centering
\vspace*{2cm}

{\LARGE\bfseries Extending Marginal Reputation\\[6pt]
to Persistent Markovian States}

\vspace{1.5cm}

{\large
Kyle Elliott Mathewson\textsuperscript{1} \\[6pt]
with AI Collaborators
}

\vspace{1.5cm}

{\large \PaperDate}

\vspace{1cm}

\noindent\textsuperscript{1}Faculty of Science, University of Alberta

\vspace{2cm}

\begin{abstract}
\noindent
We extend the main result (Theorem~1) of Luo \& Wolitzky (2024), ``Marginal Reputation,'' from i.i.d.\ states to persistent Markovian states. The extension reveals a new phenomenon: when the Stackelberg strategy reveals the state, short-run player beliefs permanently deviate from the stationary distribution, causing the Nash correspondence to become state-contingent. We introduce the concept of \emph{belief-robustness} and present two results. \textbf{Theorem~1\textsuperscript{$\prime$}} (belief-robust case): when short-run best responses are invariant to the filtering belief $F(\cdot|\theta)$, the original commitment payoff $V(s_1^*)$ holds exactly under Markov states with no correction. \textbf{Theorem~1\textsuperscript{$\prime\prime$}} (general case): the \emph{Markov commitment payoff} $V_{\mathrm{Markov}}(s_1^*) = \sum_\theta \pi(\theta) \cdot \inf_{B(s_1^*, F(\cdot|\theta))} u_1 \leq V(s_1^*)$ provides the appropriate bound, with equality if and only if the game is belief-robust. The gap $V(s_1^*) - V_{\mathrm{Markov}}$ quantifies the \emph{cost of persistence in reputation games}---a new economic object measuring how state persistence enables short-run players to condition behavior on the revealed state. For the deterrence game with baseline parameters ($\alpha=\BaseAlpha$, $\beta=\BaseBeta$), this cost is \PayoffOverestimation\% of the i.i.d.\ payoff. Our framework interpolates continuously between i.i.d.\ (Luo--Wolitzky) and perfectly persistent (Pei 2020) states. The paper also documents the human--AI research process, including computational verification across seven analysis modules with eight diagnostic figures.
\end{abstract}

\vspace{0.5cm}

\noindent\textbf{Keywords:} Reputation, repeated games, Markov states, optimal transport, belief-robustness, corrected bounds

\vspace{0.5cm}

\noindent\textbf{Original Paper:} ``Marginal Reputation'' by Daniel Luo and Alexander Wolitzky, MIT Department of Economics, December 2024.

\end{titlepage}

% ========== Table of Contents ==========
\tableofcontents
\newpage

% ========== Main Sections ==========
% ================================================================
\section{Introduction}\label{sec:intro}

Luo \& Wolitzky (2024) establish a striking connection between reputation theory in repeated games and optimal transport theory. Their main result, Theorem~1, shows that a patient long-run player can secure her \emph{commitment payoff} $V(s_1^*)$ in any Nash equilibrium, provided her Stackelberg strategy $s_1^*$ is \emph{confound-defeating} and \emph{not behaviorally confounded}. Throughout their analysis, states are drawn \textbf{i.i.d.\ across periods}. The authors note (footnote~9) that the extension to persistent states is an open question.

\subsection{The Challenge of Markov States}

The extension from i.i.d.\ to Markov states introduces a fundamental new phenomenon. We employ a \emph{lifted state} construction $\tilde\theta_t = (\theta_t, \theta_{t-1})$. Since the original chain already possesses a stationary distribution $\pi$, the purpose of lifting is not to create stationarity but to encode Markov private information into a type space $\tilde\Theta = \Theta \times \Theta$ with joint stationary distribution $\tilde\rho$, playing the role of the exogenous signal distribution in Luo--Wolitzky's optimal transport formulation. However, the extension is not a straightforward substitution: under the simultaneous-move timing of Luo--Wolitzky (2024, Section~3.1), the short-run player does not observe the long-run player's current action before choosing. When the Stackelberg strategy is state-revealing (e.g., $s_1^*(G)=A$, $s_1^*(B)=F$), the public history reveals past states, so the short-run player's belief about the current payoff-relevant state $\theta_t$ is the filtering distribution $F(\cdot|\theta_{t-1})$ rather than the stationary distribution $\pi$. This creates a permanent structural gap: short-run behavior becomes state-contingent, and the Nash correspondence $B(s_1^*)$ must be replaced by a state-dependent object $B(s_1^*, F(\cdot|\theta_{t-1}))$.

\subsection{Main Results}

These observations lead to two main theorems. Theorem~1$'$ addresses the \emph{belief-robust} case: when the short-run player's best-response set $B(s_1^*, F(\cdot|\theta))$ is constant across states $\theta$---a condition we call \emph{belief-robustness}---the i.i.d.\ bound $V(s_1^*)$ holds exactly. The entire proof machinery (KL bound, OT robustness, monotonicity) applies without modification; belief-robustness ensures the filtering belief gap is irrelevant.

Theorem~1$''$ handles the general case. For all supermodular games with Markov states, a corrected bound holds:
\[
    V_{\mathrm{Markov}}(s_1^*) := \sum_{\theta' \in \Theta} \pi(\theta') \cdot \inf_{(\alpha_0, \alpha_2) \in B(s_1^*, F(\cdot|\theta'))} \sum_{\theta \in \Theta} F(\theta|\theta') \cdot u_1\bigl(\theta,\, s_1^*(\theta, \theta'),\, \alpha_2\bigr),
\]
where $\theta'$ indexes the previous state (determining SR's belief) and $\theta$ indexes the current state (entering the payoff). The bound satisfies $V_{\mathrm{Markov}} = V(s_1^*)$ if and only if the game is belief-robust. The difference $V(s_1^*) - V_{\mathrm{Markov}}$---the \emph{effect of persistence}---can be positive or negative, quantifying how state persistence affects reputation value. This is a new economic object that the i.i.d.\ framework cannot capture.

\subsection{Outline}

Section~\ref{sec:model} presents the model with the lifted state construction. Section~\ref{sec:belief_robust} introduces the key new concept of belief-robustness. Section~\ref{sec:theorems} states the two main theorems. Section~\ref{sec:proof} contains the proof sketch, tracing each step of the Luo--Wolitzky argument and identifying where modifications are needed for Markov states. Section~\ref{sec:supermodular} extends the supermodular case. Section~\ref{sec:example} works out the deterrence game in both belief-robust and non-belief-robust versions. Section~\ref{sec:interpolation} discusses the continuous interpolation between i.i.d.\ and persistent states and the economic implications. Section~\ref{sec:discussion} discusses open questions. Appendix~\ref{app:kl} verifies the KL chain rule and filter stability. Appendix~\ref{app:computational} documents the computational framework.

% ================================================================
\section{The Extended Model}\label{sec:model}

We maintain all notation and conventions from Luo \& Wolitzky (2024, Sections~3.1--3.2), modifying only the state process.

\subsection{State Process}

Let $\Theta$ be a finite set.

\begin{assumption}[Markov States]\label{ass:markov}
The state $\theta_t \in \Theta$ follows a \textbf{stationary ergodic Markov chain} with:
\begin{enumerate}[label=(\alph*)]
    \item Transition kernel $F(\cdot | \theta)$ for each $\theta \in \Theta$, so that $\Prob(\theta_{t+1} = \theta' | \theta_t = \theta) = F(\theta' | \theta)$.
    \item Unique stationary distribution $\pi \in \Delta(\Theta)$ satisfying
    \begin{equation}\label{eq:stationary}
        \pi(\theta) = \sum_{\theta' \in \Theta} \pi(\theta') F(\theta | \theta') \quad \text{for all } \theta \in \Theta.
    \end{equation}
    \item The chain is \textbf{irreducible and aperiodic} (ensuring ergodicity).
\end{enumerate}
\end{assumption}

\begin{remark}
When $F(\cdot | \theta) = \pi(\cdot)$ for all $\theta$, the chain has no memory and we recover the i.i.d.\ case of the original paper. The two-state case with $\Theta = \{G, B\}$ is parameterized by $\alpha = \Prob(B|G)$ and $\beta = \Prob(G|B)$, giving $\pi(G) = \beta/(\alpha+\beta)$.
\end{remark}

\subsection{Lifted State Space}\label{subsec:lifted}

The central construction is the \emph{lifted state}:

\begin{definition}[Lifted State]\label{def:lifted}
Define
\begin{equation}\label{eq:lifted}
    \tilde\theta_t \;=\; (\theta_t,\, \theta_{t-1}) \;\in\; \tilde\Theta \;=\; \Theta \times \Theta.
\end{equation}
\end{definition}

The process $(\tilde\theta_t)_{t \geq 1}$ is itself a Markov chain on $\tilde\Theta$ with transition probabilities
\begin{equation}
    \tilde F\bigl((\theta', \theta) \,\big|\, (\theta, \theta'')\bigr) \;=\; F(\theta' | \theta)
\end{equation}
and stationary distribution
\begin{equation}\label{eq:lifted_stationary}
    \tilde\rho(\theta, \theta') \;=\; \pi(\theta') \cdot F(\theta | \theta').
\end{equation}

\begin{proposition}\label{prop:lifted_ergodic}
Under Assumption~\ref{ass:markov}, the lifted chain $(\tilde\theta_t)$ on $\tilde\Theta$ is ergodic with unique stationary distribution $\tilde\rho$.
\end{proposition}

\begin{proof}
Irreducibility of the original chain ensures connectivity of the lifted chain: for any lifted states $(\theta_a, \theta_b)$ and $(\theta_d, \theta_c)$, there exists a finite path with positive probability connecting them. Aperiodicity of the original chain implies aperiodicity of the lifted chain. Uniqueness of $\tilde\rho$ follows from the Perron--Frobenius theorem.
\end{proof}

\begin{remark}[Purpose of the Lifting]
The lifted state provides a Markov structure on which the optimal transport framework and cyclical monotonicity characterizations apply. The key property is that $\tilde\theta_t$ has a \emph{fixed, known} stationary distribution $\tilde\rho$, playing precisely the role of the i.i.d.\ signal distribution $\rho$ in the original paper.
\end{remark}

\subsection{Stage Game}

The stage game is identical to Luo \& Wolitzky's Section~3.1, except:
\begin{enumerate}[label=(\roman*)]
    \item The long-run player's private information each period is $\tilde\theta_t = (\theta_t, \theta_{t-1})$.
    \item A stage-game strategy for player~1 is $s_1 : \tilde\Theta \to \Delta(A_1)$, a \emph{Markov strategy}.
    \item Payoffs depend on the current state: $u_1(\theta_t, a_1, \alpha_2)$.
\end{enumerate}

We restrict throughout to payoffs $u_1(\theta_t, a_1, \alpha_2)$ that depend on $\theta_t$ alone (not the full lifted state $\tilde\theta_t$). This covers all standard applications---deterrence, trust, signaling---and avoids unmotivated generalization.

\subsection{Joint Distribution and Commitment Types}

Under Markov strategy $s_1$ and the stationary distribution $\tilde\rho$, the joint distribution over $(\tilde\theta, a_1)$ is:
\begin{equation}\label{eq:joint}
    \gamma(s_1)[\tilde\theta, a_1] \;=\; \tilde\rho(\tilde\theta) \cdot s_1(\tilde\theta)[a_1].
\end{equation}

A commitment type $\omega_{s_1} \in \Omega$ plays Markov strategy $s_1 : \tilde\Theta \to \Delta(A_1)$ every period. The type space $\Omega$ is countable with full-support prior $\mu_0 \in \Delta(\Omega)$.

% ================================================================
\section{Belief-Robustness: The Key New Concept}\label{sec:belief_robust}

This section introduces the central new concept needed for the Markov extension. The issue is simple: when the Stackelberg strategy reveals the state, short-run players learn $\theta_t$ and form beliefs about $\theta_{t+1}$ using the filtering distribution $F(\cdot|\theta_t)$, which generically differs from the stationary distribution $\pi$.

\subsection{Filtering Beliefs}

\begin{definition}[Filtering Belief]\label{def:filtering}
Given a state-revealing Stackelberg strategy $s_1^*$ (i.e., $s_1^*(\theta) \neq s_1^*(\theta')$ for $\theta \neq \theta'$), the \textbf{filtering belief} in state $\theta$ is
\begin{equation}
    F(\cdot|\theta_t) = \Prob(\theta_{t+1} = \cdot \mid \theta_t),
\end{equation}
the one-step-ahead predictive distribution conditional on the current state.
\end{definition}

For the two-state chain $\Theta = \{G, B\}$ with parameters $(\alpha, \beta)$:
\begin{align}
    F(G|G) &= 1-\alpha, \qquad F(G|B) = \beta.
\end{align}
The stationary distribution gives $\pi(G) = \beta/(\alpha+\beta)$. The gap between the filtering belief and the stationary distribution is:
\begin{equation}\label{eq:belief_gap}
    \E\bigl[|F(G|\theta_t) - \pi(G)|\bigr] = \frac{2\alpha\beta|1-\alpha-\beta|}{(\alpha+\beta)^2}.
\end{equation}
This equals zero \textbf{if and only if} $\alpha + \beta = 1$, which is precisely the i.i.d.\ case. For the baseline parameters $(\alpha=\BaseAlpha, \beta=\BaseBeta)$, the expected gap is \BeliefGapBaseline.

\subsection{The Belief-Robustness Condition}

\begin{definition}[Belief-Robustness]\label{def:belief_robust}
A game $(u_1, u_2)$ with Stackelberg strategy $s_1^*$ and Markov chain $(\Theta, F)$ is \textbf{belief-robust} if the short-run player Nash correspondence satisfies
\begin{equation}\label{eq:belief_robust}
    B(s_1^*, F(\cdot|\theta)) = B(s_1^*, F(\cdot|\theta')) \quad \text{for all } \theta, \theta' \in \Theta.
\end{equation}
\end{definition}

In words: the SR best-response set does not change when SR learns the current state. Under this condition, the belief gap documented in~\eqref{eq:belief_gap} is irrelevant---SR plays the same regardless.

\subsection{When Does Belief-Robustness Hold?}

For the deterrence game with SR threshold $\mu^*$ (the belief level at which SR is indifferent between cooperating and defecting):

\begin{proposition}\label{prop:br_condition}
Belief-robustness holds if and only if
\begin{equation}\label{eq:br_interval}
    \mu^* \notin \bigl[\min_\theta F(G|\theta),\; \max_\theta F(G|\theta)\bigr] = [\beta,\; 1-\alpha].
\end{equation}
\end{proposition}

\begin{proof}
The SR best response depends on whether $F(G|\theta_t) \gtrless \mu^*$. If $\mu^* < \beta$, then $F(G|\theta_t) \geq \beta > \mu^*$ for all $\theta_t$, so SR always cooperates. If $\mu^* > 1-\alpha$, then $F(G|\theta_t) \leq 1-\alpha < \mu^*$ for all $\theta_t$, so SR always defects. In either case, $B(s_1^*, F(\cdot|\theta))$ is constant. Conversely, if $\mu^* \in [\beta, 1-\alpha]$, there exist states $\theta, \theta'$ with $F(G|\theta) > \mu^* > F(G|\theta')$, so SR cooperates after $\theta$ and defects after $\theta'$.
\end{proof}

\begin{remark}[Economic Interpretation]
Belief-robustness fails when the SR indifference threshold lies between the conditional beliefs for different states. This happens when:
\begin{enumerate}[label=(\roman*)]
    \item The game has belief-sensitive SR behavior (threshold near $\pi$).
    \item The chain is persistent enough that $F(\cdot|\theta)$ varies substantially across states.
    \item The strategy reveals state information to SR.
\end{enumerate}
Persistence hurts the LR player if and only if the SR threshold lies in the ``danger zone'' $[\beta, 1-\alpha]$.
\end{remark}

\begin{remark}[Baseline Example]
For the baseline parameters $(\alpha=\BaseAlpha, \beta=\BaseBeta)$: the danger zone is $[\BaseBeta, 1-\BaseAlpha] = [0.5, 0.7]$. The SR threshold $\mu^* = \SRThreshold$ lies inside this interval, so the baseline deterrence example is \textbf{not} belief-robust. Changing SR payoffs to make $\mu^* = \BRThreshold < \beta = \BaseBeta$ would restore belief-robustness.
\end{remark}

% ================================================================
\section{Main Theorems}\label{sec:theorems}

We state two results. Theorem~1$'$ recovers the exact i.i.d.\ bound under belief-robustness. Theorem~1$''$ provides the general Markov bound.

\subsection{Definitions on the Expanded State Space}

All definitions from Luo \& Wolitzky (2024) carry over to $\tilde\Theta$, with strategies mapping $\tilde\Theta \to \Delta(A_1)$.

\begin{definition}[Confound-Defeating, Extended]\label{def:CD_extended}
A Markov strategy $s_1^* : \tilde\Theta \to \Delta(A_1)$ is \textbf{confound-defeating} if for every $(\alpha_0, \alpha_2) \in B_0(s_1^*)$, the joint distribution $\gamma(\alpha_0, s_1^*)$ is the \emph{unique solution} to:
\begin{equation}\label{eq:OT_extended}
    \OT\bigl(\tilde\rho(\alpha_0),\, \phi(\alpha_0, s_1^*);\, \alpha_2\bigr): \quad \max_{\gamma \in \Delta(\tilde\Theta \times A_1)} \int u_1(\tilde\theta, a_1, \alpha_2)\, d\gamma
\end{equation}
subject to $\pi_{\tilde\Theta}(\gamma) = \tilde\rho(\alpha_0)$ and $\pi_{A_1}(\gamma) = \phi(\alpha_0, s_1^*)$.
\end{definition}

\begin{definition}[Not Behaviorally Confounded, Extended]\label{def:NBC_extended}
$s_1^*$ is \textbf{not behaviorally confounded} if for any $\omega_{s_1'} \in \Omega$ with $s_1' \neq s_1^*$ and any $(\alpha_0, \alpha_2) \in B_1(s_1^*)$, we have $p(\alpha_0, s_1^*, \alpha_2) \neq p(\alpha_0, s_1', \alpha_2)$.
\end{definition}

\subsection{Theorem~1\texorpdfstring{$'$}{'} (Belief-Robust Extension)}

\begin{theorem}[Belief-Robust Markov Extension]\label{thm:belief_robust}
Let $\theta_t$ follow a stationary ergodic Markov chain on finite $\Theta$ (Assumption~\ref{ass:markov}). Let $\tilde\theta_t = (\theta_t, \theta_{t-1})$ with stationary distribution $\tilde\rho$. Suppose:
\begin{enumerate}[label=(\roman*)]
    \item $\omega_{s_1^*} \in \Omega$, where $s_1^* : \tilde\Theta \to \Delta(A_1)$ is a Markov strategy;
    \item $s_1^*$ is confound-defeating on $\tilde\Theta$ (Definition~\ref{def:CD_extended});
    \item $s_1^*$ is not behaviorally confounded (Definition~\ref{def:NBC_extended});
    \item The game is \textbf{belief-robust} with respect to $s_1^*$ and $(\Theta, F)$ (Definition~\ref{def:belief_robust}).
\end{enumerate}
Then:
\begin{equation}\label{eq:thm_br}
    \boxed{\liminf_{\delta \to 1}\, \underline{U}_1(\delta) \;\geq\; V(s_1^*)}
\end{equation}
where $V(s_1^*) = \inf_{(\alpha_0, \alpha_2) \in B(s_1^*)} u_1(\alpha_0, s_1^*, \alpha_2)$ is the commitment payoff, identical to the i.i.d.\ case.
\end{theorem}

\begin{remark}
Under belief-robustness, the SR belief gap is irrelevant: SR plays the same best response regardless of the filtering belief $F(\cdot|\theta)$. All the confirmed proof machinery---KL counting bound, OT robustness, monotonicity---applies without modification.
\end{remark}

\subsection{Theorem~1\texorpdfstring{$''$}{''} (General Corrected Bound)}

\begin{definition}[Markov Commitment Payoff]\label{def:V_markov}
The \textbf{Markov commitment payoff} is
\begin{equation}\label{eq:V_markov}
    V_{\mathrm{Markov}}(s_1^*) := \sum_{\theta' \in \Theta} \pi(\theta') \cdot \inf_{(\alpha_0, \alpha_2) \in B(s_1^*, F(\cdot|\theta'))} \sum_{\theta \in \Theta} F(\theta|\theta') \cdot u_1\bigl(\theta,\, s_1^*(\theta, \theta'),\, \alpha_2\bigr).
\end{equation}
Here $\theta'$ indexes the \emph{previous} state $\theta_{t-1}$, which is known to SR through the state-revealing strategy and determines the filtering belief $F(\cdot|\theta')$ about the current state $\theta_t$ (Remark~\ref{rem:timing}). The variable $\theta$ indexes the \emph{current} state $\theta_t$, which enters the payoff $u_1$. The formula averages over previous states using the stationary distribution $\pi$, applies the \textbf{state-contingent} Nash correspondence $B(s_1^*, F(\cdot|\theta'))$ determined by the previous state, and takes the conditional expectation of the payoff over the current state.
\end{definition}

\begin{theorem}[General Markov Extension]\label{thm:general}
Under conditions (i)--(iii) of Theorem~\ref{thm:belief_robust}, with ergodic Markov states and confound-defeating $s_1^*$ on $\tilde\Theta$:
\begin{equation}\label{eq:thm_general}
    \boxed{\liminf_{\delta \to 1}\, \underline{U}_1(\delta) \;\geq\; V_{\mathrm{Markov}}(s_1^*)}
\end{equation}
with $V_{\mathrm{Markov}}(s_1^*) = V(s_1^*)$ if and only if the game is belief-robust.
\end{theorem}

\begin{remark}[Necessity of Belief-Robustness]\label{rem:necessity}
The ``if and only if'' in Theorem~\ref{thm:general} requires comment. \emph{Sufficiency} is straightforward: belief-robustness forces the same Nash correspondence in every state, so the state-contingent infima coincide with the unconditional infimum. \emph{Necessity} holds under a genericity condition: when $u_1$ has strictly increasing differences in $(\theta, a_1)$ (as assumed in the supermodular case, Proposition~\ref{prop:supermodular}) and the SR threshold $\mu^*$ lies in the interior of the interval $[\min_\theta F(G|\theta), \max_\theta F(G|\theta)]$, the state-contingent best-response sets differ strictly across states, producing different infima. For non-generic parameters---where distinct best-response sets happen to yield the same infimum value---$V_{\mathrm{Markov}} = V(s_1^*)$ could hold without belief-robustness.
\end{remark}

\begin{remark}[Relationship Between Theorems]
The two results are nested: Theorem~\ref{thm:belief_robust} is the special case of Theorem~\ref{thm:general} where belief-robustness forces $V_{\mathrm{Markov}} = V(s_1^*)$. The difference $V(s_1^*) - V_{\mathrm{Markov}}$---the \emph{effect of persistence}---can be positive or negative. When the stationary belief $\pi$ induces favorable SR behavior (e.g., $\pi(G) > \mu^*$ so SR cooperates under i.i.d.), persistence can only cause some states to trigger defection, yielding $V_{\mathrm{Markov}} \leq V(s_1^*)$. Conversely, when $\pi$ induces unfavorable SR behavior (e.g., $\pi(G) < \mu^*$ so SR always defects under i.i.d.), persistence can enable state-contingent cooperation, yielding $V_{\mathrm{Markov}} > V(s_1^*)$---a new economic phenomenon absent from the i.i.d.\ framework.
\end{remark}

\begin{remark}[Continuity in Chain Parameters]
$V_{\mathrm{Markov}}(s_1^*)$ is a continuous function of the chain parameters $(\alpha, \beta)$. As $\alpha + \beta \to 1$ (the i.i.d.\ limit), $F(\cdot|\theta) \to \pi(\cdot)$ for all $\theta$, so $V_{\mathrm{Markov}} \to V(s_1^*)$. The difference $V(s_1^*) - V_{\mathrm{Markov}}$ vanishes continuously and may be positive or negative away from the i.i.d.\ line.
\end{remark}

\subsection{Extension to Behaviorally Confounded Strategies (Theorem 2)}\label{subsec:theorem2}

The salience-based extension (Luo \& Wolitzky, 2024, Appendix~A, Theorem~2) also generalizes to the Markov setting.

\begin{theorem}[Extended Theorem 2]\label{thm:theorem2}
Under the same Markov setup, if $s_1^*$ is confound-defeating on $\tilde\Theta$ and has salience $\beta_s$ (defined identically to Luo \& Wolitzky, 2024, but with confounding weights computed on $\tilde\Theta$), then:
\begin{equation}
    \liminf_{\delta \to 1}\, \underline{U}_1(\delta) \;\geq\; \beta_s\, V_{\mathrm{Markov}}(s_1^*) + (1 - \beta_s)\, V_0(s_1^*).
\end{equation}
Under belief-robustness, $V_{\mathrm{Markov}}$ is replaced by $V(s_1^*)$. If $s_1^*$ is not behaviorally confounded, $\beta_s = 1$ and this reduces to Theorems~\ref{thm:belief_robust} or~\ref{thm:general} respectively.
\end{theorem}

\begin{proof}[Proof sketch]
The proof of Theorem~2 in Luo \& Wolitzky (2024) follows from Theorem~1 via Lemma~7 (the salience bound). Lemma~7 uses the submartingale property of $\mu_t(\omega_{s_1^*} | \Omega_\eta(s_1^*) \setminus \{\omega^R\}, h_t)$, which holds by Bayesian updating regardless of the signal process. The remainder of the argument---compactness, limiting, the three-case analysis---extends as in Section~\ref{sec:proof}. The only modification is the payoff bound: under belief-robustness, the full $V(s_1^*)$ is used; in the general case, $V_{\mathrm{Markov}}(s_1^*)$ replaces $V(s_1^*)$.
\end{proof}

% ================================================================
\section{Proof Sketch}\label{sec:proof}

The proof follows the five-step structure of Luo \& Wolitzky (2024, Section~4.2). At each step, we identify whether the i.i.d.\ assumption was used and, if so, how belief-robustness or the corrected bound handles the modification.

\subsection{Overview: Where i.i.d.\ Is Actually Used}

\begin{table}[h]
\centering
\begin{tabular}{@{}lcc@{}}
\toprule
\textbf{Proof Step} & \textbf{i.i.d.\ used?} & \textbf{Fix needed} \\
\midrule
Step 0: OT / confound-defeating & No & Replace $Y_0$ with $\tilde\Theta$ \\
\textbf{Step 1: Lemma 1 (equilibrium)} & \textbf{Yes} & \textbf{SR belief issue} \\
Step 2: Lemma 2 (KL bound) & \textbf{No} & None \\
Step 3: Lemma 3 (martingale) & Partially & Ergodicity + filter stability \\
Step 4: Lemma 4 (combining) & No & Uses corrected BR \\
\textbf{Step 5: Payoff bound} & \textbf{Yes} & \textbf{Belief-robust or $V_{\mathrm{Markov}}$} \\
\bottomrule
\end{tabular}
\caption{Where the i.i.d.\ assumption enters the proof. Bold rows indicate where the Luo--Wolitzky argument fails under Markov states and correction is needed.}
\label{tab:iid_usage}
\end{table}

The table reveals a noteworthy pattern: the purely information-theoretic steps (the KL bound and the martingale convergence) require no modification or only mild conditions, while the game-theoretic steps (the equilibrium implications and the payoff bound) are where the i.i.d.\ assumption does essential work. This reflects the distinction between the \emph{mathematical tools}, which are process-independent, and their \emph{semantic interpretation} within the reputation game, which depends on the information structure.


\subsection{Step 0: OT / Confound-Defeating Extension}\label{subsec:step0}

The state space is $\tilde\Theta = \Theta \times \Theta$ instead of $Y_0$, but the entire optimal transport framework carries over without change. The OT problem $\OT(\tilde\rho, \phi; \alpha_2)$ on $\tilde\Theta \times A_1$ is a finite-dimensional linear program, structurally identical to the Luo--Wolitzky formulation on $Y_0 \times A_1$.

\begin{proposition}[Extension of Proposition 5]\label{prop:CM_extended}
A joint distribution $\gamma \in \Delta(\tilde\Theta \times A_1)$ with marginals $\tilde\rho$ and $\phi$ uniquely solves $\OT(\tilde\rho, \phi; \alpha_2)$ if and only if $\supp(\gamma) \subset \tilde\Theta \times A_1$ is \textbf{strictly $u_1(\cdot, \alpha_2)$-cyclically monotone}.
\end{proposition}

\begin{proof}
This is Proposition~5 of Luo--Wolitzky applied to $X = \tilde\Theta$ and $Y = A_1$. The proof (Luo \& Wolitzky, 2024, Appendix~C) is a purely combinatorial argument about finite optimal transport problems and does not depend on the time-series structure of the data. The argument uses only: (a) finiteness of $\tilde\Theta \times A_1$ (which holds since $\Theta$ is finite), and (b) the characterization of OT solutions via cyclical monotonicity (Rochet 1987; Santambrogio 2015). Both hold on the expanded state space.
\end{proof}

\begin{corollary}[Extension of Corollary 1]\label{cor:CD_CM}
$s_1^*$ is confound-defeating if and only if $\supp(s_1^*) \subset \tilde\Theta \times A_1$ is strictly $u_1$-cyclically monotone (when $u_1$ is cyclically separable) or strictly $u_1(\cdot, \alpha_2)$-cyclically monotone for all $(\alpha_0, \alpha_2) \in B_0(s_1^*)$ (in general).
\end{corollary}

Computational evidence confirms this robustness: the OT support stability margin exceeds $\OTStabilityMargin$ in \OTStabilityPct\% of the $(\alpha,\beta)$ parameter space (Figure~\ref{fig:ot_robustness}), demonstrating that the confound-defeating property is preserved under the belief perturbations that arise from Markov dynamics.


\subsection{Step 1: Lemma 1 --- Equilibrium Implications}\label{subsec:step1}

\begin{lemma}[Extension of Lemma 1]\label{lem:equil}
Fix a Nash equilibrium $(\sigma_0^*, \sigma_1^*, \sigma_2^*)$. For any $\varepsilon > 0$, there exists $\eta > 0$ such that if:
\begin{enumerate}[label=(\arabic*)]
    \item $\|p(\sigma_0^*, s_1^*, \sigma_2^* | h_t) - p(\sigma_0^*, \sigma_1^*, \sigma_2^* | h_t)\| \leq \eta$, and
    \item $\|p(\sigma_0^*, \sigma_1^*(\omega^R), \sigma_2^* | h_t) - p(\sigma_0^*, s_1^*, \sigma_2^* | h_t)\| \leq \eta$,
\end{enumerate}
then $\|\sigma_1^*(h_t, \omega^R) - s_1^*\| \leq \varepsilon$.
\end{lemma}

\begin{proof}
The argument is a per-period one-shot deviation analysis that uses confound-defeatingness and the equilibrium condition. Suppose $\|\sigma_1^*(h_t, \omega^R) - s_1^*\| > \varepsilon$. Condition~(1) and the Nash equilibrium condition imply $(\sigma_0^*(h_t), \sigma_2^*(h_t)) \in B_\eta(s_1^*)$, as $\sigma_1^*(h_t)$ $\eta$-confirms it against $s_1^*$. Then condition~(2), combined with $\|\sigma_1^*(h_t, \omega^R) - s_1^*\| > \varepsilon$ and the confound-defeating property, implies there exists $\tilde{s}_1$ such that:
\[
    p(\sigma_0^*, \tilde{s}_1, \sigma_2^* | h_t) = p(\sigma_0^*, \sigma_1^*(\omega^R), \sigma_2^* | h_t) \quad \text{and} \quad u_1(\sigma_0^*, \tilde{s}_1, \sigma_2^* | h_t) > u_1(\sigma_0^*, \sigma_1^*(\omega^R), \sigma_2^* | h_t).
\]
Deviating from $\sigma_1^*(h_t, \omega^R)$ to $\tilde{s}_1$ is then a profitable one-shot deviation that is signal-preserving, contradicting the equilibrium assumption. The strategy space is now Markov strategies on $\tilde\Theta$ instead of static strategies on $Y_0$, but the one-shot deviation argument is identical.
\end{proof}

\textbf{Where i.i.d.\ matters for Step~1.} This is the first point where the i.i.d.\ assumption enters the Luo--Wolitzky proof substantively. In the i.i.d.\ case, the one-shot deviation objective takes the form $u_1(\theta, a_1, \alpha_2) + \delta V_{\mathrm{cont}}^{a_1}$, where the continuation value $V_{\mathrm{cont}}^{a_1}$ depends only on $a_1$ because future states are independent of the current state $\theta$. Adding a function of $a_1$ alone to the objective does not change the optimal transport solution, so confound-defeatingness with respect to $u_1$ suffices.

In the Markov case, the continuation value $V_{\mathrm{cont}}(\theta_t, a_1, h_t)$ depends on $\theta_t$ through the transition kernel $F$. The effective one-shot deviation objective becomes $w(\tilde\theta, a_1) = u_1(\tilde\theta, a_1, \alpha_2) + \delta g(\theta_t, a_1, h_t)$ for some history-dependent function $g$, and adding this $\theta_t$-dependent term can in principle change the OT solution.

\begin{remark}[Continuation Value Subtlety]\label{rem:continuation}
\textbf{Resolution for the belief-robust case.} Under strict supermodularity of $u_1$ in $(\tilde\theta, a_1)$, the co-monotone coupling is optimal for all objectives of the form $u_1 + g$ provided $g$ preserves the supermodular structure. Since $g(\theta_t, a_1, h_t)$ is supermodular in $(\theta_t, a_1)$ whenever $V_{\mathrm{cont}}$ is increasing in $\theta_t$ for each $a_1$ (which holds when higher states have higher continuation values), the OT solution is unchanged. All the paper's applications (deterrence, trust, signaling) are supermodular, and under belief-robustness the SR behavior is constant across states, so the continuation value perturbation is absorbed by the supermodular structure and the OT solution remains unchanged.

\textbf{Resolution for the general case.} Two approaches are available. First, one may strengthen the confound-defeating condition: require $s_1^*$ to be confound-defeating for all objectives of the form $u_1 + g$ where $g : \tilde\Theta \times A_1 \to \R$ is bounded. This is stronger than the Luo--Wolitzky condition but closes the gap. Second, a continuity argument is available: by filter stability (Proposition~\ref{prop:filter_stability}), the filtering distribution $\pi_t(h_t)$ converges to the stationary distribution $\tilde\rho$ exponentially fast. Since confound-defeating is an open condition (unique OT solution is robust to small perturbations of the marginals), confound-defeating at $\tilde\rho$ implies approximate confound-defeating at $\pi_t(h_t)$ for large $t$. Combined with $\delta \to 1$ (which makes the continuation value perturbation small relative to the stage-game payoff), this yields the result. The general case is handled by Theorem~\ref{thm:general}, which replaces the static Nash correspondence $B(s_1^*, \pi)$ with the state-contingent correspondence $B(s_1^*, F(\cdot|\theta_t))$.
\end{remark}


\subsection{Step 2: Lemma 2 --- KL Counting Bound}\label{subsec:step2}

This is the key technical step where one might expect the i.i.d.\ assumption to be essential. It is not.

\begin{lemma}[Extension of Lemma 2]\label{lem:KL}
For any $\eta > 0$ and any Nash equilibrium $(\sigma_0^*, \sigma_1^*, \sigma_2^*)$, the expected number of periods $t$ where $h_t \notin H_t^\eta$ is bounded by:
\begin{equation}\label{eq:KL_bound}
    \E_Q\!\left[\#\{t : h_t \notin H_t^\eta\}\right] \;\leq\; \bar{T}(\eta, \mu_0) \;:=\; \frac{-2\log\mu_0(\omega_{s_1^*})}{\eta^2}.
\end{equation}
\textbf{The bound is identical to the i.i.d.\ case.}
\end{lemma}

\begin{proof}
The argument uses three ingredients, \emph{none of which require i.i.d.}

\medskip
\noindent\textbf{(a) Chain rule for KL divergence.}
For any joint distribution over $(y_0, y_1, \ldots, y_{T-1})$:
\begin{equation}\label{eq:KL_chain}
    \KL(P^T \| Q^T) \;=\; \sum_{t=0}^{T-1} \E_P\!\left[\KL\!\left(P_{y_t | h_{t-1}} \;\big\|\; Q_{y_t | h_{t-1}}\right)\right].
\end{equation}
This is a general property of KL divergence that holds for \emph{arbitrary} joint distributions, including those generated by Markov chains. It is a consequence of the chain rule for KL divergence (Cover \& Thomas, 2006, Theorem~2.5.3), which states:
\[
    \KL(P(X_1, \ldots, X_n) \| Q(X_1, \ldots, X_n)) = \sum_{i=1}^{n} \E_P\!\left[\KL(P(X_i | X_1, \ldots, X_{i-1}) \| Q(X_i | X_1, \ldots, X_{i-1}))\right].
\]
No independence across periods is assumed. A self-contained verification is provided in Appendix~\ref{app:kl}.

\medskip
\noindent\textbf{(b) Total KL bound from Bayesian updating.}
The Bayesian updating identity gives:
\begin{equation}\label{eq:total_KL}
    \sum_{t=0}^{T-1} \E_Q\!\left[\KL(p_t \| q_t)\right] \;\leq\; -\log\mu_0(\omega_{s_1^*})
\end{equation}
where $p_t = p(\sigma_0^*, s_1^*, \sigma_2^* | h_t)$ and $q_t = p(\sigma_0^*, \sigma_1^*, \sigma_2^* | h_t)$.
This follows from $\mu_T(\omega_{s_1^*}) \leq 1$ and the telescoping identity:
\[
    \log\frac{\mu_T(\omega_{s_1^*})}{\mu_0(\omega_{s_1^*})} = \sum_{t=0}^{T-1} \log\frac{p_t(y_{1,t})}{q_t(y_{1,t})} = \sum_{t=0}^{T-1} \log\frac{p(\sigma_0^*, s_1^*, \sigma_2^* | h_t)[y_{1,t}]}{p(\sigma_0^*, \sigma_1^*, \sigma_2^* | h_t)[y_{1,t}]}.
\]
Taking expectations under $Q$ and using $\E_Q[\log(p_t/q_t)] = \KL(p_t \| q_t)$ gives~\eqref{eq:total_KL}. This is a consequence of Bayes' rule alone. \textbf{No independence across periods is used.}

\medskip
\noindent\textbf{(c) Pinsker's inequality (per-period).}
For each period $t$:
\begin{equation}\label{eq:pinsker}
    \|p_t - q_t\|^2 \;\leq\; 2\,\KL(p_t \| q_t).
\end{equation}
This is a per-period inequality requiring no temporal structure.

\medskip
\noindent\textbf{Combining:}
In each ``distinguishing period'' where $\|p_t - q_t\| > \eta$, Pinsker gives $\KL(p_t \| q_t) \geq \eta^2/2$. Summing:
\[
    \frac{\eta^2}{2} \cdot \#\{\text{distinguishing periods}\} \;\leq\; \sum_t \KL(p_t \| q_t) \;\leq\; -\log\mu_0(\omega_{s_1^*}).
\]
Hence $\#\{\text{distinguishing periods}\} \leq -2\log\mu_0(\omega_{s_1^*})/\eta^2 = \bar{T}(\eta, \mu_0)$.
\end{proof}

\begin{remark}
This is the key surprise of the extension. The initial conjecture was that a mixing-time correction factor $\tau_{\mathrm{mix}}$ would be needed. It is not: the KL chain rule and Bayesian updating identity hold for general stochastic processes. Monte Carlo verification ($N=\KLMonteCarloN$ simulations, $T=\KLMonteCarloPeriods$ periods) confirms that the empirical distribution of distinguishing-period counts is nearly identical for Markov and i.i.d.\ processes (Figure~\ref{fig:kl_bound}).
\end{remark}


\subsection{Step 3: Lemma 3 --- Martingale Convergence}\label{subsec:step3}

\begin{lemma}[Extension of Lemma 3]\label{lem:martingale}
For all $\zeta > 0$, there exists a set of infinite histories $G(\zeta) \subset H^\infty$ satisfying $Q(G(\zeta)) > 1 - \zeta$ and a period $\hat{T}(\zeta)$ (independent of $\delta$ and the choice of equilibrium) such that, for any $h \in G(\zeta)$ and any $t \geq \hat{T}(\zeta)$:
\[
    \mu_t(\cdot | h) \in M_\zeta \;:=\; \bigl\{\mu \in \Delta(\Omega) : \mu(\{\omega^R, \omega_{s_1^*}\}) \geq 1 - \zeta\bigr\}.
\]
\end{lemma}

\begin{proof}[Proof sketch]
The proof has two parts.

\medskip
\noindent\textbf{Part A: Per-equilibrium convergence (Extension of Lemma 9).}

The posterior $\mu_t(\omega | h)$ over $\Omega$ is a bounded martingale under $Q$ (the measure induced by commitment type $\omega_{s_1^*}$). This is a consequence of Bayesian updating and holds regardless of the signal structure. By the \textbf{martingale convergence theorem}, $\mu_t(\omega | h) \to \mu_\infty(\omega | h)$ $Q$-a.s.\ for each $\omega$.

We need to show $\mu_\infty(\{\omega^R, \omega_{s_1^*}\} | h) = 1$ $Q$-a.s.\ The critical step is that for any $\omega_{s_1}$ with $\mu_\infty(\omega_{s_1} | h) > 0$, the signal distributions under $s_1$ and $s_1^*$ must agree asymptotically. In the i.i.d.\ case, this follows immediately from the KL bound. In the Markov case, we proceed as follows. First, the per-period signal distribution under commitment type $\omega_{s_1}$ depends on the \emph{filtering distribution} $\pi(\theta_t | h_t, s_1)$---the posterior over the current state given public signals. Second, for an \textbf{ergodic} Markov chain, the filtering distribution satisfies \emph{filter stability} (also known as filter forgetting): regardless of the initial condition, the posterior $\pi(\theta_t | h_t, s_1)$ eventually concentrates on values determined by the observation process, and the effect of the initial condition decays exponentially. This is a classical result for HMMs on finite state spaces (Chigansky \& Liptser, 2004; Del Moral, 2004). Third, the KL bound from Lemma~\ref{lem:KL} (which holds unchanged) implies:
\begin{equation}\label{eq:signal_convergence}
    \lim_{t \to \infty} \left\|p_{Y_1}(\sigma_0^*, s_1 | h_t) - p_{Y_1}(\sigma_0^*, \tilde{s}_1 | h_t, \Omega \setminus \{\omega^R\})\right\| = 0
\end{equation}
$Q$-a.s., exactly as in the Luo--Wolitzky proof of Lemma~9 (their Appendix~B.2). The KL chain rule argument that yields this convergence is valid for arbitrary signal processes. Since $s_1^*$ is not behaviorally confounded, any type with the same asymptotic signal distribution must be $s_1^*$ itself, hence $\mu_\infty(\{\omega^R, \omega_{s_1^*}\} | h) = 1$.

Computational evidence across a $\FilterGridSize \times \FilterGridSize$ parameter grid confirms that the fitted forgetting rate $\lambda$ correlates with the chain's second eigenvalue $|1-\alpha-\beta|$ at $r > \FilterCorrelation$, with exponential decay fits achieving $R^2 > 0.99$ throughout (Figure~\ref{fig:filter_stability}).

\medskip
\noindent\textbf{Part B: Uniformity over equilibria.}

The uniformity argument ($\hat{T}$ independent of $\delta$ and the equilibrium) uses \textbf{compactness} of $B_1(s_1^*)^{H^\infty}$ under the sup-norm topology, \textbf{Egorov's theorem} (a general measure-theoretic result), and \textbf{continuity} of finite-dimensional distributions $Q^T$ as strategies vary. With Markov states, the space of Markov strategies $s_1 : \tilde\Theta \to \Delta(A_1)$ is compact ($\tilde\Theta$ is finite, $\Delta(A_1)$ is compact). The compactness of $B_1(s_1^*)^{H^\infty}$ follows by the same product topology argument. Egorov's theorem is a general result requiring only a finite measure space. The continuity of $Q^T$ in strategies uses finiteness and continuity of the signal structure, which holds with Markov states.

The proof of uniformity then follows the original argument in Appendix~B.2 of Luo--Wolitzky: suppose for contradiction that $\hat{T}$ cannot be chosen uniformly; extract a convergent subsequence using compactness; apply Egorov's theorem to obtain a contradiction with $Q$-a.s.\ convergence from Part~A.
\end{proof}


\subsection{Step 4: Lemma 4 --- Combining the Pieces}\label{subsec:step4}

\begin{lemma}[Extension of Lemma 4]\label{lem:combining}
There exist strictly positive functions $\zeta(\eta)$ and $\xi(\eta)$, satisfying $\lim_{\eta \to 0} \zeta(\eta) = \lim_{\eta \to 0} \xi(\eta) = 0$, such that if $h_t \in H_t^\eta$ and $\mu_t(\cdot | h_t) \in M_{\zeta(\eta)}$, then:
\[
    (\sigma_0^*(h_t),\, \sigma_2^*(h_t)) \;\in\; \hat{B}_{\xi(\eta)}(s_1^*).
\]
\end{lemma}

\begin{proof}
This is a per-period argument combining Lemma~\ref{lem:equil} with the definition of $M_\zeta$ and the confirmed best response structure. It uses only the stage-game structure and the proximity of the posterior to $\{\omega^R, \omega_{s_1^*}\}$. If $h_t \in H_t^\eta$, then $(\sigma_0^*(h_t), \sigma_2^*(h_t)) \in B_\eta(s_1^*)$; and if additionally $\mu_t(\cdot | h_t) \in M_{\zeta(\eta)}$, then the posterior concentrates on $\{\omega^R, \omega_{s_1^*}\}$, from which it follows (via Lemma~\ref{lem:equil} and continuity) that $(\sigma_0^*(h_t), \sigma_2^*(h_t)) \in \hat{B}_{\xi(\eta)}(s_1^*)$ for appropriate $\xi(\eta)$. No independence across periods is used: the argument is identical to that of Luo \& Wolitzky (2024).
\end{proof}


\subsection{Step 5: The Payoff Bound}\label{subsec:step5}

This is the second place where the i.i.d.\ assumption enters the Luo--Wolitzky proof substantively, and where the two theorems diverge.

\begin{proof}[Proof of Theorems~\ref{thm:belief_robust} and~\ref{thm:general}]
Fix $\varepsilon > 0$. Choose $\eta$ small enough so that:
\[
    \inf_{(\alpha_0, \alpha_2) \in \hat{B}_{\xi(\eta)}(s_1^*)} u_1(\alpha_0, s_1^*, \alpha_2) \;\geq\; V(s_1^*) - \frac{\varepsilon}{3}.
\]

On the $(1 - \zeta(\eta))$-probability event $G(\zeta(\eta))$, for $t \geq \hat{T}(\zeta(\eta))$:
\begin{enumerate}[label=(\roman*)]
    \item The expected number of periods where $h_t \notin H_t^\eta$ is at most $\bar{T}(\eta, \mu_0)$ (Lemma~\ref{lem:KL}).
    \item $\mu_t(\cdot | h_t) \in M_{\zeta(\eta)}$ (Lemma~\ref{lem:martingale}).
    \item In ``good'' periods (where both conditions hold), $(\sigma_0^*(h_t), \sigma_2^*(h_t)) \in \hat{B}_{\xi(\eta)}(s_1^*)$ (Lemma~\ref{lem:combining}).
\end{enumerate}

Front-loading the bad periods and using the discount factor:
\begin{equation}\label{eq:payoff_bound}
    U_1(\delta) \;\geq\; (1 - \delta^{\bar{T} + \hat{T}}) \cdot \underline{u}_1 \;+\; \delta^{\bar{T} + \hat{T}} \cdot \left(V(s_1^*) - \frac{\varepsilon}{3}\right).
\end{equation}

As $\delta \to 1$:
\begin{equation}\label{eq:main_result_limit}
    \liminf_{\delta \to 1}\, \underline{U}_1(\delta) \;\geq\; V(s_1^*) - \frac{\varepsilon}{3}.
\end{equation}

Taking $\varepsilon \to 0$ gives $\liminf_{\delta \to 1}\, \underline{U}_1(\delta) \geq V(s_1^*)$.

\textbf{Belief-robust case (Theorem~\ref{thm:belief_robust}).} Under belief-robustness, the SR best response is constant across states, so the LR player receives at least $\inf_{B(s_1^*)} u_1 = V(s_1^*)$ in every good period. The argument above applies verbatim and yields $\liminf_{\delta\to 1} \underline{U}_1(\delta) \geq V(s_1^*)$.

\textbf{General case (Theorem~\ref{thm:general}).} Without belief-robustness, the SR player's best response in state $\theta_t$ is drawn from the state-contingent correspondence $B(s_1^*, F(\cdot|\theta_t))$. The LR player receives at least $\inf_{B(s_1^*, F(\cdot|\theta_t))} u_1$ in state $\theta_t$ during good periods. Averaging over the ergodic distribution of states and applying the same front-loading argument gives $\liminf_{\delta\to 1} \underline{U}_1(\delta) \geq V_{\mathrm{Markov}}(s_1^*)$.
\end{proof}

\begin{remark}[Role of Mixing Time]
The mixing time $\tau_{\mathrm{mix}}$ does \emph{not} enter either payoff bound. It affects only the \textbf{rate of convergence}---specifically, the constant $\hat{T}(\zeta)$ in Lemma~\ref{lem:martingale}, which may be larger for slowly mixing chains. The limit as $\delta \to 1$ is unaffected.
\end{remark}

% ================================================================
\section{The Supermodular Case}\label{sec:supermodular}

\subsection{Monotonicity on the Lifted Space}

\begin{proposition}[Extension of Proposition 7]\label{prop:supermodular}
Suppose $u_1$ is \textbf{strictly supermodular} in $(\tilde\theta, a_1)$ for some orders $\succeq_{\tilde\Theta}$ on $\tilde\Theta$ and $\succeq_{A_1}$ on $A_1$, for all $\alpha_2$. Then the following are equivalent:
\begin{enumerate}[label=(\arabic*)]
    \item $s_1^*$ is confound-defeating.
    \item $s_1^*$ is monotone: if $\tilde\theta \succ \tilde\theta'$, $a_1 \in \supp(s_1^*(\tilde\theta))$, $a_1' \in \supp(s_1^*(\tilde\theta'))$, then $a_1 \succeq a_1'$.
    \item For any $(\alpha_0, \alpha_2)$, $\gamma(\alpha_0, s_1^*)$ is the \textbf{co-monotone coupling} of $\tilde\rho(\alpha_0)$ and $\phi(\alpha_0, s_1^*)$.
\end{enumerate}
\end{proposition}

\begin{proof}
The equivalence $(1) \Leftrightarrow (3)$ follows from Lemma~6 of Luo \& Wolitzky (2024) applied to $\tilde\Theta \times A_1$: under strict supermodularity, the co-monotone coupling is the unique solution to the OT problem (Santambrogio, 2015, Lemma~2.8). The equivalence $(2) \Leftrightarrow (3)$ follows from the definition of monotonicity and co-monotone coupling. The proof is a purely combinatorial argument on the expanded state space and does not reference the temporal structure of the signal process.
\end{proof}

\subsection{Payoffs Depending Only on $\theta_t$}

If $u_1(\tilde\theta, a_1, \alpha_2) = u_1(\theta_t, a_1, \alpha_2)$, then $u_1$ is supermodular in $(\tilde\theta, a_1)$ if and only if it is supermodular in $(\theta_t, a_1)$, using any order on $\tilde\Theta$ that is consistent with the order on the first coordinate (e.g., the lexicographic order). The relevant order on $\tilde\Theta$ is the \emph{first-coordinate order}: $(\theta_t, \theta_{t-1}) \succeq (\theta_t', \theta_{t-1}')$ if and only if $\theta_t \succeq \theta_t'$. Under this order, the supermodularity condition is \textbf{unchanged} from the i.i.d.\ case: it depends only on the payoff structure in $(\theta_t, a_1)$, not on the Markov dynamics.

Note that under the first-coordinate order, states differing only in $\theta_{t-1}$ are incomparable (not strictly ordered), so the strict increasing-differences condition imposes no constraint between such states. Strict supermodularity in $(\tilde\theta, a_1)$ therefore reduces to strict supermodularity in $(\theta_t, a_1)$ for payoffs depending only on $\theta_t$.

Computational evidence confirms this: for $\theta_t$-dependent payoffs, \SupermodFraction{} out of \SupermodTotal{} orderings of the lifted space $\tilde\Theta$ preserve supermodularity---exactly those consistent with the first-coordinate ranking (Figure~\ref{fig:monotonicity}).

\begin{figure}[ht]
\centering
\includegraphics[width=0.8\textwidth]{figures/fig_monotonicity.png}
\caption{Supermodularity fraction by payoff type on the lifted space. For $\theta_t$-only payoffs, \SupermodFraction/\SupermodTotal{} orderings preserve supermodularity. For transition-dependent payoffs, the fraction drops dramatically.}
\label{fig:monotonicity}
\end{figure}

\subsection{Transition-Dependent Payoffs}

When payoffs depend on the full lifted state $(\theta_t, \theta_{t-1})$---e.g., escalation penalties that depend on whether the state deteriorated---the ordering problem becomes harder. Only a small fraction of orderings on $\tilde\Theta$ preserve supermodularity in general. This is a genuine limitation of the Markov extension for non-standard payoff structures.

\subsection{Extended Bounds}

\begin{corollary}[Extended Lower Bound]\label{cor:lower}
Under the conditions of Proposition~\ref{prop:supermodular} with $\theta_t$-only payoffs:
\begin{equation}
    \liminf_{\delta \to 1}\, \underline{U}_1(\delta) \;\geq\; v_{\mathrm{mon}} \;:=\; \sup\left\{V_{\mathrm{Markov}}(s_1) : s_1 \text{ monotone on } \tilde\Theta,\; \omega_{s_1} \in \Omega\right\}.
\end{equation}
Under belief-robustness, $V_{\mathrm{Markov}}(s_1)$ can be replaced by $V(s_1)$.
\end{corollary}

\begin{corollary}[Extended Upper Bound]\label{cor:upper}
If $u_1$ is cyclically separable and $\mu_0(\omega^R) \to 1$, then:
\begin{equation}
    \bar{U}_1(\delta) \;<\; \bar{v}_1^{CM} + \varepsilon
\end{equation}
where $\bar{v}_1^{CM}$ is the supremum over $u_1$-cyclically monotone strategies on $\tilde\Theta$.
\end{corollary}

\begin{proof}
The upper bound follows from the extension of Lemma~5: in any equilibrium, $\sigma_1^*(h_t, \omega^R)$ must solve $\OT(\sigma_0^*(h_t), \phi(\sigma_0^*(h_t), \sigma_1^*(h_t, \omega^R)), \sigma_2^*(h_t))$, hence is $u_1$-cyclically monotone. This is a per-period optimality condition and does not use the i.i.d.\ assumption.
\end{proof}

% ================================================================
\section{Worked Example: Deterrence Game with Markov Attacks}\label{sec:example}

We illustrate both theorems using the deterrence game with Markov attacks. We present the full game setup, a formal proposition establishing when the extended theorem applies, concrete numerical calculations, and both a belief-robust and a non-belief-robust version.

\subsection{Setup}

The state $\theta_t \in \{G(\text{ood}), B(\text{ad})\}$ follows a Markov chain:
\begin{align}
    \Prob(G | G) &= 1 - \alpha, \qquad \Prob(B | G) = \alpha, \\
    \Prob(G | B) &= \beta, \qquad \Prob(B | B) = 1 - \beta,
\end{align}
with $\alpha, \beta \in (0,1)$. The unique stationary distribution is:
\begin{equation}
    \pi(G) = \frac{\beta}{\alpha + \beta}, \qquad \pi(B) = \frac{\alpha}{\alpha + \beta}.
\end{equation}

The long-run player chooses $a_1 \in \{A(\text{cquiesce}), F(\text{ight})\}$. The short-run player, observing the history of $a_1$ but not $\theta$, chooses $a_2 \in \{C(\text{ooperate}), D(\text{efect})\}$. Payoffs conditional on $a_2 = D$ (or more generally against SR strategy $\alpha_2$) are:
\begin{equation}
    u_1(G, A) = 1, \quad u_1(G, F) = x, \quad u_1(B, A) = y, \quad u_1(B, F) = 0,
\end{equation}
with $x, y \in (0,1)$. (See Luo \& Wolitzky, Section~2.1, for the full payoff matrix with $(g, l)$ parameters.)

The Stackelberg strategy is $s_1^*(G) = A$, $s_1^*(B) = F$ (ignoring $\theta_{t-1}$): the long-run player acquiesces in good states and fights in bad states.

\subsection{Lifted State Distribution}

The lifted state is $\tilde\theta_t = (\theta_t, \theta_{t-1}) \in \{(G,G),\, (G,B),\, (B,G),\, (B,B)\}$, with stationary distribution:

\begin{center}
\begin{tabular}{@{}cc@{}}
\toprule
$\tilde\theta$ & $\tilde\rho(\tilde\theta)$ \\
\midrule
$(G,G)$ & $\beta(1-\alpha)/(\alpha+\beta)$ \\
$(G,B)$ & $\alpha\beta/(\alpha+\beta)$ \\
$(B,G)$ & $\alpha\beta/(\alpha+\beta)$ \\
$(B,B)$ & $\alpha(1-\beta)/(\alpha+\beta)$ \\
\bottomrule
\end{tabular}
\end{center}

\subsection{Markov Deterrence Proposition}

\begin{proposition}[Markov Deterrence]\label{prop:deterrence}
Consider the deterrence game with Markov attacks.
\begin{enumerate}[label=(\arabic*)]
    \item \textbf{If $x + y < 1$ (supermodular):} Under the belief-robust condition (Proposition~\ref{prop:br_condition}), a patient long-run player secures at least $V(s_1^*) = \beta/(\alpha + \beta)$ in any Nash equilibrium, for any $\mu_0 > 0$. In the general (non-belief-robust) case, the bound is $V_{\mathrm{Markov}}(s_1^*) \leq V(s_1^*)$.
    \item \textbf{If $x + y > 1$ (submodular):} As $\mu_0 \to 0$, the long-run player's payoff approaches the minmax payoff.
\end{enumerate}
\end{proposition}

\begin{proof}
Since $u_1$ depends only on $\theta_t$ and $x + y < 1$ gives strict supermodularity in $(\theta_t, a_1)$ (with orders $G \succ B$ and $A \succ F$), the supermodularity condition on $\tilde\Theta \times A_1$ is satisfied (Section~\ref{sec:supermodular}).

The strategy $s_1^*(G) = A$, $s_1^*(B) = F$ is monotone ($G \succ B \implies A \succ F$). By Proposition~\ref{prop:supermodular}, $s_1^*$ is confound-defeating. If $s_1^*$ is not behaviorally confounded (which holds generically; see Definition~\ref{def:NBC_extended}), then the theorems apply. Under belief-robustness (Theorem~\ref{thm:belief_robust}):
\[
    \liminf_{\delta \to 1}\, \underline{U}_1(\delta) \;\geq\; V(s_1^*) \;=\; \frac{\beta}{\alpha + \beta}.
\]
In the general case (Theorem~\ref{thm:general}), the bound is $V_{\mathrm{Markov}}(s_1^*)$, which equals $V(s_1^*)$ if and only if the game is belief-robust.

For part (2), when $x + y > 1$, the payoff is strictly submodular. By the extended upper bound (Corollary~\ref{cor:upper}), the only cyclically monotone strategies are \emph{anti-monotone} (higher state $\to$ lower action), which gives the long-run player at most her minmax payoff.
\end{proof}

\subsection{Version 1: Belief-Robust ($\mu^* = \BRThreshold$)}

With SR payoffs calibrated so the indifference threshold is $\mu^* = \BRThreshold < \beta = \BaseBeta$, the SR player always cooperates regardless of the revealed state, since $\mu^* = \BRThreshold < \beta = \BaseBeta \leq F(G|\theta)$ for all $\theta$. The game is \textbf{belief-robust} (Proposition~\ref{prop:br_condition}), and by Theorem~\ref{thm:belief_robust}:
\[
    \liminf_{\delta\to 1} \underline{U}_1(\delta) \geq V(s_1^*) = \BRPayoff.
\]
The bound is exact and identical to the i.i.d.\ case.

\subsection{Version 2: Non-Belief-Robust ($\mu^* = \SRThreshold$)}

With SR payoffs giving threshold $\mu^* = \SRThreshold \in [\BaseBeta, 1-\BaseAlpha] = [0.5, 0.7]$, the SR best response depends on the revealed state:

\begin{table}[ht]
\centering
\renewcommand{\arraystretch}{1.3}
\begin{tabular}{@{}ccccc@{}}
\toprule
\textbf{State $\theta$} & \textbf{$\pi(\theta)$} & \textbf{SR Belief $F(G|\theta)$} & \textbf{SR Action} & \textbf{LR Payoff} \\
\midrule
$G$ & \BasePiG & $\SRBeliefAfterG > \SRThreshold$ & Cooperate & $u_1(G, A, C)$ \\
$B$ & \BasePiB & $\SRBeliefAfterB < \SRThreshold$ & Defect & $u_1(B, F, D)$ \\
\bottomrule
\end{tabular}
\caption{State-contingent SR behavior in the non-belief-robust deterrence game. SR cooperates in good states (where $F(G|G) = \SRBeliefAfterG > \mu^* = \SRThreshold$) but defects in bad states (where $F(G|B) = \SRBeliefAfterB < \mu^*$).}
\label{tab:sr_behavior}
\end{table}

By Theorem~\ref{thm:general}, the corrected bound is:
\[
    V_{\mathrm{Markov}} = \pi(G) \cdot u_1(G, A, C) + \pi(B) \cdot u_1(B, F, D) = \PayoffFiltered.
\]

\subsection{Concrete Numerical Example}\label{subsec:numerical}

Let $\alpha = 0.3$ (probability of transitioning $G \to B$), $\beta = 0.5$ (probability of transitioning $B \to G$), $x = 0.3$, $y = 0.4$, so $x + y = 0.7 < 1$ (supermodular).

\medskip
\noindent\textbf{Stationary distribution:}
\[
    \pi(G) = \frac{0.5}{0.3 + 0.5} = \frac{5}{8} = 0.625, \qquad \pi(B) = \frac{0.3}{0.8} = 0.375.
\]

\noindent\textbf{Lifted stationary distribution:}
\[
\begin{aligned}
    \tilde\rho(G,G) &= 0.625 \times 0.7 = 0.4375, \\
    \tilde\rho(G,B) &= 0.375 \times 0.5 = 0.1875, \\
    \tilde\rho(B,G) &= 0.625 \times 0.3 = 0.1875, \\
    \tilde\rho(B,B) &= 0.375 \times 0.5 = 0.1875.
\end{aligned}
\]

\noindent\textbf{Commitment payoff (i.i.d.\ benchmark):} Under $s_1^*(G) = A$, $s_1^*(B) = F$:
\[
    V(s_1^*) = \pi(G) \cdot u_1(G, A) + \pi(B) \cdot u_1(B, F) = 0.625 \times 1 + 0.375 \times 0 = 0.625.
\]

\noindent\textbf{Comparison with i.i.d.:} If the state were i.i.d.\ with $\Prob(G) = 0.625$, the Stackelberg payoff would be identical ($p = 0.625$). The difference is in the \emph{dynamics}: with persistence ($\alpha = 0.3$), attacks come in clusters. The signal process $\{y_{1,t}\}$ exhibits autocorrelation (runs of ``Fight'' and ``Acquiesce'' actions), which provides an \textbf{additional identification channel} beyond marginal frequencies. This makes the confound-defeating condition \emph{easier} to verify in the supermodular case, although (as Section~\ref{sec:belief_robust} shows) the resulting payoff bound may differ from the i.i.d.\ case when belief-robustness fails.

\noindent\textbf{KL bound:} If $\mu_0(\omega_{s_1^*}) = 0.01$ and $\eta = 0.1$:
\[
    \bar{T}(0.1, \mu_0) = \frac{-2 \log(0.01)}{0.01} = \frac{2 \times 4.605}{0.01} = 921 \text{ periods}.
\]
This bound is \textbf{identical} to what it would be in the i.i.d.\ case with the same prior.

\subsection{The Overestimation Gap}

\begin{center}
\renewcommand{\arraystretch}{1.3}
\begin{tabular}{@{}lcc@{}}
\toprule
\textbf{Scenario} & \textbf{LR Average Payoff} & \textbf{Assumption} \\
\midrule
Stationary beliefs (i.i.d.\ assumption) & \PayoffStationary & $\mu = \pi(G)$ always \\
Filtered beliefs (reality) & \PayoffFiltered & $\mu = F(G|\theta_t)$ \\
\midrule
\textbf{Overestimation} & \textbf{\PayoffOverestimation\%} & \\
\bottomrule
\end{tabular}
\end{center}

The overestimation arises because the i.i.d.\ analysis assumes SR always faces belief $\pi(G) = \BasePiG > \SRThreshold$, so SR always cooperates. In reality, SR defects in bad states (where $F(G|B) = \SRBeliefAfterB < \SRThreshold$), reducing the LR payoff by \PayoffOverestimation\%.

\subsection{Limiting Cases}

\begin{center}
\renewcommand{\arraystretch}{1.3}
\begin{tabular}{@{}llll@{}}
\toprule
\textbf{Regime} & \textbf{Mixing} & \textbf{Stackelberg payoff} & \textbf{Behavior} \\
\midrule
Fast mixing ($\alpha, \beta$ large) & $\tau_{\mathrm{mix}}$ small & $V = \frac{\beta}{\alpha+\beta}$ (cf.\ $p$ in Luo--Wolitzky) & Recovers LW Prop.~1 \\
Moderate persistence & $\tau_{\mathrm{mix}}$ moderate & $V_{\mathrm{Markov}} \leq \frac{\beta}{\alpha+\beta}$ & \textbf{New result} \\
Near-perfect persistence ($\alpha, \beta \to 0$) & $\tau_{\mathrm{mix}} \to \infty$ & $V \to \pi_0(G)$ & Weakens toward Pei \\
\bottomrule
\end{tabular}
\end{center}

In the fast-mixing regime, the filtering beliefs $F(\cdot|\theta)$ are close to $\pi$, so belief-robustness holds generically and $V_{\mathrm{Markov}} \approx V(s_1^*)$. In the moderate-persistence regime, the gap between $V_{\mathrm{Markov}}$ and $V(s_1^*)$ depends on whether the SR threshold falls in the danger zone $[\beta, 1-\alpha]$. In the near-perfect-persistence regime, the framework degrades as mixing time diverges, and Pei's (2020) different approach becomes necessary.

\subsection{Comparison Table}

\begin{table}[ht]
\centering
\renewcommand{\arraystretch}{1.3}
\begin{tabular}{@{}lccc@{}}
\toprule
\textbf{Quantity} & \textbf{i.i.d.} & \textbf{Markov (belief-robust)} & \textbf{Markov (general)} \\
\midrule
SR belief about $\theta_{t+1}$ & $\pi$ & $\pi$ & $F(\cdot|\theta_t)$ \\
SR behavior & Static & Static & State-contingent \\
Commitment payoff & $V(s_1^*)$ & $V(s_1^*)$ & $V_{\mathrm{Markov}} \leq V(s_1^*)$ \\
Gap from i.i.d. & 0 & 0 & $\BeliefGapFormula$ \\
\bottomrule
\end{tabular}
\caption{Summary of the three regimes for the deterrence game.}
\label{tab:comparison}
\end{table}

\subsection{Figures}

\begin{figure}[ht]
\centering
\includegraphics[width=0.85\textwidth]{figures/fig_payoff_gap.png}
\caption{LR payoff comparison: stationary belief assumption gives \PayoffStationary{} vs.\ filtered belief reality of \PayoffFiltered, a \PayoffOverestimation\% overestimation. The gap is entirely explained by SR defection in bad states.}
\label{fig:payoff_gap}
\end{figure}

\begin{figure}[ht]
\centering
\includegraphics[width=0.85\textwidth]{figures/fig_nash_dynamics.png}
\caption{Belief trajectory crossing the BR threshold $\mu^* = \SRThreshold$. The SR player's belief $F(G|\theta_t)$ oscillates between \SRBeliefAfterG{} (after $G$) and \SRBeliefAfterB{} (after $B$), crossing $\mu^*$ with each state transition. Disagreement rate: \SRDisagreement\%.}
\label{fig:nash_dynamics}
\end{figure}

\begin{figure}[ht]
\centering
\includegraphics[width=0.85\textwidth]{figures/fig_belief_gap.png}
\caption{Analytical belief gap $2\alpha\beta|1-\alpha-\beta|/(\alpha+\beta)^2$ across the $(\alpha,\beta)$ parameter space. The gap equals zero along the anti-diagonal $\alpha+\beta=1$ (i.i.d.\ line) and increases with persistence $|1-\alpha-\beta|$.}
\label{fig:belief_gap}
\end{figure}

% ================================================================
\section{Interpolation Between i.i.d.\ and Persistent}\label{sec:interpolation}

Our framework provides a continuous interpolation between the i.i.d.\ setting (Luo--Wolitzky 2024) and increasingly persistent Markov states.

\subsection{The Interpolation Landscape}

The interpolation is two-dimensional in the $(\alpha, \beta)$ parameter space:

\begin{itemize}
    \item \textbf{Along $\alpha+\beta = 1$ (i.i.d.\ line):} $F(\cdot|\theta) = \pi(\cdot)$ for all $\theta$, so $V_{\mathrm{Markov}} = V(s_1^*)$. No gap.
    \item \textbf{Away from $\alpha+\beta = 1$:} $V_{\mathrm{Markov}} < V(s_1^*)$, with gap increasing as $|1-\alpha-\beta|$ grows.
    \item \textbf{Corners $(\alpha,\beta) \to (0,0)$ (near-perfect persistence):} $V_{\mathrm{Markov}} \to$ state-by-state payoff; gap maximized.
\end{itemize}

The mean TV distance $\|F(\cdot|\theta) - \pi\|$ averaged over the parameter space is \TVMean{} (Figure~\ref{fig:belief_deviation}), confirming that belief deviation from the stationary distribution is the norm, not the exception, for Markov states.

\begin{figure}[ht]
\centering
\includegraphics[width=0.85\textwidth]{figures/fig_belief_deviation.png}
\caption{Mean TV distance $\|F(\cdot|\theta) - \pi\|$ across the $(\alpha,\beta)$ parameter space. The deviation vanishes along $\alpha+\beta=1$ (i.i.d.) and increases toward the corners (high persistence). Average: \TVMean.}
\label{fig:belief_deviation}
\end{figure}

\subsection{Recovery of Existing Results}

\begin{description}
    \item[i.i.d.\ (Luo--Wolitzky 2024):] $F(\cdot|\theta) = \pi(\cdot)$ for all $\theta$. Theorems~\ref{thm:belief_robust} and~\ref{thm:general} both reduce to the original Theorem~1 with $V_{\mathrm{Markov}} = V(s_1^*)$.

    \item[Perfectly persistent (Pei 2020):] $F(\cdot|\theta) = \delta_\theta$. The chain is not ergodic, so our framework does not directly apply. As $\alpha, \beta \to 0$, the gap $V(s_1^*) - V_{\mathrm{Markov}}$ diverges (the rate of convergence in $\delta$ degrades), and Pei's different approach is needed.
\end{description}

\subsection{The Cost of Persistence}

The gap $V(s_1^*) - V_{\mathrm{Markov}}$ is a new economic object: the \emph{cost of persistence in reputation games}. It quantifies how much the LR player loses because state persistence causes SR to adjust behavior state-by-state.

For the deterrence game with $\mu^* = \SRThreshold$:
\begin{itemize}
    \item The cost is $\PayoffStationary - \PayoffFiltered = \PayoffGapAbsolute$ (\PayoffOverestimation\% of the i.i.d.\ payoff).
    \item The cost is increasing in $|1-\alpha-\beta|$ (persistence).
    \item The cost vanishes as $\alpha+\beta \to 1$ (i.i.d.\ limit).
\end{itemize}

This provides a direct link between the dynamics of the economic environment and the value of reputation.

% ================================================================
\section{Methodology: The Correction Process}\label{sec:methodology}

This section documents the correction methodology as a case study in AI-assisted mathematical research. The story of ``AI generated a plausible-looking proof that was wrong in specific ways, identified computationally and fixed'' is itself a contribution to the AI-for-math literature.

\subsection{The Correction Pipeline}

The revision followed a four-stage pipeline:

\medskip
\noindent\textbf{Stage 1: AI Conjecture.}
Multiple specialized AI agents (paper parsing, proof verification, example computation) developed the initial extension claim in under 5 hours. The claim: Theorem~1 extends to Markov states with no correction needed, using the lifted state $\tilde\theta_t = (\theta_t, \theta_{t-1})$.

\medskip
\noindent\textbf{Stage 2: Expert Critique.}
Daniel Luo identified the fatal flaw: i.i.d.\ disciplines SR information sets. With Markov states and state-revealing strategies, SR beliefs are $F(\cdot|\theta_t)$, not $\pi$. The Nash correspondence $B(s_1^*, \mu)$ changes period-to-period.

\medskip
\noindent\textbf{Stage 3: Computational Testing.}
Seven analysis modules (21 scripts, 40+ figures) systematically tested each claim:
\begin{itemize}
    \item SA1--SA2: Belief deviation quantification (confirmed: mean TV = \TVMean, gap = \BeliefGapBaseline)
    \item SA3: KL bound verification (confirmed: extends verbatim)
    \item SA4: Filter stability (confirmed: $r > \FilterCorrelation$)
    \item SA5: OT robustness (confirmed: \OTStabilityPct\% stability)
    \item SA6: Nash dynamics (identified: \SRDisagreement\% disagreement, \PayoffOverestimation\% overestimation)
    \item SA7: Monotonicity (confirmed: \SupermodFraction/\SupermodTotal{} orderings for $\theta_t$-only payoffs)
\end{itemize}

\medskip
\noindent\textbf{Stage 4: Corrected Results.}
The computational evidence guided two corrected theorems: Theorem~\ref{thm:belief_robust} (belief-robust, exact) and Theorem~\ref{thm:general} (general, corrected bound). Both are supported by computational verification.

\subsection{Lessons for AI-Assisted Research}

\begin{enumerate}[label=(\arabic*)]
    \item \textbf{AI proof sketches require adversarial testing.} The initial proof was ``nicely written nonsense''---aesthetically convincing but mathematically flawed in specific ways.
    \item \textbf{Computational testing can precisely localize errors.} The seven analysis modules identified exactly which claims survive and which fail, enabling targeted correction rather than wholesale rejection.
    \item \textbf{The correction is more interesting than the original claim.} Belief-robustness and $V_{\mathrm{Markov}}$ are genuinely new economic concepts that the i.i.d.\ framework cannot capture.
    \item \textbf{Transparency enables scientific progress.} All agent transcripts, analysis scripts, and figures are available in the project repository.
\end{enumerate}

% ================================================================
\section{Discussion and Open Questions}\label{sec:discussion}

\subsection{Summary}

We have shown that extending Marginal Reputation to Markov states is more subtle than initially claimed, requiring a distinction between two regimes. In belief-robust games---where the short-run player's best-response set does not depend on the revealed state---the i.i.d.\ commitment payoff bound $V(s_1^*)$ holds exactly (Theorem~\ref{thm:belief_robust}). In general games, the corrected Markov commitment payoff $V_{\mathrm{Markov}}(s_1^*) \leq V(s_1^*)$ provides the appropriate bound (Theorem~\ref{thm:general}). The gap between the two, $V(s_1^*) - V_{\mathrm{Markov}}$, is a new economic object---the cost of persistence---quantifying how state persistence affects reputation-building by enabling the short-run player to condition behavior on the revealed state.

\subsection{Open Questions}

Several directions merit further investigation.

The \textbf{belief-robustness landscape} remains incompletely characterized. For the deterrence game, Proposition~\ref{prop:br_condition} gives a clean criterion in terms of the SR threshold and the filtering beliefs. For general games with richer action spaces, the geometry of the belief-robustness condition may be more complex. An important question is whether belief-robustness is generic or exceptional within economically relevant classes of games.

The \textbf{computation of $V_{\mathrm{Markov}}$} is straightforward for the two-state deterrence game but may be challenging for general supermodular games, where it requires solving state-contingent Nash equilibria for each $\theta \in \Theta$ and integrating over the ergodic distribution. Closed-form expressions or tight bounds for broad classes of games would make Theorem~\ref{thm:general} more practically useful.

A natural question concerns \textbf{$\varepsilon$-perturbed strategies}. If the commitment type plays $s_1^\varepsilon(\theta) = (1-\varepsilon) s_1^*(\theta) + \varepsilon \cdot \text{uniform}$ for small $\varepsilon > 0$, the strategy is no longer state-revealing, and filter stability (SA4) suggests that beliefs may converge to the stationary distribution. Whether $V_{\mathrm{Markov}}(s_1^\varepsilon) \to V(s_1^*)$ as $\varepsilon \to 0$, uniformly in other parameters, would provide a ``smoothing'' route to the full bound that circumvents the belief-robustness requirement.

The \textbf{rate of convergence}---how fast $\underline{U}_1(\delta) \to V_{\mathrm{Markov}}$ as $\delta \to 1$---is not addressed by our analysis. The rate likely depends on both the mixing time $\tau_{\mathrm{mix}}$ and the belief-robustness margin $\min_\theta |F(G|\theta) - \mu^*|$, and characterizing this dependence would be valuable for applications.

Extensions to \textbf{continuous state spaces}, where $\Theta$ is infinite (e.g., $\R$), would require the OT problem to be formulated in infinite dimensions. The result should extend under compactness and continuity conditions, but care is needed with the cyclical monotonicity characterization.

Finally, the case of \textbf{non-revealing strategies}---commitment strategies with full support on $A_1$ for all $\theta$, so that the signal does not perfectly identify the state---deserves separate treatment. For such strategies, filter stability suggests that the belief dynamics may be more benign than in the state-revealing case, and it is plausible that the full bound $V(s_1^*)$ is recoverable without the belief-robustness condition. A related notion of \textbf{approximate belief-robustness}, defined as $\sup_{\theta,\theta'} d_H(B(s_1^*, F(\cdot|\theta)), B(s_1^*, F(\cdot|\theta'))) \leq \varepsilon$, may yield a bound of the form $V_{\mathrm{Markov}} \geq V(s_1^*) - C\varepsilon$ for some constant $C$.

\subsection{Conclusion}

Persistence in states creates a fundamental tension between the long-run player's reputation-building and the short-run player's state-learning. When the Stackelberg strategy reveals the state, short-run players learn the state sequence and adjust their behavior accordingly, reducing the long-run player's commitment payoff by exactly the amount of behavioral adjustment. This tension---invisible in the i.i.d.\ framework and quantified here for the first time---is a genuinely new economic insight that enriches the marginal reputation framework. The concepts of belief-robustness and the Markov commitment payoff provide the tools to analyze reputation in dynamic environments where states exhibit persistence, answering the open question posed by Luo \& Wolitzky (2024, footnote~9).


% ========== Appendices ==========
\appendix
% ================================================================
\section{KL Chain Rule Verification}\label{app:kl}

For completeness, we verify that the chain rule for KL divergence holds for general stochastic processes---the key technical fact ensuring the counting bound (Lemma~\ref{lem:KL}) requires no modification for Markov states.

\subsection{The Chain Rule for KL Divergence}

\begin{lemma}\label{lem:kl_chain}
Let $P$ and $Q$ be probability measures on $(X_0, X_1, \ldots, X_{T-1})$. Then:
\[
    \KL(P \| Q) = \sum_{t=0}^{T-1} \E_P\!\left[\KL\!\left(P(X_t | X_0, \ldots, X_{t-1}) \;\big\|\; Q(X_t | X_0, \ldots, X_{t-1})\right)\right].
\]
\end{lemma}

\begin{proof}
By the chain rule for probability distributions:
\begin{align}
    \KL(P \| Q) &= \E_P\!\left[\log \frac{P(X_0, \ldots, X_{T-1})}{Q(X_0, \ldots, X_{T-1})}\right] \\
    &= \E_P\!\left[\log \prod_{t=0}^{T-1} \frac{P(X_t | X_0, \ldots, X_{t-1})}{Q(X_t | X_0, \ldots, X_{t-1})}\right] \\
    &= \sum_{t=0}^{T-1} \E_P\!\left[\log \frac{P(X_t | X_0, \ldots, X_{t-1})}{Q(X_t | X_0, \ldots, X_{t-1})}\right] \\
    &= \sum_{t=0}^{T-1} \E_P\!\left[\KL(P(X_t | X_0, \ldots, X_{t-1}) \| Q(X_t | X_0, \ldots, X_{t-1}))\right].
\end{align}
No independence assumption is used anywhere. The decomposition follows purely from the chain rule for joint distributions $P(X_0, \ldots, X_{T-1}) = \prod_t P(X_t | X_{<t})$ and linearity of expectation.
\end{proof}

\subsection{Filter Stability for Ergodic HMMs}

\begin{proposition}[Filter Stability; cf.\ Chigansky \& Liptser 2004]\label{prop:filter_stability}
Let $(\theta_t)$ be an ergodic Markov chain on finite $\Theta$ with transition kernel $F$, observed through a channel $y_t \sim g(\cdot | \theta_t)$ (where $g$ has full support). Then the filter $\pi_t(\cdot) = \Prob(\theta_t = \cdot | y_0, \ldots, y_t)$ satisfies:
\[
    \sup_{\pi_0, \pi_0'} \|\pi_t - \pi_t'\| \;\leq\; C \cdot \lambda^t
\]
for some $C > 0$ and $\lambda \in (0,1)$, where $\pi_t$ and $\pi_t'$ are filters starting from priors $\pi_0$ and $\pi_0'$ respectively.
\end{proposition}

This ensures that the initial condition of the Markov chain is ``forgotten'' exponentially fast, so the per-period signal distribution converges to a limit determined by the observation process alone---the key property used in Step~3 of the proof.

\subsection{Monte Carlo Verification}

\begin{figure}[ht]
\centering
\includegraphics[width=0.85\textwidth]{figures/fig_kl_bound.png}
\caption{KL counting bound comparison: Markov vs.\ i.i.d.\ settings. Monte Carlo simulation with $N=\KLMonteCarloN$ runs and $T=\KLMonteCarloPeriods$ periods confirms the bound $\bar{T}(\eta, \mu_0) = -2\log\mu_0(\omega_{s_1^*})/\eta^2$ is valid and nearly identical in both settings.}
\label{fig:kl_bound}
\end{figure}

% ================================================================
\section{Computational Framework}\label{app:computational}

This appendix documents the computational analysis that informed the revision. All scripts and figures are available in the project repository.

\subsection{Analysis Modules}

Seven analysis modules (SA1--SA7) systematically tested each claim from the initial draft:

\begin{center}
\renewcommand{\arraystretch}{1.3}
\begin{tabular}{@{}clcc@{}}
\toprule
\textbf{Module} & \textbf{Focus} & \textbf{Scripts} & \textbf{Key Finding} \\
\midrule
SA1 & Belief deviation & 3 & Mean TV = \TVMean \\
SA2 & State-revealing analysis & 3 & Gap = \BeliefGapBaseline{} (analytical) \\
SA3 & KL bound verification & 3 & Extends verbatim \\
SA4 & Filter stability & 3 & $r > \FilterCorrelation$ \\
SA5 & OT robustness & 3 & \OTStabilityPct\% stable \\
SA6 & Nash dynamics & 3 & \PayoffOverestimation\% overestimation \\
SA7 & Monotonicity & 3 & \SupermodFraction/\SupermodTotal \\
\bottomrule
\end{tabular}
\end{center}

Total: 21 scripts, 8 diagnostic figures. Runtime: approximately 8 minutes on a standard laptop. No GPU required.

\subsection{Reproducibility}

The analysis pipeline is fully reproducible:
\begin{enumerate}[label=(\arabic*)]
    \item Dependencies: \texttt{numpy}, \texttt{scipy}, \texttt{matplotlib}, \texttt{seaborn} (Python 3.8+).
    \item Entry point: \texttt{scripts/generate\_paper.sh} runs the full pipeline (analysis $\to$ statistics $\to$ PDF).
    \item Statistics are auto-generated: \texttt{scripts/extract\_stats.py} produces \texttt{stats.tex}, ensuring the paper always reflects the latest computational results.
\end{enumerate}

\subsection{Repository Structure}

The full project repository contains all artifacts from both phases of the research:

\begin{small}
\begin{verbatim}
mathTest/
+-- revisedTexPaper/          # This paper
|   +-- main.tex              # Master file (inputs sections)
|   +-- stats.tex             # Auto-generated statistics macros
|   +-- sections/             # 12 modular .tex files
|   +-- figures/              # 8 essential figures
|   +-- scripts/              # 7 analysis + 3 automation scripts
|   +-- response_letter.tex   # Point-by-point response
|   +-- build.sh              # Compile paper
+-- Agent1206_workspace/      # Computational testing framework
|   +-- agent_framework.py    # Reusable Agent class
|   +-- orchestrator.py       # Runs all 21 scripts
|   +-- shared/markov_utils.py
|   +-- SA{1-7}_*/            # 7 subagent directories
|   |   +-- task.md / report.md
|   |   +-- SSA*_*/           # 3 sub-subagent dirs each
|   +-- reports/final_report.md
+-- OPReview/                 # Daniel Luo's original feedback
+-- AgentReports/             # Phase 1 agent reports
|   +-- Agent1206plan.md      # Review + testing plan
+-- promptHistory/            # All 15 agent transcripts
+-- texPaper/                 # Original (uncorrected) paper
+-- index.html                # Interactive web demo (9 tabs)
+-- author-review.html        # Critique visualization
+-- revision.html             # Revision explorer
+-- revision_summary.md       # One-page summary
\end{verbatim}
\end{small}

\noindent Total artifacts: 15 agent transcripts, 28 task/report files, 21 analysis scripts (full) + 7 consolidated scripts, 40+ figures, 2 compiled PDFs, 9 interactive HTML pages.

\subsection{Additional Figures}

\begin{figure}[ht]
\centering
\includegraphics[width=0.85\textwidth]{figures/fig_filter_stability.png}
\caption{Filter forgetting rate $\lambda$ vs.\ $|1-\alpha-\beta|$ across a $\FilterGridSize\times\FilterGridSize$ parameter grid. The fitted correlation exceeds $r = \FilterCorrelation$, confirming exponential forgetting with rate proportional to the chain's second eigenvalue. More informative signals accelerate forgetting.}
\label{fig:filter_stability}
\end{figure}

\begin{figure}[ht]
\centering
\includegraphics[width=0.85\textwidth]{figures/fig_ot_robustness.png}
\caption{OT support stability margin across the $(\alpha,\beta)$ parameter space. The co-monotone coupling $(G\to A, B\to F)$ remains the OT solution for perturbations up to $\varepsilon = \OTStabilityMargin$, with stability margin $\geq \OTStabilityMargin$ in \OTStabilityPct\% of the parameter space.}
\label{fig:ot_robustness}
\end{figure}


% ========== References ==========
\section*{References}
\addcontentsline{toc}{section}{References}

\begin{enumerate}[label={[\arabic*]}, leftmargin=*, itemsep=3pt]
\item Chigansky, P.\ and R.\ Liptser (2004). ``Stability of nonlinear filters in nonmixing case.'' \textit{Annals of Applied Probability}, 14(4): 2038--2056.

\item Cover, T.\ M.\ and J.\ A.\ Thomas (2006). \textit{Elements of Information Theory}, 2nd ed. Wiley.

\item Del Moral, P.\ (2004). \textit{Feynman--Kac Formulae: Genealogical and Interacting Particle Systems with Applications}. Springer.

\item Fudenberg, D.\ and D.\ K.\ Levine (1992). ``Maintaining a Reputation When Strategies Are Imperfectly Observed.'' \textit{Review of Economic Studies}, 59(3): 561--579.

\item Gossner, O.\ (2011). ``Simple Bounds on the Value of a Reputation.'' \textit{Econometrica}, 79(5): 1627--1651.

\item Luo, D.\ and A.\ Wolitzky (2024). ``Marginal Reputation.'' MIT Department of Economics Working Paper.

\item Mailath, G.\ J.\ and L.\ Samuelson (2006). \textit{Repeated Games and Reputations: Long-Run Relationships}. Oxford University Press.

\item Pei, H.\ (2020). ``Reputation Effects under Interdependent Values.'' \textit{Econometrica}, 88(5): 2175--2202.

\item Rochet, J.-C.\ (1987). ``A Necessary and Sufficient Condition for Rationalizability in a Quasi-linear Context.'' \textit{Journal of Mathematical Economics}, 16(2): 191--200.

\item Santambrogio, F.\ (2015). ``Optimal Transport for Applied Mathematicians.'' Birkh\"auser.
\end{enumerate}

\end{document}
