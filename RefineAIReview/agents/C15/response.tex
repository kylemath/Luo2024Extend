\subsection*{C15: Confound-Defeating Under Persistence (Section~8.6)}

\textbf{Reviewer comment:}
\begin{quote}
``Persistence thus strengthens identification, making confound-defeating conditions easier to satisfy in the supermodular case.''---Section~6 shows the supermodularity/monotonicity characterization of confound-defeating is UNCHANGED from i.i.d. In what sense is it ``easier to satisfy''?
\end{quote}

\textbf{Response:}
The reviewer is right to flag the imprecision. The mathematical characterization of confound-defeating is indeed \emph{unchanged}: in the supermodular case, confound-defeating is equivalent to monotonicity of $s_1^*$ on $\tilde\Theta$ (Proposition~6.1), which under $\theta_t$-only payoffs reduces to the same monotonicity condition as in the i.i.d.\ case.

What we intended to convey---and what we have now stated precisely---is that persistence provides a richer \emph{identification channel} for empirically distinguishing strategies, even though the formal condition is the same. Under persistence, a conditional strategy (``fight when detecting an attack'') generates autocorrelated action sequences that are statistically distinguishable from an unconditional strategy (``fight 50\% of the time''), even when per-period frequencies match. This does not make the mathematical condition weaker or easier to satisfy; rather, it means the data generated by persistent states carries more information about whether the condition holds.

The revised text reads: ``Persistence thus provides a richer identification channel: the temporal patterns induced by state-contingent strategies are more statistically distinguishable, facilitating empirical verification of the confound-defeating condition, even though the mathematical characterization (Proposition~6.1) is unchanged from the i.i.d.\ case.''
