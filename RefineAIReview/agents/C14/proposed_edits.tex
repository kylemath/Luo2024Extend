% ---- Proposed Edits for C14: Ambiguous definition of best-response set in Lemma 5.8 ----
% File: revisedTexPaper/sections/sec_05_proof.tex

% ============================================================
% EDIT 1: Add clarifying remark before Lemma 5.8 (lem:combining)
% ============================================================

% --- OLD (lines 174-176) ---
\subsection{Step 4: Lemma 4 --- Combining the Pieces}\label{subsec:step4}

\begin{lemma}[Extension of Lemma 4]\label{lem:combining}
% --- END OLD ---

% --- NEW ---
\subsection{Step 4: Lemma 4 --- Combining the Pieces}\label{subsec:step4}

\begin{remark}[Notation: Confirmed vs.\ State-Contingent Best-Response Sets]\label{rem:B_hat_notation}
The set $\hat{B}_{\xi}(s_1^*)$ below denotes the \emph{$\xi$-confirmed best-response set} of Luo--Wolitzky (2024, Definition~3): the set of short-run best-response profiles $(\alpha_0, \alpha_2)$ that remain optimal when the posterior over types concentrates within total variation $\xi$ of $\{\omega^R, \omega_{s_1^*}\}$. This is distinct from the \emph{state-contingent} Nash correspondence $B(s_1^*, F(\cdot|\theta))$ introduced in Definition~\ref{def:V_markov}, which captures short-run behavior under filtering beliefs in state~$\theta$. The confirmed set $\hat{B}_{\xi}(s_1^*)$ governs behavior conditional on the posterior being near the commitment type; the state-contingent set $B(s_1^*, F(\cdot|\theta))$ governs behavior conditional on the revealed state. Under belief-robustness, $B(s_1^*, F(\cdot|\theta))$ is constant across $\theta$ and the distinction is immaterial for Step~5.
\end{remark}

\begin{lemma}[Extension of Lemma 4]\label{lem:combining}
% --- END NEW ---
