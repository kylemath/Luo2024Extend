% C21 Response: Inconsistent payoffs and Stackelberg strategy

\noindent\textbf{Comment C21.} \emph{Only D-row payoffs are shown; acquiesce dominates fight in state $B$ looking at these alone. Full C-row makes $s_1^*(B)=F$ correct but isn't displayed.}

\medskip
\noindent\textbf{Response.} The reviewer correctly observes that the displayed D-conditional payoffs alone make $s_1^*(B) = F$ appear suboptimal, since $u_1(B, A, D) = 0.4 > u_1(B, F, D) = 0$. We have added a clarifying remark and a display of the full payoff structure.

The Stackelberg commitment strategy $s_1^*$ is not chosen to maximize single-period payoffs against defection, but to maximize the \emph{reputation payoff}---the expected payoff when SR eventually best-responds to the commitment type. The key mechanism is that $s_1^*(G) = A$, $s_1^*(B) = F$ is \textbf{state-contingent}, allowing SR to infer the state from LR's action. When SR cooperates in response to favorable beliefs, LR captures the cooperation surplus in good states. The short-run cost of fighting in bad states ($u_1(B, F, D) = 0$ vs.\ $u_1(B, A, D) = 0.4$) is offset by the reputation benefit: SR cooperation in good states yields $u_1(G, A, C) > u_1(G, A, D) = 1$.

We have added a display of the full payoff matrix (both C and D columns) after the D-conditional payoffs in Section~7.1, with a brief note explaining that $s_1^*$ maximizes the reputation payoff, not the per-period payoff against defection. This allows the reader to verify directly that $s_1^*(B) = F$ is optimal in the full game.
