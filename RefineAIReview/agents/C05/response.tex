\subsection*{C05: Sufficient Condition for Supermodularity (Remark~5.4)}

\textbf{Reviewer comment:}
\begin{quote}
``$g(\theta_t, a_1, h_t)$ is supermodular in $(\theta_t, a_1)$ whenever $V_{\mathrm{cont}}$ is increasing in $\theta_t$ for each $a_1$''---Monotonicity in $\theta_t$ for each $a_1$ does NOT imply increasing differences in $(\theta, a_1)$. Supermodularity requires $V_{\mathrm{cont}}(\theta_H, a_H) - V_{\mathrm{cont}}(\theta_H, a_L) \geq V_{\mathrm{cont}}(\theta_L, a_H) - V_{\mathrm{cont}}(\theta_L, a_L)$, which is strictly stronger than monotonicity.
\end{quote}

\textbf{Response:}
The reviewer is correct that monotonicity does not imply supermodularity (increasing differences) in general, and we apologize for the imprecise statement. However, the conclusion of the remark is recoverable via a \emph{stronger} argument that we should have stated.

In the paper's model, state transitions are \textbf{exogenous}: the chain $\theta_t$ evolves according to $F(\cdot|\theta_{t-1})$ independently of actions. Consequently, the continuation value conditional on the current state satisfies:
\[
    V_{\mathrm{cont}}(\theta_t) = \sum_{\theta'} F(\theta'|\theta_t)\, V(\theta'),
\]
which depends on $\theta_t$ but \textbf{not on $a_1$}. The perturbation $g(\theta_t, a_1, h_t) = \delta V_{\mathrm{cont}}(\theta_t)$ is therefore a function of $\theta_t$ alone. Adding a function that depends only on $\theta$ (not on $a_1$) to a supermodular objective $u_1(\tilde\theta, a_1, \alpha_2)$ preserves supermodularity trivially: the $g$ terms cancel in the increasing-differences comparison, leaving the original increasing differences of $u_1$ unchanged.

We have corrected Remark~5.4 to state this argument directly rather than invoking the incorrect monotonicity-implies-supermodularity claim. We also note the caveat that if the continuation value depended on $a_1$ (e.g., through signal effects in a model with action-dependent transitions), the correct condition would be increasing differences of $g$ in $(\theta_t, a_1)$, not mere monotonicity.
