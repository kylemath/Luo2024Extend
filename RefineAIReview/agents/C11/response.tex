% ============================================================
% Response to Comment C11: Incorrect inference from filter stability
% ============================================================

\subsection*{Comment C11 (HIGH): Incorrect Inference from Filter Stability}

\textbf{Reviewer comment.} Same core issue as C07, also in Section~10.2. ``Filter stability (SA4) suggests that beliefs may converge to the stationary distribution'' conflates two distinct notions.

\medskip
\textbf{Disposition: Fully accepted.}

\medskip
\textbf{Response.} We agree that the inference is invalid. Filter stability is a statement about the \emph{contraction rate} of the filtering operator: two filters initialized at different priors converge to each other. The \emph{fixed point} to which they converge is determined by the observation channel, not by the stationary distribution of the hidden chain. For an informative channel (small $\varepsilon$), the filter's fixed point tracks the current hidden state $\theta_t$ rather than the unconditional distribution~$\pi$.

The structural analogy is precise: filter stability tells us that a contraction mapping has a unique fixed point, but does not determine where that fixed point lies. The inference ``filter stability $\Rightarrow$ beliefs converge to $\pi$'' would require the additional premise that $\pi$ is the filter's fixed point, which holds only when the observation channel is uninformative.

This comment targets the same passage as C07, and the correction described in our response to C07 resolves both issues simultaneously. The revised paragraph:
\begin{enumerate}
    \item Correctly states that filter stability guarantees prior forgetting, not convergence to~$\pi$;
    \item Explains that for small~$\varepsilon$, the filter tracks~$\theta_t$ rather than settling at~$\pi$;
    \item Includes a footnote explicitly distinguishing filter stability, chain mixing, and belief convergence to~$\pi$.
\end{enumerate}

We additionally note that the revised text avoids the invalid inference by reframing the benefit of $\varepsilon$-perturbation: the strategy degrades state revelation, reducing the effective belief gap, rather than causing beliefs to converge to~$\pi$.

\medskip
\textbf{Changes made:}
\begin{itemize}
    \item Section~10.2: Same revision as C07 (rewritten $\varepsilon$-perturbed strategies paragraph); the joint fix addresses both the conceptual error (C07) and the invalid inference (C11).
\end{itemize}
