% C12 Response: Contradiction between Figure 6 caption and content

\noindent\textbf{Comment C12.} \emph{Caption says distributions are ``nearly identical'' but means are $8.1$ vs $12.7$. Both are far below $921$, so the bound holds, but ``nearly identical'' overstates.}

\medskip
\noindent\textbf{Response.} The reviewer is correct that ``nearly identical'' mischaracterizes the relationship. The i.i.d.\ and Markov distributions of distinguishing-period counts have means of $8.1$ and $12.7$ respectively---a $57\%$ relative difference that is not ``nearly identical.''

The important point is that both means are far below the analytical bound $\bar{T}(\eta, \mu_0) = 921$: the i.i.d.\ mean is $0.9\%$ and the Markov mean is $1.4\%$ of the bound. The KL counting bound holds with large margin in both settings, which is the substantive mathematical claim.

We have revised the figure caption and the corresponding remark in Section~5.2 to state:

\begin{quote}
``\ldots confirms that the bound $\bar{T}(\eta, \mu_0) = 921$ is valid in both settings, with empirical means well below the bound (i.i.d.: $\statIidMeanCount$; Markov: $\statMarkovMeanCount$). The Markov process produces moderately more distinguishing periods due to autocorrelation, but both remain far below the theoretical bound.''
\end{quote}

This accurately describes the computational evidence without overstating distributional similarity.
