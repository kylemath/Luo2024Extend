% ================================================================
% Manager D: Consolidated Response — Notation, Clarity, and Structural Issues
% Comments: C10, C14, C16, C17, C20
% ================================================================

\subsection*{D. Notation, Clarity, and Structural Issues}

We address five comments concerning notation clarity, missing content, and structural issues. All are resolved through text additions and revisions; no mathematical results are affected.

\bigskip

% ---- C10 ----
\noindent\textbf{Comment C10 (HIGH):} \emph{``A.3 Monte Carlo Verification'' appears to be empty (header only). Main text (Section~5.4) relies on it.}

\medskip
\noindent\textbf{Response:} We thank the reviewer for identifying this omission. Appendix~A.3 contained only a figure without explanatory text. We have added a detailed description of the Monte Carlo verification procedure:

\begin{itemize}
    \item \textbf{Setup:} Two parallel processes---a Markov chain with parameters $(\alpha, \beta)$ and an i.i.d.\ process with the same stationary distribution $\pi$---are simulated for $T = 5{,}000$ periods across $N = 500$ independent runs.
    \item \textbf{Methodology:} For each run, Bayesian posteriors are computed and the number of \emph{distinguishing periods} (where the predicted signal distributions under commitment and rational type hypotheses differ by more than $\eta$ in total variation) is recorded.
    \item \textbf{Finding:} The empirical mean distinguishing-period count is $\statIidMeanCount$ (i.i.d.) and $\statMarkovMeanCount$ (Markov), both well below the theoretical bound $\bar{T}(\eta, \mu_0)$, confirming that the KL counting bound requires no mixing-time correction.
\end{itemize}

\noindent\textbf{Edit:} Appendix~A.3 now contains full methodological text before Figure~\ref{fig:kl_bound}. \\
\textbf{Disposition:} Fully accepted.

\bigskip

% ---- C14 ----
\noindent\textbf{Comment C14 (MEDIUM):} \emph{$\hat{B}_{\xi(\eta)}(s_1^*)$ appears without redefinition in the Markov extension. Readers may confuse it with state-dependent $B(s_1^*, F(\cdot|\theta))$.}

\medskip
\noindent\textbf{Response:} We agree that the notation requires disambiguation. We have added Remark~5.6 before Lemma~5.8 clarifying:
\begin{itemize}
    \item $\hat{B}_{\xi}(s_1^*)$ is the \emph{$\xi$-confirmed best-response set} from Luo--Wolitzky (2024, Definition~3): the set of SR best-response profiles that remain optimal when the posterior concentrates within $\xi$ of $\{\omega^R, \omega_{s_1^*}\}$.
    \item $B(s_1^*, F(\cdot|\theta))$ is the \emph{state-contingent Nash correspondence}: SR behavior conditional on the revealed state $\theta$.
    \item Under belief-robustness, $B(s_1^*, F(\cdot|\theta))$ is constant across $\theta$ and the distinction is immaterial.
\end{itemize}

\noindent\textbf{Edit:} New Remark~5.6 added to Section~5.4 of the proof sketch. \\
\textbf{Disposition:} Fully accepted.

\bigskip

% ---- C16 ----
\noindent\textbf{Comment C16 (MEDIUM):} \emph{``Since $s_1^*$ is not behaviorally confounded, any type with the same asymptotic signal distribution must be $s_1^*$ itself'' relies on an implicit identification between asymptotic conditional signal distributions and stationary $p(\alpha_0, s_1, \alpha_2)$. Need to make explicit that filter stability + ergodicity ensure this.}

\medskip
\noindent\textbf{Response:} The reviewer correctly identifies an implicit step that is automatic in the i.i.d.\ case but requires justification under Markov dynamics. We have expanded Part~A of Lemma~5.7 to make the logical chain explicit:
\begin{enumerate}
    \item \textbf{KL bound} (Lemma~5.6) implies per-period signal convergence (process-independent).
    \item \textbf{Filter stability} (Proposition~A.2) ensures the filtering distribution converges exponentially fast to a limit determined by the strategy and transition kernel.
    \item \textbf{Ergodicity} identifies this limit as the unique stationary signal distribution $p_\infty(\alpha_0, s_1, \alpha_2)$.
    \item The \textbf{not-behaviorally-confounded condition} (Definition~4.3), applied to these stationary distributions, yields $\mu_\infty(\{\omega^R, \omega_{s_1^*}\} | h) = 1$.
\end{enumerate}

\noindent\textbf{Edit:} Expanded text in Part~A of Lemma~5.7 (Section~5.3). \\
\textbf{Disposition:} Fully accepted.

\bigskip

% ---- C17 ----
\noindent\textbf{Comment C17 (LOW):} \emph{Section~1.1 says lifted state ``provides a stationary distribution'' which reads as if the original chain lacked one. Actually $\theta_t$ already has $\pi$; lifting creates a type space for Markov private info.}

\medskip
\noindent\textbf{Response:} We agree the phrasing was misleading. The original chain $\theta_t$ possesses a unique stationary distribution $\pi$ by Assumption~1(b). We have revised the sentence to clarify that the lifting encodes Markov private information into a type space $\tilde\Theta$, and that $\tilde\rho$ plays the role of the i.i.d.\ signal distribution in the OT framework.

\noindent\textbf{Edit:} One sentence revised in Section~1.1. \\
\textbf{Disposition:} Fully accepted.

\bigskip

% ---- C20 ----
\noindent\textbf{Comment C20 (MEDIUM):} \emph{Section~9.5 says Stackelberg well-definedness for persuasion games is ``acknowledged as an open question in Section~10.'' But Section~10.2 doesn't mention it.}

\medskip
\noindent\textbf{Response:} The reviewer has identified a genuine cross-reference error. Section~9.5 claims the topic is discussed in Section~10, but Section~10.2 omits it. We have added a new paragraph to Section~10.2:

\begin{quote}
For \textbf{persuasion games}, the Stackelberg strategy is defined via concavification of the sender's value function. Under Markov dynamics, the receiver's prior varies state-by-state through $F(\cdot|\theta)$, and the optimal persuasion scheme may differ for each prior. Whether a single state-independent Stackelberg strategy exists that is simultaneously optimal across all filtering priors remains open. This issue, first identified by Luo~(2026), is specific to games where the commitment strategy solves an optimization over belief distributions.
\end{quote}

\noindent\textbf{Edit:} New paragraph added to Section~10.2 (Open Questions). \\
\textbf{Disposition:} Fully accepted.
